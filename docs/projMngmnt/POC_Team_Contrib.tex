\documentclass{article}

\usepackage{float}
\restylefloat{table}

\usepackage{booktabs}

\title{Team Contributions: POC\\\progname}

\author{\authname}

\date{}

%% Comments

\usepackage{color}

\newif\ifcomments\commentstrue %displays comments
%\newif\ifcomments\commentsfalse %so that comments do not display

\ifcomments
\newcommand{\authornote}[3]{\textcolor{#1}{[#3 ---#2]}}
\newcommand{\todo}[1]{\textcolor{red}{[TODO: #1]}}
\else
\newcommand{\authornote}[3]{}
\newcommand{\todo}[1]{}
\fi

\newcommand{\wss}[1]{\authornote{blue}{SS}{#1}} 
\newcommand{\plt}[1]{\authornote{magenta}{TPLT}{#1}} %For explanation of the template
\newcommand{\an}[1]{\authornote{cyan}{Author}{#1}}

%% Common Parts

\newcommand{\progname}{ProgName} % PUT YOUR PROGRAM NAME HERE
\newcommand{\authname}{Team \#, Team Name
\\ Student 1 name
\\ Student 2 name
\\ Student 3 name
\\ Student 4 name} % AUTHOR NAMES                  

\usepackage{hyperref}
    \hypersetup{colorlinks=true, linkcolor=blue, citecolor=blue, filecolor=blue,
                urlcolor=blue, unicode=false}
    \urlstyle{same}
                                


\begin{document}

\maketitle

This document summarizes the contributions of each team member up to the POC
Demo.  The time period of interest is the time between the beginning of the term
and the POC demo.

\section{Demo Plans}

For our upcoming Proof of Concept (POC) demo, we will showcase the core functionalities of the website. The goal is to demonstrate the key user flows and interactions while keeping the design straightforward and focused on functionality. The demo will highlight the following features:

\begin{itemize}
    \item \textbf{Authentication System:} Users will be able to register, log in, and be assigned a role (user or admin) to access their respective dashboards.
    
    \item \textbf{User and Admin Dashboards:}
    \begin{itemize}
        \item \textbf{User Dashboard:} The user will have access to a form for submitting reimbursement requests. Once a form is submitted, the user can view the status of their request, which will be updated as the request progresses.
        
        \item \textbf{Admin Dashboard:} The admin will be able to view all reimbursement requests submitted by users. Admins can update the status of each request (in progress, accepted, rejected), and these status changes will be reflected on the user’s dashboard.
    \end{itemize}
\end{itemize}

This demo will serve as a basic proof of concept, focusing on demonstrating essential functionalities without incorporating overly complex elements. The emphasis is on ensuring that the primary user interactions, such as registration, dashboard navigation, request submission, request review, and request approval/denial work as intended. This will provide a solid foundation for future development and refinement as the project progresses.

\subsection{Next Steps and Feedback}

As this is a preliminary demonstration, we aim to gather feedback on whether the functionalities outlined above meet the requirements and if any improvements are needed. Any suggestions for enhancements, as well as additional features that may be valuable for the next phase, will be highly appreciated.


\section{Team Meeting Attendance}

\begin{table}[H]
\centering
\begin{tabular}{ll}
\toprule
\textbf{Student} & \textbf{Meetings}\\
\midrule
Total & 8\\
Omar & 8\\
Sufyan & 8\\
Taaha & 8\\
Housam & 8\\
Rachid & 8\\
\bottomrule
\end{tabular}
\end{table}

We did not track our meeting issues in the team repo unfortunately, all of our calls were held in our private discord server. However, moving forward we will make an effort to track all team activities on Git as well. 

\section{Supervisor/Stakeholder Meeting Attendance}

\begin{table}[H]
\centering
\begin{tabular}{ll}
\toprule
\textbf{Student} & \textbf{Meetings}\\
\midrule
Total & 3\\
Omar & 3\\
Sufyan & 3\\
Taaha & 3\\
Housam & 3\\
Rachid & 3\\
\end{tabular}
\end{table}

We call our supervisor for important milestone discussion, we will begin to call more for more feedback based on the website. 
\section{Lecture Attendance}

\begin{table}[H]
\centering
\begin{tabular}{ll}
\toprule
\textbf{Student} & \textbf{Meetings}\\
\midrule
Total & 12\\
Omar & 3/12\\
Sufyan & 5/12\\
Taaha & 4/12\\
Housam & 5/12\\
Rachid & 5
/12\\
\bottomrule
\end{tabular}
\end{table}

We usually try to go to lectures as a team, just to make sure we are all on the same page during the project. 
\section{TA Document Discussion Attendance}

\begin{table}[H]
\centering
\begin{tabular}{ll}
\toprule
\textbf{Student} & \textbf{Lectures}\\
\midrule
Total & 3\\
Omar & 2\\
Sufyan & 2\\
Taaha & 2\\
Housam & 2\\
Rachid & 2\\
\bottomrule
\end{tabular}
\end{table}

We all missed one meeting for the VnV report due to unexpected circumstances, we hope to continue to meet for the upcoming milestones. 
\section{Commits}

\begin{table}[H]
\centering
\begin{tabular}{lll}
\toprule
\textbf{Student} & \textbf{Commits} & \textbf{Percent}\\
\midrule
Total & 136 & 100\% \\
Omar & 16 & 12\% \\ 
Sufyan & 33 & 24\% \\  
Taaha & 24 & 18\% \\  
Housam & 42 & 31\% \\  
Rachid & 21 & 15\% \\  
\bottomrule
\end{tabular}
\end{table}

\section{Issue Tracker}

\begin{table}[H]
\centering
\begin{tabular}{lll}
\toprule
\textbf{Student} & \textbf{Authored (O+C)} & \textbf{Assigned (C only)}\\
\midrule
Omar & 4 & 0 \\
Sufyan & 5 & 0 \\
Taaha & 5 & 1 \\
Housam & 6 & 2 \\
Rachid & 6 & 0 \\
\bottomrule
\end{tabular}
\end{table}

Most issues have been closed yet, but when we revise and update our previous documents we will go through all of the issues. 
\section{CICD}

To ensure consistent quality within our project, we will use a CICD pipeline integrated with GitHub Actions. Our CICD strategy focuses on:

\begin{itemize}
    \item \textbf{Automated Testing and Linting}: Every time a commit is pushed to the repository, GitHub Actions will automatically multiple checks, including running unit tests, integration tests, and linting checks. This will help us catch errors early in the development cycle, ensuring code quality and reducing bugs.

    \item \textbf{PDF Build Automation for Documentation}: Whenever a change is made to any LaTeX file in the repository, GitHub Actions will automatically compile the updated LaTeX document to PDF. This ensures that our documentation is always up-to-date and accessible in PDF format. If the build fails, team members will receive an alert, allowing them to address any issues immediately.

    \item \textbf{Automated Deployment to Staging Environment}: For every change that is pushed to the main branch, GitHub Actions will deploy the latest version of the application to a staging environment. This will allow the team and stakeholders to review and test features. 

\end{itemize}

\subsection{Extras}

\begin{enumerate}
    \item Useability Testing \\
    Conduct formal usability testing with a variety of stakeholders (students, financial staff) to refine the interface and ensure a smooth user experience.
    \item User Documentation \\
    Create comprehensive user documentation in the form of written documentation and video tutorials that guides end-users through every step of submitting expenses, reviewing budgets, and navigating the platform.
\end{enumerate}


\end{document}