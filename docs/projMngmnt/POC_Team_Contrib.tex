\documentclass{article}

\usepackage{float}
\restylefloat{table}

\usepackage{booktabs}

\title{Team Contributions: POC\\\progname}

\author{\authname}

\date{}

%% Comments

\usepackage{color}

\newif\ifcomments\commentstrue %displays comments
%\newif\ifcomments\commentsfalse %so that comments do not display

\ifcomments
\newcommand{\authornote}[3]{\textcolor{#1}{[#3 ---#2]}}
\newcommand{\todo}[1]{\textcolor{red}{[TODO: #1]}}
\else
\newcommand{\authornote}[3]{}
\newcommand{\todo}[1]{}
\fi

\newcommand{\wss}[1]{\authornote{blue}{SS}{#1}} 
\newcommand{\plt}[1]{\authornote{magenta}{TPLT}{#1}} %For explanation of the template
\newcommand{\an}[1]{\authornote{cyan}{Author}{#1}}

%% Common Parts

\newcommand{\progname}{ProgName} % PUT YOUR PROGRAM NAME HERE
\newcommand{\authname}{Team \#, Team Name
\\ Student 1 name
\\ Student 2 name
\\ Student 3 name
\\ Student 4 name} % AUTHOR NAMES                  

\usepackage{hyperref}
    \hypersetup{colorlinks=true, linkcolor=blue, citecolor=blue, filecolor=blue,
                urlcolor=blue, unicode=false}
    \urlstyle{same}
                                


\begin{document}

\maketitle

This document summarizes the contributions of each team member up to the POC
Demo.  The time period of interest is the time between the beginning of the term
and the POC demo.

\section{Demo Plans}

We will be demonstrating the preliminary creation of our website, we hope to showcase the ability for users to login, upload and be notified by the application depending on the reimbursement request. The back-end and front-end will be skeletons of the final build but will still be functionally working. 

\section{Team Meeting Attendance}

\begin{table}[H]
\centering
\begin{tabular}{ll}
\toprule
\textbf{Student} & \textbf{Meetings}\\
\midrule
Total & 8\\
Omar & 8\\
Sufyan & 8\\
Taaha & 8\\
Housam & 8\\
Rachid & 8\\
\bottomrule
\end{tabular}
\end{table}

We did not track our meeting issues in the team repo unfortunately, all of our calls were held in our private discord server. However, moving forward we will make an effort to track all team activities on Git as well. 

\section{Supervisor/Stakeholder Meeting Attendance}

\begin{table}[H]
\centering
\begin{tabular}{ll}
\toprule
\textbf{Student} & \textbf{Meetings}\\
\midrule
Total & 3\\
Omar & 3\\
Sufyan & 3\\
Taaha & 3\\
Housam & 3\\
Rachid & 3\\
\end{tabular}
\end{table}

We call our supervisor for important milestone discussion, we will begin to call more for more feedback based on the website. 
\section{Lecture Attendance}

\begin{table}[H]
\centering
\begin{tabular}{ll}
\toprule
\textbf{Student} & \textbf{Meetings}\\
\midrule
Total & 26\\
Omar & 17/26\\
Sufyan & 19/26\\
Taaha & 18/26\\
Housam & 19/26\\
Rachid & 17/26\\
\bottomrule
\end{tabular}
\end{table}

We usually try to go to lectures as a team, just to make sure we are all on the same page during the project. 
\section{TA Document Discussion Attendance}

\begin{table}[H]
\centering
\begin{tabular}{ll}
\toprule
\textbf{Student} & \textbf{Lectures}\\
\midrule
Total & 3\\
Omar & 2\\
Sufyan & 2\\
Taaha & 2\\
Housam & 2\\
Rachid & 2\\
\bottomrule
\end{tabular}
\end{table}

We all missed one meeting for the VnV report due to unexpected circumstances, we hope to continue to meet for the upcoming milestones. 
\section{Commits}

\begin{table}[H]
\centering
\begin{tabular}{lll}
\toprule
\textbf{Student} & \textbf{Commits} & \textbf{Percent}\\
\midrule
Total & 136 & 100\% \\
Omar & 16 & 12\% \\ 
Sufyan & 33 & 24\% \\  
Taaha & 24 & 18\% \\  
Housam & 42 & 31\% \\  
Rachid & 21 & 15\% \\  
\bottomrule
\end{tabular}
\end{table}

\section{Issue Tracker}

\begin{table}[H]
\centering
\begin{tabular}{lll}
\toprule
\textbf{Student} & \textbf{Authored (O+C)} & \textbf{Assigned (C only)}\\
\midrule
Omar & 4 & 0 \\
Sufyan & 5 & 0 \\
Taaha & 5 & 1 \\
Housam & 6 & 2 \\
Rachid & 6 & 0 \\
\bottomrule
\end{tabular}
\end{table}

Most issues have been closed yet, but when we revise and update our previous documents we will go through all of the issues. 
\section{CICD}

To ensure consistent quality within our project, we will use a CICD pipeline integrated with GitHub Actions. Our CICD strategy focuses on:

\begin{itemize}
    \item \textbf{Automated Testing and Linting}: Every time a commit is pushed to the repository, GitHub Actions will automatically multiple checks, including running unit tests, integration tests, and linting checks. This will help us catch errors early in the development cycle, ensuring code quality and reducing bugs.

    \item \textbf{PDF Build Automation for Documentation}: Whenever a change is made to any LaTeX file in the repository, GitHub Actions will automatically compile the updated LaTeX document to PDF. This ensures that our documentation is always up-to-date and accessible in PDF format. If the build fails, team members will receive an alert, allowing them to address any issues immediately.

    \item \textbf{Automated Deployment to Staging Environment}: For every change that is pushed to the main branch, GitHub Actions will deploy the latest version of the application to a staging environment. This will allow the team and stakeholders to review and test features. 

    \item \textbf{Version Control and Release Management}: At each major project milestone, GitHub Actions will assist in creating a tagged release. This includes an archived version of the code, documentation, and any necessary deployment instructions, providing a clear record of progress.

\end{itemize}

\end{document}