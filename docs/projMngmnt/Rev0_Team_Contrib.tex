\documentclass{article}

\usepackage{float}
\restylefloat{table}

\usepackage{booktabs}

\title{Team Contributions: Rev 0\\\progname}

\author{\authname}

\date{}

%% Comments

\usepackage{color}

\newif\ifcomments\commentstrue %displays comments
%\newif\ifcomments\commentsfalse %so that comments do not display

\ifcomments
\newcommand{\authornote}[3]{\textcolor{#1}{[#3 ---#2]}}
\newcommand{\todo}[1]{\textcolor{red}{[TODO: #1]}}
\else
\newcommand{\authornote}[3]{}
\newcommand{\todo}[1]{}
\fi

\newcommand{\wss}[1]{\authornote{blue}{SS}{#1}} 
\newcommand{\plt}[1]{\authornote{magenta}{TPLT}{#1}} %For explanation of the template
\newcommand{\an}[1]{\authornote{cyan}{Author}{#1}}

%% Common Parts

\newcommand{\progname}{ProgName} % PUT YOUR PROGRAM NAME HERE
\newcommand{\authname}{Team \#, Team Name
\\ Student 1 name
\\ Student 2 name
\\ Student 3 name
\\ Student 4 name} % AUTHOR NAMES                  

\usepackage{hyperref}
    \hypersetup{colorlinks=true, linkcolor=blue, citecolor=blue, filecolor=blue,
                urlcolor=blue, unicode=false}
    \urlstyle{same}
                                


\begin{document}

\maketitle

This document summarizes the contributions of each team member for the Rev 0
Demo.  The time period of interest is the time between the POC demo and the Rev
0 demo.

\section{Demo Plans}

For our upcoming Proof of Concept (POC) demo, we will showcase the core functionalities of the website, building on the features demonstrated previously. The goal is to highlight both the established and newly added functionalities, ensuring a user-friendly and efficient design. The demo will include the following features:

\begin{itemize}
    \item \textbf{Authentication and Role-Based Access:} Users will be able to register, log in, and be assigned specific permissions. These permissions will dynamically control which pages and functionalities they can access.

    \item \textbf{Requests and Approval Workflow:}
    \begin{itemize}
        \item Users can submit reimbursement or payment requests via a structured form.
        \item Admins or authorized personnel can review submitted requests and update their status.
        \item Status updates will be reflected dynamically, allowing users to track their requests in real time.
        \item Approved requests will automatically update the corresponding group's budget allocation.
        \item The system will provide real-time feedback to admins, displaying the remaining budget after each approval.
    \end{itemize}

    \item \textbf{Analytics and Reporting:}
    \begin{itemize}
        \item The Analytics Page will provide visual insights into various financial and operational metrics, helping administrators and users track key trends and activity.
        \item Graphs and charts will help admins monitor financial health and make data-driven decisions.
    \end{itemize}

    \item \textbf{Role and Group Management:}
    \begin{itemize}
        \item The Roles Page allows admins to create, edit, and assign roles with customized permissions.
        \item The Groups Page enables the management of different groups, linking them to specific users and budget allocations.
        \item These tools ensure flexible user management and controlled access to system functionalities.
    \end{itemize}

    
    \item \textbf{New Features:}
    \begin{itemize}
        \item \textbf{Image Processing for Receipt Scanning:} A new feature allowing users to upload receipt images, which will be automatically scanned to extract relevant details, reducing manual input requirements.
        \item \textbf{Personalized User Entries:} Enhanced functionality to personalize and save user-specific information, reducing redundancy in input for repeated submissions.
        \item \textbf{User Manual and Tutorials:} A comprehensive guide and set of tutorials will be provided to assist users in navigating the application, ensuring smooth onboarding and efficient utilization of the system.
    \end{itemize}
\end{itemize}

This demo will serve as the first revision of our proof of concept, focusing on demonstrating essential functionalities alongside key enhancements. The emphasis remains on delivering a seamless user experience, with features such as registration, dashboard navigation, receipt scanning, request submission, request review, and status updates working as intended. 

This foundational demonstration will pave the way for future development and refinement as the project progresses.


\section{Team Meeting Attendance}


Please note many of our meetings were informal, meeting in person after class as well as a call on discord to ask questions about what is currently being worked on and for clarification from other group members.
\begin{table}[H]
    \centering
    \begin{tabular}{ll}
    \toprule
    \textbf{Student} & \textbf{Meetings}\\
    \midrule
    Total & 2\\
    Sufyan Motala & 2\\
    Rachid Khneisser & 2\\
    Housam Alamour & 2\\
    Omar Muhammad & 2\\
    Taaha Atif & 2\\
    \bottomrule
    \end{tabular}
    \end{table}

\wss{If needed, an explanation for the counts can be provided here.}

\section{Supervisor/Stakeholder Meeting Attendance}

\begin{table}[H]
\centering
\begin{tabular}{ll}
\toprule
\textbf{Student} & \textbf{Meetings}\\
\midrule
Total & 1\\
Sufyan Motala & 1\\
Rachid Khneisser & 1\\
Housam Alamour & 1\\
Omar Muhammad & 1\\
Taaha Atif & 1\\
\bottomrule
\end{tabular}
\end{table}

The count is only 1 because we only met with our supervisor for one meeting so far since the POC demo. We have scheduled weekly meetings so this will be better moving forward.

\section{Lecture Attendance}

\begin{table}[H]
\centering
\begin{tabular}{ll}
\toprule
\textbf{Student} & \textbf{Lectures}\\
\midrule
Total & 1\\
Sufyan Motala & 1\\
Rachid Khneisser & 1\\
Housam Alamour & 1\\
Omar Muhammad & 0\\
Taaha Atif & 0\\
\bottomrule
\end{tabular}
\end{table}

Only the lecture discussing the rev 0 demo was attended by the team. Not all members could make it, but updates were provided in our Discord for those who missed.

\section{TA Document Discussion Attendance}

\begin{table}[H]
\centering
\begin{tabular}{ll}
\toprule
\textbf{Student} & \textbf{Lectures}\\
\midrule
Total & 1\\
Sufyan Motala & 1\\
Rachid Khneisser & 1\\
Housam Alamour & 1\\
Omar Muhammad & 1\\
Taaha Atif & 1\\
\bottomrule
\end{tabular}
\end{table}

The count is only 1 because we only met with the TA once for our informal meeting to discuss the design docs (MG/MIS).

\section{Commits}

\begin{table}[H]
\centering
\begin{tabular}{lll}
\toprule
\textbf{Student} & \textbf{Commits} & \textbf{Percent}\\
\midrule
Total & 47 & 100\% \\
Housam Alamour & 11 & 23\% \\
Sufyan Motala & 6 & 13\% \\
Omar Muhammad & 9 & 19\% \\
Rachid Khneisser & 13 & 28\% \\
Taaha Atif & 2 & 4\% \\
\bottomrule
\end{tabular}
\end{table}

\section{Issue Tracker}

\begin{table}[H]
\centering
\begin{tabular}{lll}
\toprule
\textbf{Student} & \textbf{Authored (O+C)} & \textbf{Assigned (C only)}\\
\midrule
Name 1 & 3 & 0 \\
Name 2 & 2 & 0 \\
Name 3 & 2 & 0 \\
Name 4 & 0 & 0 \\
Name 5 & 0 & 0 \\
\bottomrule
\end{tabular}
\end{table}

\section{CICD}

Our project utilizes Continuous Integration and Continuous Deployment (CICD) to streamline development and maintain code quality. Our CICD pipeline includes:

\begin{itemize}
    \item \textbf{Linter \& Formatter:} Automated linting and formatting tools ensure that all code adheres to a consistent style and best practices, reducing errors and improving readability.
    \item \textbf{LaTeX Generation:} Our pipeline automates the compilation of LaTeX documents, allowing for quick and consistent generation of project reports and documentation.
\end{itemize}

These automated processes help maintain a smooth development workflow, enforce coding standards, and ensure that documentation remains up to date.

\end{document}
