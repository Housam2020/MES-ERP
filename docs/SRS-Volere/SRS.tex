% THIS DOCUMENT IS FOLLOWS THE VOLERE TEMPLATE BY Suzanne Robertson and James Robertson
% ONLY THE SECTION HEADINGS ARE PROVIDED
%
% Initial draft from https://github.com/Dieblich/volere
%
% Risks are removed because they are covered by the Hazard Analysis
\documentclass[12pt]{article}

\usepackage{booktabs}
\usepackage{tabularx}
\usepackage{hyperref}
\hypersetup{
    bookmarks=true,         % show bookmarks bar?
      colorlinks=true,      % false: boxed links; true: colored links
    linkcolor=red,          % color of internal links (change box color with linkbordercolor)
    citecolor=green,        % color of links to bibliography
    filecolor=magenta,      % color of file links
    urlcolor=cyan           % color of external links
}

\newcommand{\lips}{\textit{Insert your content here.}}

%% Comments

\usepackage{color}

\newif\ifcomments\commentstrue %displays comments
%\newif\ifcomments\commentsfalse %so that comments do not display

\ifcomments
\newcommand{\authornote}[3]{\textcolor{#1}{[#3 ---#2]}}
\newcommand{\todo}[1]{\textcolor{red}{[TODO: #1]}}
\else
\newcommand{\authornote}[3]{}
\newcommand{\todo}[1]{}
\fi

\newcommand{\wss}[1]{\authornote{blue}{SS}{#1}} 
\newcommand{\plt}[1]{\authornote{magenta}{TPLT}{#1}} %For explanation of the template
\newcommand{\an}[1]{\authornote{cyan}{Author}{#1}}

%% Common Parts

\newcommand{\progname}{ProgName} % PUT YOUR PROGRAM NAME HERE
\newcommand{\authname}{Team \#, Team Name
\\ Student 1 name
\\ Student 2 name
\\ Student 3 name
\\ Student 4 name} % AUTHOR NAMES                  

\usepackage{hyperref}
    \hypersetup{colorlinks=true, linkcolor=blue, citecolor=blue, filecolor=blue,
                urlcolor=blue, unicode=false}
    \urlstyle{same}
                                


\begin{document}

\title{Software Requirements Specification for \progname: subtitle describing software} 
\author{\authname}
\date{\today}
	
\maketitle

~\newpage

\pagenumbering{roman}

\tableofcontents

~\newpage

\section*{Revision History}

\begin{tabularx}{\textwidth}{p{3cm}p{2cm}X}
\toprule {\textbf{Date}} & {\textbf{Version}} & {\textbf{Notes}}\\
\midrule
Date 1 & 1.0 & Notes\\
Date 2 & 1.1 & Notes\\
\bottomrule
\end{tabularx}

~\\

~\newpage
\section{Purpose of the Project}
\subsection{User Business}
\lips
\subsection{Goals of the Project}
\lips
\section{Stakeholders}
\subsection{Client}
\lips
\subsection{Customer}
\lips
\subsection{Other Stakeholders}
\lips
\subsection{Hands-On Users of the Project}
\lips
\subsection{Personas}
\lips
\subsection{Priorities Assigned to Users}
\lips
\subsection{User Participation}
\lips
\subsection{Maintenance Users and Service Technicians}
\lips

\section{Mandated Constraints}
\subsection{Solution Constraints}
\begin{enumerate}
  \item The app must handle the throughput of the 60 student groups to the MES wihtout trouble.
  \item The app must inform student groups the status of their request. 
  \item The app must notify the MES for any new requests made 
  \item The app must be up and running at all times, if not a message must be shown signifying why the app is offline. 
\end{enumerate}
\subsection{Implementation Environment of the Current System}
\lips
\subsection{Partner or Collaborative Applications}
\lips
\subsection{Off-the-Shelf Software}
\lips
\subsection{Anticipated Workplace Environment}
\lips
\subsection{Schedule Constraints}
\lips
\subsection{Budget Constraints}
\lips
\subsection{Enterprise Constraints}
\lips

\section{Naming Conventions and Terminology}
\subsection{Glossary of All Terms, Including Acronyms, Used by Stakeholders
involved in the Project}

\subsection{Glossary of All Terms, Including Acronyms, Used by Stakeholders Involved in the Project}

\begin{itemize}
    \item \textbf{MES}: McMaster Engineering Society. The student organization responsible for overseeing financial and academic matters for engineering students at McMaster University.    
    \item \textbf{Audit Log}: A record of all actions performed within the system, used for tracking who accessed or modified data and when these actions took place.
    \item \textbf{Club Financial Manager}: A designated member of a club responsible for managing that club's financial records and submitting reimbursement requests on behalf of the club.
    \item \textbf{Budget Allocation}: The amount of funding designated for specific clubs or activities by the MES team and is used to track and control spending.
    \item \textbf{Vendor Contracts}: Agreements between MES and external suppliers for goods or services provided to the engineering society or its member clubs.
    \item \textbf{Approval Workflow}: The process that a reimbursement request goes through before getting final approval.
    \item \textbf{Financial Dashboard}: A feature of the system that provides an overview of the current financial standing, showing budget tracking, reimbursement status, and pending requests.
    \item \textbf{CI/CD}: Continuous Integration/Continuous Deployment. A software development practice that automatically tests and deploys code changes to ensure the system is always in a working state.
\end{itemize}


\section{Relevant Facts And Assumptions}
\subsection{Relevant Facts}
\begin{itemize}
  \item There is no current application built for this project.
  \item The project will be built primarily in ReactJS.
  \item There are 60 student groups to account for.
  \item The current process for submitting checks is manual, which takes far too long.
  \item The input should be generalized, with no differentiation between student groups.
  \item The application is intended to be created with future growth in mind.
\end{itemize}



\subsection{Business Rules}
\begin{itemize}
  \item Users will be notified when their reciepts are re-imbursed.
  \item The site will notify the MES when a request comes through.
  \item Users will be notified of an expected deadline for when the requests will be fulfilled.
  \item Users will be notified of any maintenance times the site will need.
  \item All financial recors will be recoreded in the backend of the appication 
  \item Each student group must submit their own request by a deadline given by the MES.
\end{itemize}

\subsection{Assumptions}
\begin{itemize}
  \item Student groups and the MES are in contact and know the ground rules of what to expect from purhcases.
  \item We are expecting the MES to provide us with a server to run our software
  \item The system will be coded in ReactJS
  \item The system will be built from scratch 
  \item The MES is expected to provide us with test cases for the app
  \item Internet is expected to be provided, as well as a suitable system with enough RAM/storage for the app. 
  \item All 60 student groups will use the application and provide as even more test cases for the app. 
  \item Reciepts will be compatible in any form for the app.
\end{itemize}

\section{The Scope of the Work}
\subsection{The Current Situation}
\lips
\subsection{The Context of the Work}
\lips
\subsection{Work Partitioning}
\lips
\subsection{Specifying a Business Use Case (BUC)}
\lips

\section{Business Data Model and Data Dictionary}
\subsection{Business Data Model}
\lips
\subsection{Data Dictionary}
\lips

\section{The Scope of the Product}
\subsection{Product Boundary}
\lips
\subsection{Product Use Case Table}
\lips
\subsection{Individual Product Use Cases (PUC's)}
\lips

\section{Functional Requirements}
\subsection{Functional Requirements}

\begin{itemize}

  \item \textbf{Requirement Number}: 001
  \item \textbf{Description}: The system must allow club financial managers to submit reimbursement requests.
  \item \textbf{Rationale}: This functionality ensures that clubs can request refunds for approved expenses efficiently and without error.
  \item \textbf{Fit Criterion}: A club financial manager must be able to submit a reimbursement request successfully, which should appears in the MES staff's pending approval list within 1 minute.
  \item \textbf{Priority}: High
  \item \textbf{Originator}: Club Financial Manager
  \item \textbf{Dependencies}: Requires the role-based access system to be implemented.
  
  \bigskip

  \item \textbf{Requirement Number}: 002
  \item \textbf{Description}: The system must allow MES staff to approve or reject reimbursement requests.
  \item \textbf{Rationale}: Ensures proper management of reimbursement requests.
  \item \textbf{Fit Criterion}: A reimbursement request must be updated with the approval or rejection status within 5 seconds of the MES staff submitting their decision.
  \item \textbf{Priority}: High
  \item \textbf{Originator}: MES Finance Team
  \item \textbf{Dependencies}: Dependent on successful submission of reimbursement requests by club financial managers.

  \bigskip

  \item \textbf{Requirement Number}: 003
  \item \textbf{Description}: The system must send notifications to club financial managers when their reimbursement requests are approved or rejected.
  \item \textbf{Rationale}: This ensures that club financial managers are informed of the status of their reimbursement requests.
  \item \textbf{Fit Criterion}: A notification must be sent within 5 seconds of approval or rejection of a reimbursement request.
  \item \textbf{Priority}: Medium
  \item \textbf{Originator}: MES Finance Team
  \item \textbf{Dependencies}: Must follow the approval or rejection of a request by MES staff.

  \bigskip

  \item \textbf{Requirement Number}: 004
  \item \textbf{Description}: The system must allow administrators to access all financial records, audit logs, and user activity.
  \item \textbf{Rationale}: Administrators need full access for auditing, tracking, and system oversight.
  \item \textbf{Fit Criterion}: Administrators must be able to view any financial records or user actions within 5 seconds of requesting the information.
  \item \textbf{Priority}: High
  \item \textbf{Originator}: System Administrators
  \item \textbf{Dependencies}: Requires the role-based access system to be implemented.
  
  \bigskip
  
  \item \textbf{Requirement Number}: 005
  \item \textbf{Description}: The system must allow users to view a reason for rejection when a reimbursement request is rejected.
  \item \textbf{Rationale}: To help users correct issues with their requests.
  \item \textbf{Fit Criterion}: On rejection, the system must display or notify the user of the rejection reason within 5 seconds.
  \item \textbf{Priority}: Medium
  \item \textbf{Originator}: Club Financial Managers

\end{itemize}

\section{Look and Feel Requirements}
\subsection{Appearance Requirements}
\begin{enumerate}
  \item The platform must adhere to the established MES color scheme of maroon, white, and red to ensure a consistent visual identity across all MES platforms. \\
  Rationale: This creates a seamless brand experience for users who are familiar with MES's website and other communications.
  \item The platform must be fully responsive to adjust seamlessly to various screen sizes including desktop, tablet, and mobile devices. \\
  Rationale: Many users, especially students, are likely to access the platform from mobile devices. A responsive design ensures the platform is accessible to all users.
  \item  Animations should be kept minimal and used only when necessary to avoid distractions. \\
  Rationale: Minimal animation improves the user experience through feedback, without being very distracting, hence keeping the focus on the functionality of the platform.
\end{enumerate}
\subsection{Style Requirements}
\begin{enumerate}
  \item The system must not have elements that depict violence, nudity or language that is discriminatory, vulgar, or derogatory. \\ 
  Rationale: The system should not offensive or inappropriate to users.
  \item There will be no visually unpleasing elements such as colour combinations that strain the eyes. \\ 
  Rationale: There should not be any elements that make the application harder to read.
  \item The platform's tone should remain professional and formal. \\
  Rationale: A professional tone builds trust with users by emphasizing the seriousness of the platform, particularly given its purpose in handling financial workflows.
  \item Form fields should be simple and clear, with error states highlighted in red to help users easily identify and correct mistakes. \\
  Rationale: Well-designed form fields minimize confusion during data entry, while clear error messages guide users through resolving issues.
  \item Clear and intuitive icons should be used for navigation and key actions, such as submitting expenses or reviewing approvals. \\
  Rationale: Icons provide immediate visual cues that reduces the need for text explanations.
\end{enumerate}

\section{Usability and Humanity Requirements}
\subsection{Ease of Use Requirements}
\lips
\subsection{Personalization and Internationalization Requirements}
\lips
\subsection{Learning Requirements}
\lips
\subsection{Understandability and Politeness Requirements}
\lips
\subsection{Accessibility Requirements}
\lips

\section{Performance Requirements}
\subsection{Speed and Latency Requirements}
\lips
\subsection{Safety-Critical Requirements}
\lips
\subsection{Precision or Accuracy Requirements}
\lips
\subsection{Robustness or Fault-Tolerance Requirements}
\lips
\subsection{Capacity Requirements}
\lips
\subsection{Scalability or Extensibility Requirements}
\lips
\subsection{Longevity Requirements}
\lips

\section{Operational and Environmental Requirements}
\subsection{Expected Physical Environment}
\begin{enumerate}
  \item The system will be in a office room.
  \item The systems location will be on the McMaster Campus. 
  \item The product will be useable in natural office enviornments. 
\end{enumerate}
\subsection{Wider Environment Requirements}
\begin{enumerate}
  \item The hardware of the system will not be in an unusual enviornment
\end{enumerate}

\subsection{Requirements for Interfacing with Adjacent Systems}
\begin{enumerate}
  \item The app will be usable on a desktop or laptop.
  \item The app will be runnable on windows, mac and linux 
  \item The app will work on the last 4 versions of Chrome, Firefox and edge 
\end{enumerate}
\subsection{Productization Requirements}
\begin{enumerate}
  \item N/A (this product is not for sale) 
\end{enumerate}
\subsection{Release Requirements}
\begin{enumerate}
  \item The product will follow an agile software development lifestyle.
  \item The product will only have maintenance is a crucial bug is found. 
  \item Each newer release will not make an older one fail. 
  \item Products must be throughouly tested before each release. 
  \item A bug found will be made a ticket which will be given a deadline to solve by and then a new release build will be made.  
\end{enumerate}

\section{Maintainability and Support Requirements}
\subsection{Maintenance Requirements}
\lips
\subsection{Supportability Requirements}
\lips
\subsection{Adaptability Requirements}
\lips

\section{Security Requirements}
\subsection{Access Requirements}

\begin{itemize}
  \item Administrators must have full access to all system functionalities, including creating, modifying, and deleting records for all users, managing user roles, and viewing audit logs and financial reports.
  \item MES staff must have access to financial records, reimbursement requests, audit logs, and the ability to approve or reject reimbursement requests.
  \item Club financial managers must have access to their respective club's financial records, submission of reimbursement requests, and viewing of their club's budget tracking, but no access to financial data from other clubs.
  \item Only administrators and MES staff must be able to access sensitive data such as vendor contracts, budget allocations, and event financials.
  \item Club financial managers must not have access to sensitive financial information beyond their own club's data.
  \item Administrators and MES staff must be able to modify financial records, approve budgets, and manage payments, while club financial managers can only submit reimbursement requests.
  \item All access to information and functions, such as reimbursement approval and budget tracking, must be restricted according to user roles to protect the confidentiality of the financial data.
\end{itemize}

\subsection{Integrity Requirements}

\begin{itemize}
    \item The system must ensure that all financial data, such as budgets, expense reports, and reimbursement requests, remain accurate and have not been tampered with. This includes preventing inaccurate or unauthorized modifications to any financial records. This is to ensure that submitted data is valid and consistent with the system's requirements.
    \item The product must protect itself from attacks coming from external sources and from unintentional mistakes by authorized users. If an external attack occurs, the system should log the attempt and prevent any changes to financial data.
    \item The system shall prevent internal misuse, such as incorrect data entry, through validation rules and error handling mechanisms.
    \item To maintain data integrity, the system must perform automated backups of all important financial data. In the event of system failure, data corruption, or accidental deletion, these backups should allow the system to recover with minimal data loss. Recovery processes should be tested regularly to ensure reliability.
    \item The system shall maintain detailed logs of all actions taken within the platform, such as data modifications, approvals, or deletions. These logs will be immutable and can be used to restore data or identify the source of an error if data becomes corrupted or tampered with.
\end{itemize}

\subsection{Privacy Requirements}

\begin{itemize}
  \item The system must comply with all relevant privacy laws and regulations to ensure that personal data is handled in accordance with these laws.
  \item Before collecting any personal information, the system must inform users of the types of data collected, how it will be used, and who will have access to it.
  \item The system must obtain consent from users before collecting personal information. Users must have the ability to revoke data collection consent at any time, and the system must stop processing or storing the data on the user.
  \item Users must be able to view, edit, or request corrections to their personal data stored in the system.
  \item The system must notify users of any changes to its information or privacy policy. Users should have the opportunity to review and give feedback or consent to the changes before any further processing of their data.
  \item The system must protect all personal information with encryption or other security measures to ensure it cannot be accessed, modified, or deleted by unauthorized users. 
\end{itemize}

\subsection{Audit Requirements}

\begin{itemize}
  \item The system must retain detailed records of all user transactions, including financial submissions, approvals, modifications, and deletions, ensuring that all actions can be traced back to the responsible user.
  \item Audit logs must record the date and time of each transaction, the user involved, and the specific actions taken, ensuring full traceability of the system's usage.
  \item The audit logs must be immutable, ensuring that no user can modify or delete the audit records.
  \item The system must retain audit logs for a minimum period of one year, or as required by applicable financial regulations.
  \item The system must record login attempts, both successful and unsuccessful, to provide a complete history of user access to the system.
  \item Access to the audit logs must be restricted to authorized users, such as administrators and auditors, ensuring confidentiality of the audit data.
  \item The system must ensure that audit data is protected from unauthorized access or tampering, using security measures such as encryption and access control.
\end{itemize}

\subsection{Immunity Requirements}

\begin{itemize}
  \item The system must implement protective measures against unauthorized programs such as viruses, malwares, and spywares, ensuring that the system remains secure from threats.
  \item The system must regularly scan for potential security vulnerabilities and unauthorized software, using antivirus and anti-malware tools to detect threats.
  \item The system must be capable of eliminating any detected malicious software to prevent it from affecting the integrity of the system or its data.
  \item The system must provide real-time monitoring for suspicious activities that show attempted infections.
  \item The system must ensure that all software components and libraries used in the product are regularly updated with the latest security patches to reduce the risk of infection by known vulnerabilities.
\end{itemize}

\section{Cultural Requirements}
\subsection{Cultural Requirements}
\begin{enumerate}
  \item The application must be available in English. \\
  Rationale: English is the primary language spoken by citizens in Ontario and thus shall be the primary method of communication.
  \item The application shall use British spelling. \\
  Rationale: As the application is for use within McMaster University, the predominant spelling method is using British spelling.
  \item The system must be designed to be inclusive and usable by people from diverse cultural, linguistic, and social backgrounds. This includes ensuring gender-neutral language and avoiding culturally biased terms. \\
  Rationale:  MES is a diverse community composed of students and staff from various cultural and linguistic backgrounds. The platform must reflect this diversity by avoiding assumptions about users' cultural norms or preferences.
  \item The platform should allow users to input and view dates and times in different time zones and formats (e.g., 24-hour vs 12-hour clock, day-month-year vs month-day-year formats). \\
  Rationale: MES includes international students and staff who may be accustomed to different time zones and date formats. Flexibility in how time and dates are presented enhances the usability for these diverse users.
\end{enumerate}

\section{Compliance Requirements}
\subsection{Legal Requirements}
\lips
\subsection{Standards Compliance Requirements}
\lips

\section{Open Issues}
\begin{itemize}
  \item \textbf{Issue Number}: 001
  \item \textbf{Cross-reference}: Affects Functional Requirement 003.
  \item \textbf{Summary}: The team is deciding if notifications should be sent through email, SMS, or through in-app alerts.
  \item \textbf{Stakeholders Involved}: Project Team, MES IT Team.
  \item \textbf{Action}: Perform a survey with stakeholders to determine the preferred notification method.
  \item \textbf{Resolution}: Expected decision within one week.
  
  \bigskip
  
  \item \textbf{Issue Number}: 002
  \item \textbf{Cross-reference}: Affects Functional Requirement 001.
  \item \textbf{Summary}: It is not clear what types of reimbursement requests that will require additional validation.
  \item \textbf{Stakeholders Involved}: Project Team, MES Finance Team.
  \item \textbf{Action}: Stakeholders will be meeting to define validation rules for reimbursement requests.
  \item \textbf{Resolution}: Expected within two weeks.
  
  \bigskip
  
  \item \textbf{Issue Number}: 003
  \item \textbf{Cross-reference}: Affects Functional Requirement 004.
  \item \textbf{Summary}: It is not clear how long audit logs should be stored in the system.
  \item \textbf{Stakeholders Involved}: Project Team, MES IT Team.
  \item \textbf{Action}: Stakeholders will be meeting to determine data retention policies. 
  \item \textbf{Resolution}: Expected decision within two weeks.
  
  \bigskip

  \item \textbf{Issue Number}: 004
  \item \textbf{Cross-reference}: Affects Functional Requirement 002.
  \item \textbf{Summary}: It is not clear whether some requests should get multiple approvals from multiple people. For example, a reimbursement request showing high levels of spending needing to be reimbursed.
  \item \textbf{Stakeholders Involved}: Project Team, MES Finance Team.
  \item \textbf{Action}: Stakeholders will be meeting to define thresholds and cases for multiple approvals.
  \item \textbf{Resolution}: Expected decision within two weeks.
  
  \bigskip
  
  \item \textbf{Issue Number}: 005
  \item \textbf{Cross-reference}: Affects Look and Feel Requirements.
  \item \textbf{Summary}: It has not yet been determined how the platform's responsiveness will be prioritized across desktop, tablet, and mobile.
  \item \textbf{Stakeholders Involved}: Project Team.
  \item \textbf{Action}: Stakeholders will be meeting to perform UI/UX design mockups to determine design priorities.
  \item \textbf{Resolution}: Expected decision within two weeks.
\end{itemize}

\section{Off-the-Shelf Solutions}
\subsection{Ready-Made Products}
\begin{enumerate}
  \item There is no pre-existing solution for this problem, this problem is MES specific and has never been done before.
\end{enumerate}
\subsection{Reusable Components}
\begin{enumerate}
  \item The main library we will use is ReactJS, there will be no other components that can be used/reused. 
\end{enumerate}
\subsection{Products That Can Be Copied}
\begin{enumerate}
  \item Any budgeting apps UI can be copied and taken over for the front end of the application. With that frontend along with some changes it can be applied to our app. 
\end{enumerate}

\section{New Problems}
\subsection{Effects on the Current Environment}
\lips
\subsection{Effects on the Installed Systems}
\lips
\subsection{Potential User Problems}
\lips
\subsection{Limitations in the Anticipated Implementation Environment That May
Inhibit the New Product}
\lips
\subsection{Follow-Up Problems}
\lips

\section{Tasks}
\subsection{Project Planning}
\lips
\subsection{Planning of the Development Phases}
\lips

\section{Migration to the New Product}
\subsection{Requirements for Migration to the New Product}
\lips
\subsection{Data That Has to be Modified or Translated for the New System}
\lips

\section{Costs}
\lips
\section{User Documentation and Training}
\subsection{User Documentation Requirements}
\lips
\subsection{Training Requirements}
\lips

\section{Waiting Room}
\begin{enumerate}
  \item A mobile app version of the platform should be developed for easier on-the-go access to financial tracking and submissions. \\
  Rationale: A dedicated mobile app would improve accessibility for users who need to submit expenses or view budgets while away from their desktops. Although this would be beneficial, it is not critical for the platform's core functionality as responsive design already ensures usability on mobile devices.
  \item Implement machine learning-based expense categorization to automatically categorize expenses based on historical data. \\
  Rationale: Automated categorization of expenses would reduce manual entry and improve accuracy, providing an intelligent system that learns from user inputs. This is a technically complex feature that could be revisited after the platform's initial release.
  \item Provide users with customizable dashboards where they can select and prioritize the financial metrics most relevant to them. \\
  Rationale: Customizable dashboards allow users to personalize their experience and focus on the data that matters most to them. While this is an attractive feature, it can be delayed until after the core dashboard functionalities are developed.
  \item Enable multi-language functionality, allowing users to switch between different languages for a more inclusive experience. \\
  Rationale: MES is a diverse community with users from various linguistic backgrounds. Multi-language support would enhance accessibility and usability for all users. However, this feature is not critical for the platform's initial release and can be considered in future updates.
\end{enumerate}

\section{Ideas for Solution}
\lips

\newpage{}
\section*{Appendix --- Reflection}

The information in this section will be used to evaluate the team members on the
graduate attribute of Lifelong Learning.  Please answer the following questions:

\begin{enumerate}
  \item What knowledge and skills will the team collectively need to acquire to
  successfully complete this capstone project?  Examples of possible knowledge
  to acquire include domain specific knowledge from the domain of your
  application, or software engineering knowledge, mechatronics knowledge or
  computer science knowledge.  Skills may be related to technology, or writing,
  or presentation, or team management, etc.  You should look to identify at
  least one item for each team member.
  \item For each of the knowledge areas and skills identified in the previous
  question, what are at least two approaches to acquiring the knowledge or
  mastering the skill?  Of the identified approaches, which will each team
  member pursue, and why did they make this choice?
\end{enumerate}

\end{document}