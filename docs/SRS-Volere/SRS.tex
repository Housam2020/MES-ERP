% THIS DOCUMENT FOLLOWS THE VOLERE TEMPLATE BY Suzanne Robertson and James Robertson
% ONLY THE SECTION HEADINGS ARE PROVIDED
%
% Initial draft from https://github.com/Dieblich/volere
%
% Risks are removed because they are covered by the Hazard Analysis
\documentclass[12pt]{article}

\usepackage{booktabs}
\usepackage{tabularx}
\usepackage{longtable}
\usepackage{hyperref}
\hypersetup{
    bookmarks=true,         % show bookmarks bar?
      colorlinks=true,      % false: boxed links; true: colored links
    linkcolor=red,          % color of internal links (change box color with linkbordercolor)
    citecolor=green,        % color of links to bibliography
    filecolor=magenta,      % color of file links
    urlcolor=cyan           % color of external links
}

\newcommand{\lips}{\textit{Insert your content here.}}

%% Comments

\usepackage{color}

\newif\ifcomments\commentstrue %displays comments
%\newif\ifcomments\commentsfalse %so that comments do not display

\ifcomments
\newcommand{\authornote}[3]{\textcolor{#1}{[#3 ---#2]}}
\newcommand{\todo}[1]{\textcolor{red}{[TODO: #1]}}
\else
\newcommand{\authornote}[3]{}
\newcommand{\todo}[1]{}
\fi

\newcommand{\wss}[1]{\authornote{blue}{SS}{#1}} 
\newcommand{\plt}[1]{\authornote{magenta}{TPLT}{#1}} %For explanation of the template
\newcommand{\an}[1]{\authornote{cyan}{Author}{#1}}

%% Common Parts

\newcommand{\progname}{ProgName} % PUT YOUR PROGRAM NAME HERE
\newcommand{\authname}{Team \#, Team Name
\\ Student 1 name
\\ Student 2 name
\\ Student 3 name
\\ Student 4 name} % AUTHOR NAMES                  

\usepackage{hyperref}
    \hypersetup{colorlinks=true, linkcolor=blue, citecolor=blue, filecolor=blue,
                urlcolor=blue, unicode=false}
    \urlstyle{same}
                                


\begin{document}

\title{Software Requirements Specification for \progname: subtitle describing software} 
\author{\authname}
\date{\today}
	
\maketitle

~\newpage

\pagenumbering{roman}

\tableofcontents

~\newpage

\section*{Revision History}

\begin{tabularx}{\textwidth}{p{3cm}p{2cm}X}
\toprule {\textbf{Date}} & {\textbf{Version}} & {\textbf{Notes}}\\
\midrule
10/11/2024 & Revision 0 & Initial draft of SRS\\
\bottomrule
\end{tabularx}

~\\

~\newpage
\section{Purpose of the Project}
\subsection{User Business}
The McMaster Engineering Society is looking to develop a finance and accounting system that will streamline the financial operations of 60 student groups. Due to recent struggles managing and keeping track of reimbursement requests, many students must wait large wait times during the process. The implementation of our solution will allow the MES to provide users with adequate tracking of reimbursement requests made as well as reduce the loss of requests and lower wait times.


\subsection{Goals of the Project}
Currently, the MES is losing track of reimbursement requests made resulting in large wait times and loss of money. We aim to provide a solution that effectively manages and organizes the large throughput of reimbursement requests sent. The effectiveness of our solution will be measured based on how well it reduces wait times as well as how many requests are lost.
Additionally, we plan on implementing custom budget creation and tracking for all users. Through this, users will be able to effectively create and manage budgets for different categories and all transactions will be tracked in real time. This would allow users to gain financial oversight, ultimately preventing going over budget. Custom budgets should be categorized correctly nine times out of ten. The system should also reduce reimbursement turnover time for users, ultimately improving the claim process. Finally, the final solution should aim to be easy for users to understand and use. There should be an intuitive way to create reimbursement requests and if required, users will be provided extra instructions where needed.

\section{Stakeholders}
\subsection{Client}
The client for this project is the McMaster Engineering Society (MES). The MES financially supports 60 different student groups and numerous individuals throughout the year through a reimbursement model. The society is responsible for approving the final product and ensuring it meets the needs of the student groups.
The MES invests in this platform to streamline financial processes, reduce administrative overhead, and improve user experience for students and administrators.
Roles and responsibilities: Maintain the server that will host the tool. Process and complete the reimbursement requests

\subsection{Customer}
The primary customers of this platform are the student groups and administrative personal who utilize the reimbursement system. Their roles and responsibilities include:
\begin{itemize}
    \item \textbf{Expense Submission:} Student leaders and members will be submitting expense reports, requiring a user-friendly interface that simplifies the process.
    \item \textbf{Feedback Providers:} Customers will offer feedback on system usability and required features, which is critical for ensuring the platform meets their expectations.
    \item \textbf{User Adoption:} The willingness of these customers to adopt the new system will largely depend on its functionality and ease of use, impacting the overall success of the project.
\end{itemize}

\subsection{Other Stakeholders}
Other stakeholders include individuals and groups that may influence or be impacted by the product. Their roles are as follows:
\begin{itemize}
    \item \textbf{Student Group Leaders:} Responsible for managing their group's finances, they will provide insights into the specific needs and challenges they face in the current reimbursement process.
    \item \textbf{MES Administrative Staff:} These individuals assist in processing reimbursements and will need training on the new system to ensure smooth operations.
    \item \textbf{McMaster Administion:} The Faculty of Engineering and CCGE faculty interact with clubs often. They may also want to ensure that our tool follows their principles and standards in terms of rules, flow, cohesiveness.
    \item \textbf{Club members:} Can also submit expense receipts, will want an easy experience.
    \item \textbf{MES IT:} Involved in the integration of the platform with existing systems, they will need detailed documentation and support for maintenance.
\end{itemize}

\subsection{Hands-On Users of the Project}
Hands-on users are the individuals who will directly interact with the platform. Their characteristics include:
\begin{itemize}
    \item \textbf{Student Leaders:} Typically aged 18-25, they manage club finances, possess an intermediate amount of technological experience, and are familiar with digital forms and spreadsheets. They would prioritize an intuitive interface that simplifies expense reporting.
    \item \textbf{Administrators:} Members of the MES staff that are aged 18-25, these users manage the overall reimbursement process, have intermediate technological experience, and require efficient tools for tracking and reviewing submissions.
    \item \textbf{Financial Administrators:} Experienced professionals (aged 30+) with an advanced understanding of financial compliance, they need robust reporting features to analyze and audit transactions effectively.
\end{itemize}

\subsection{Personas}
The development of personas helps to understand user needs more concretely. Examples include:
\begin{itemize}
  \item \textbf{Alex, the Student Leader:} 20 years old, manages a student club. Familiar with Google Forms and spreadsheets, Alex seeks efficiency in submitting reimbursement requests and values features that save time, such as automated data entry.
  \item \textbf{Jamie, the MES Administrator:} 21 years old, processes reimbursements for multiple student groups. Jamie values accuracy and compliance and prefers tools that provide a clear overview of submissions and their statuses.
  \item \textbf{James, the VP Finance:} 22 years old, Vice President of Finance for the MES. James values comprehensive financial oversight and strategic planning, and prefers tools that provide high-level summaries and detailed financial reports.
  \item \textbf{Jordan, the Club Member:} 19 years old, a member of a student club who occasionally submits reimbursement requests. Jordan values simplicity and ease of use, and prefers a straightforward interface for submitting and tracking requests.
\end{itemize}

\subsection{Priorities Assigned to Users}
Understanding user priorities is critical for product success:
\begin{itemize}
    \item \textbf{Key Users:} Student leaders are crucial to the project's success. They must be able to submit expenses easily and effectively in order for this project to be successful. Their requirements will take precedence due to their direct interaction with the system.
    \item \textbf{Secondary Users:} MES Administrators who facilitate the reimbursement process are equally as important, but their needs will be prioritized lower than those of key users.
    \item \textbf{Unimportant Users:} Casual users with minimal interaction, such as infrequent users of the system, will have their requirements considered last, as they do not impact the product's long-term success.
\end{itemize}

\subsection{User Participation}
User participation is vital for gathering accurate requirements:
\begin{itemize}
    \item \textbf{Requirements Gathering:} Key users will participate in interviews and surveys to provide insights into their needs and expectations. We will need to document how a typical expense and reimbursement occours, from the time it is submitted into the current expese system to the final payout of the expense to the user. This will provide a clear overview of how the current system works, what the roadblocks are and how we can optimize the tool to the needs of the users.
    \item \textbf{Usability Testing:} Hands-on users will be invited to participate in usability testing sessions to ensure the platform meets their requirements.
    \item \textbf{Regular Check-Ins:} Scheduled meetings with key users will provide ongoing opportunities for feedback and validation throughout the development process.
\end{itemize}

\subsection{Maintenance Users and Service Technicians}
Maintenance users play a vital role in the long-term success of the platform:
\begin{itemize}
    \item \textbf{MES IT Staff:} Responsible for system updates and troubleshooting, they will require detailed documentation on the platform's functionalities to effectively manage its operation. These users ensure the platform runs smoothly, addressing any technical issues that arise and implementing updates as necessary.
\end{itemize}




\section{Mandated Constraints}
\subsection{Solution Constraints}
\begin{enumerate}
  \item The app must handle the throughput of the 60 student groups to the MES without trouble.
  \item The app must inform student groups the status of their request. 
  \item The app must notify the appropriate parties of the MES for any new requests made.
  \item The app must allow student groups to upload receipts. 
  \item The app must be able to read the receipts that are uploaded by the student groups. 
  \item The app must be up and running at all times, if not a message must be shown signifying why the app is offline. 
\end{enumerate}
\subsection{Implementation Environment of the Current System}
\begin{enumerate}
  \item There is no current implement of any current system.
  \item The new system will be hosted on a laptop provided by the MES, this laptop will have access to Windows 10, and will have above average specs (eg, 4gb of RAM, RTX 2080).
\end{enumerate}
\subsection{Partner or Collaborative Applications}
\begin{enumerate}
  \item There is no current implement of any current system.
\end{enumerate}
\subsection{Off-the-Shelf Software}
\begin{enumerate}
  \item We will use an API for scanning receipts, A specific API has not been chosen yet.
\end{enumerate}
\subsection{Anticipated Workplace Environment}
\begin{enumerate}
  \item The workplace environment will be on our specific dev machines, we will use Git (A version control software) to collaborate and share code with each other.
\end{enumerate}
\subsection{Schedule Constraints}
\begin{enumerate}
  \item The final demonstration will be April 2nd, which will be our final design constraint.
\end{enumerate}
\subsection{Budget Constraints}
\begin{enumerate}
  \item The maximum budget of this would be \$750.
\end{enumerate}
\subsection{Enterprise Constraints}
\begin{enumerate}
  \item The product and code will be made entirely available to the MES.
\end{enumerate}

\section{Naming Conventions and Terminology}
\subsection{Glossary of All Terms, Including Acronyms, Used by Stakeholders Involved in the Project}

\begin{itemize}
    \item \textbf{MES}: McMaster Engineering Society. The student organization responsible for overseeing financial and academic matters for engineering students at McMaster University.    
    \item \textbf{Audit Log}: A record of all actions performed within the system, used for tracking who accessed or modified data and when these actions took place.
    \item \textbf{Club Financial Manager}: A designated member of a club responsible for managing that club's financial records and submitting reimbursement requests on behalf of the club.
    \item \textbf{Budget Allocation}: The amount of funding designated for specific clubs or activities by the MES team and is used to track and control spending.
    \item \textbf{Vendor Contracts}: Agreements between MES and external suppliers for goods or services provided to the engineering society or its member clubs.
    \item \textbf{Approval Workflow}: The process that a reimbursement request goes through before getting final approval.
    \item \textbf{Financial Dashboard}: A feature of the system that provides an overview of the current financial standing, showing budget tracking, reimbursement status, and pending requests.
    \item \textbf{CI/CD}: Continuous Integration/Continuous Deployment. A software development practice that automatically tests and deploys code changes to ensure the system is always in a working state.
\end{itemize}


\section{Relevant Facts And Assumptions}
\subsection{Relevant Facts}
\begin{itemize}
  \item There is no current application built for this project.
  \item The project will be built primarily in Typescript and NextJS.
  \item There are 60 student groups to account for.
  \item The current process for submitting checks is manual, which takes far too long.
  \item The input should be generalized, with no differentiation between student groups.
  \item The application is intended to be created with future growth in mind.
\end{itemize}



\subsection{Business Rules}
\begin{itemize}
  \item Users will be notified when their receipts are reimbursed.
  \item The site will notify the relevant parties MES when a request comes through.
  \item Users will be notified of an expected deadline for when the requests will be fulfilled.
  \item Users will be notified of any maintenance times the site will need.
  \item All financial records will be recorded and saved in the backend of the application 
\end{itemize}

\subsection{Assumptions}
\begin{itemize}
  \item Student groups and the MES are in contact and know the ground rules of what to expect from purchases.
  \item We are expecting the MES to provide us with a server to run our software.
  \item The system will be coded in Typescript and NextJS.
  \item The system will be built from scratch.
  \item The MES is expected to provide us with data for test cases.
  \item Internet is expected to be provided, as well as a suitable system with enough RAM/storage for the app. 
  \item Student groups will use the application for further validation testing. 
\end{itemize}

\section{The Scope of the Work}
\subsection{The Current Situation}
Currently, the McMaster Engineering Society (MES) manages financial operations for over 60 student groups. The existing process for handling reimbursement requests is inefficient and manual. This is due to the lack of a consolidated platform to manage this reimbursement workflow which leads to delays, loss of reimbursement requests, and difficulty in tracking budgets and expenses.

\subsection{The Context of the Work}
The goal of the MES-ERP project is to streamline the financial processes of the MES by providing a consolidated platform that combines reimbursement tracking, budget management, and financial reporting. The scope of the work includes allowing users to submit and track reimbursement requests, manage budgets, and generate financial reports. In addition, the system will include an audit trail of all transactions to aid the MES during audits.

\subsection{Work Partitioning}
\begin{itemize}
  \item \textbf{Reimbursement Submission Module:} A feature that allows users to submit receipts and financial documents for reimbursement.
  \item \textbf{Reimbursement Review and Approval:} A feature to allow MES administrators to review, approve, or reject reimbursement requests.
  \item \textbf{Audit and Compliance Module:} This feature will ensure that all transactions are logged and auditable, complying with MES financial policies.
  \item \textbf{Budget and User Dashboards:} A feature that provides users with an overview of their budgets, expenses, and pending requests.
\end{itemize}

\subsection{Specifying a Business Use Case (BUC)}
\textbf{BUC 1 - Submitting a Reimbursement Request:}
\begin{itemize}
  \item \textbf{Precondition:} The user must have access to the system and necessary documentation for submission.
  \item \textbf{Trigger:} A student group needs reimbursement for an expense.
  \item \textbf{Steps:}
  \begin{enumerate}
      \item The user logs into the MES-ERP system.
      \item The user navigates to the  `Submit Reimbursement' section.
      \item The user fills out the required fields, attaches relevant receipts or documents, and submits the form.
  \end{enumerate}
  \item \textbf{Postconditions:} The reimbursement request is logged in the system and sent for review by the MES administrators.
\end{itemize}

\textbf{BUC 2 - Review Reimbursement Requests:}
\begin{itemize}
  \item \textbf{Precondition:} A request has been submitted by a student group.
  \item \textbf{Trigger:} MES administrators receive a notification of a new reimbursement request.
  \item \textbf{Steps:}
  \begin{enumerate}
      \item The MES administrator logs into the system.
      \item The MES administrator reviews the reimbursement request and attached documents.
      \item The MES administrator updates the status of the request in the system with either an approval or rejection with feedback.
  \end{enumerate}
  \item \textbf{Postconditions:} The reimbursement request is either approved and sent for payment or rejected with feedback provided to the user.
\end{itemize}

\section{Business Data Model and Data Dictionary}
\subsection{Business Data Model}
\lips
\subsection{Data Dictionary}
\lips

\section{The Scope of the Product}

\subsection{Product Boundary}
The product boundary defines the scope of the McMaster Engineering Society (MES) Custom Financial Expense Reporting Platform. The product boundary distinguishes between functionalities to be automated by the platform and those that will remain manual or handled by other existing systems.

\begin{itemize}
    \item \textbf{Automated Functions:}
    \begin{itemize}
        \item Streamlining reimbursement requests for student groups.
        \item Managing payment requests efficiently.
        \item Facilitating intramural funding applications with a clear workflow.
        \item Saving repetitive information, such as void cheques and e-transfer email addresses.
        \item Customizing approval workflows to fit different organizational needs.
        \item Implementing audit tracking for compliance with financial regulations.
        \item Automating ledger tracking for real-time financial monitoring.
        \item Sending SMS and automated emails for updates and notifications to users.
        \item Processing invoices, receipts, payment notices, and remittance advice effectively.
    \end{itemize}
    \item \textbf{Manual Functions:}
    \begin{itemize}
        \item Initial data entry for historical records and previous expense submissions.
        \item User training and support for navigating the new platform.
        \item Handling of any non-digital submissions or documents.
    \end{itemize}
\end{itemize}

\subsection{Product Use Case Table}
The Product Use Case Table summarizes the specific functionalities (Product Use Cases, or PUCs) that the platform will support. This table identifies the main actors involved and outlines the input and output data associated with each use case.

\begin{table}[h]
    \centering
    \caption{Product Use Case Summary Table}
    \begin{tabular}{|c|l|l|l|}
        \hline
        \textbf{PUC No} & \textbf{PUC Name} & \textbf{Actor/s} & \textbf{Input \& Output} \\ \hline
        1 & Submit Reimbursement Request & Student Leader & Reimbursement Details (in) \\ \hline
        2 & Review Reimbursement Requests & Administrator & Reimbursement Request (in), Approval Status (out) \\ \hline
        3 & Track Payment Requests & Administrator & Payment Request (in), Payment Status (out) \\ \hline
        4 & Process Funding Applications & Student Leader & Funding Application (in), Approval Status (out) \\ \hline
        5 & Save Repetitive Information & Student Leader & Repetitive Info (in), Confirmation (out) \\ \hline
        6 & Generate Reports & Financial Auditor & Report Criteria (in), Generated Report (out) \\ \hline
        7 & Send Notifications & System & User Notification Preferences (in), SMS/Email Notification (out) \\ \hline
        8 & Manage Audit Trails & Financial Auditor & Audit Criteria (in), Audit Trail Report (out) \\ \hline
    \end{tabular}
\end{table}

\subsection{Individual Product Use Cases (PUCs)}
This section details the individual product use cases listed in the Product Use Case Table. Each use case outlines the scenario, the actors involved, and the interactions within the system.

\subsubsection{PUC 1: Submit Reimbursement Request}
\textbf{Actors:} Student Leader \\
\textbf{Scenario:} A student leader submits an expense reimbursement request through the platform.
\begin{itemize}
    \item The student logs into the system.
    \item They navigate to the reimbursement section and fill out the required fields (e.g., amount, purpose, and attaching receipts).
    \item Upon submission, the request is sent for review to the administrator.
    \item Confirmation of submission is displayed on the screen.
\end{itemize}

\subsubsection{PUC 2: Review Reimbursement Requests}
\textbf{Actors:} Administrator \\
\textbf{Scenario:} An administrator reviews submitted reimbursement requests.
\begin{itemize}
    \item The administrator logs into the system.
    \item They access the list of pending reimbursement requests.
    \item The administrator reviews the details and either approves or rejects the request.
    \item An automated notification is sent to the student leader regarding the decision.
\end{itemize}

\subsubsection{PUC 3: Track Payment Requests}
\textbf{Actors:} Administrator \\
\textbf{Scenario:} The administrator tracks the status of payment requests.
\begin{itemize}
    \item The administrator logs into the system.
    \item They navigate to the payment request section.
    \item The status of each request is displayed (e.g., pending, processed, completed).
\end{itemize}

\subsubsection{PUC 4: Process Funding Applications}
\textbf{Actors:} Student Leader \\
\textbf{Scenario:} A student leader processes an application for intramural funding.
\begin{itemize}
    \item The student accesses the funding application section.
    \item They fill out the required information and submit it for approval.
    \item The system sends a notification confirming the application submission.
\end{itemize}

\subsubsection{PUC 5: Save Repetitive Information}
\textbf{Actors:} Student Leader \\
\textbf{Scenario:} The student saves repetitive financial information for future use.
\begin{itemize}
    \item The student navigates to the settings section.
    \item They enter and save their bank details and contact information.
    \item The information is stored for future reimbursement requests.
\end{itemize}

\subsubsection{PUC 6: Generate Reports}
\textbf{Actors:} Financial Auditor \\
\textbf{Scenario:} A financial auditor generates financial reports for review.
\begin{itemize}
    \item The auditor selects criteria for the report (e.g., date range, type of expenses).
    \item The system generates the report and displays it for download.
\end{itemize}

\subsubsection{PUC 7: Send Notifications}
\textbf{Actors:} System \\
\textbf{Scenario:} The system sends notifications to users based on their preferences.
\begin{itemize}
    \item Upon important updates (e.g., approval of reimbursement), the system triggers notifications.
    \item Notifications are sent via SMS or email as specified by the user.
\end{itemize}

\subsubsection{PUC 8: Manage Audit Trails}
\textbf{Actors:} Financial Auditor \\
\textbf{Scenario:} The financial auditor reviews audit trails to ensure compliance.
\begin{itemize}
    \item The auditor accesses the audit trail section in the platform.
    \item They filter results based on date and type of activity.
    \item A detailed report is generated to review all activities related to financial transactions.
\end{itemize}

\section{Functional Requirements}
\subsection{Functional Requirements}

\begin{itemize}

  \item \textbf{Requirement Number}: 001
  \item \textbf{Description}: The system must allow club financial managers to submit reimbursement requests.
  \item \textbf{Rationale}: Ensures that clubs can request refunds for approved expenses.
  \item \textbf{Fit Criterion}: A club financial manager must be able to submit a reimbursement request successfully, which should appear in the MES staff's pending approval list.
  \item \textbf{Priority}: High
  \item \textbf{Originator}: Club Financial Manager
  
  \bigskip

  \item \textbf{Requirement Number}: 002
  \item \textbf{Description}: The system must allow MES staff to approve or reject reimbursement requests.
  \item \textbf{Rationale}: Ensures proper management of reimbursement requests.
  \item \textbf{Fit Criterion}: A reimbursement request must be updated with the approval or rejection status when MES staff submits their decision.
  \item \textbf{Priority}: High
  \item \textbf{Originator}: MES Finance Team

  \bigskip

  \item \textbf{Requirement Number}: 003
  \item \textbf{Description}: The system must send a notification to MES financial managers when their reimbursement requests are approved or rejected.
  \item \textbf{Rationale}: This ensures that club financial managers are informed of the status of their reimbursement requests.
  \item \textbf{Fit Criterion}: A notification must be sent within a short amount of time after approval or rejection of a reimbursement request to the person who requested it. In case of approval, this will include some form of receipt for the user confirming their reimbursement.
  \item \textbf{Priority}: Medium
  \item \textbf{Originator}: MES Finance Team

  \bigskip

  \item \textbf{Requirement Number}: 004
  \item \textbf{Description}: The system must allow administrators to access all financial records, audit logs, and user activity.
  \item \textbf{Rationale}: Administrators need full access for auditing and tracking user activity.
  \item \textbf{Fit Criterion}: Administrators must be able to view any financial records or user actions.
  \item \textbf{Priority}: High
  \item \textbf{Originator}: System Administrators
  
  \bigskip
  
  \item \textbf{Requirement Number}: 005
  \item \textbf{Description}: The system must allow users to view a reason for rejection when a reimbursement request is rejected.
  \item \textbf{Rationale}: To help users correct issues with their requests.
  \item \textbf{Fit Criterion}: On rejection, the system must display or notify the user of the rejection reason.
  \item \textbf{Priority}: Medium
  \item \textbf{Originator}: Club Financial Managers

\end{itemize}

\section{Look and Feel Requirements}
\subsection{Appearance Requirements}
\begin{enumerate}
  \item The platform must adhere to the established MES color scheme of maroon, white, and red to ensure a consistent visual identity across all MES platforms. \\
  Rationale: This creates a seamless brand experience for users who are familiar with MES's website and other communications.
  \item The platform must be fully responsive to adjust seamlessly to various screen sizes including desktop, tablet, and mobile devices. \\
  Rationale: Many users, especially students, are likely to access the platform from mobile devices. A responsive design ensures the platform is accessible to all users.
  \item  Animations should be kept minimal and used only when necessary to avoid distractions. \\
  Rationale: Minimal animation improves the user experience through feedback, without being very distracting, hence keeping the focus on the functionality of the platform.
\end{enumerate}
\subsection{Style Requirements}
\begin{enumerate}
  \item The system must not have elements that depict violence, nudity or language that is discriminatory, vulgar, or derogatory. \\ 
  Rationale: The system should not offensive or inappropriate to users.
  \item There will be no visually unpleasing elements such as colour combinations that strain the eyes. \\ 
  Rationale: There should not be any elements that make the application harder to read.
  \item The platform's tone should remain professional and formal. \\
  Rationale: A professional tone builds trust with users by emphasizing the seriousness of the platform, particularly given its purpose in handling financial workflows.
  \item Form fields should be simple and clear, with error states highlighted in red to help users easily identify and correct mistakes. \\
  Rationale: Well-designed form fields minimize confusion during data entry, while clear error messages guide users through resolving issues.
  \item Clear and intuitive icons should be used for navigation and key actions, such as submitting expenses or reviewing approvals. \\
  Rationale: Icons provide immediate visual cues that reduces the need for text explanations.
\end{enumerate}

\section{Usability and Humanity Requirements}
\subsection{Ease of Use Requirements}

\begin{enumerate}
  
  \item The solution's UI elements such as buttons and forms will adhere to consistent styling in order to allow for an easy to understand user experience. \\
  Rationale: Maintaining consistency allows user to easily understand the system while interacting with the UI.
  \item The solution will have an organized main menu that will allow users to navigate quickly between numerous tabs. \\
  Rationale: Having a platform that is easy to use requires a good navigation menu in order to reduce time taken for users when finding the tab they wish to visit.
\end{enumerate}
\subsection{Personalization and Internationalization Requirements}

\begin{enumerate}
  
  \item Users will have the ability to organize reimbursement claims based on date. \\
  Rationale: Being able to organize the claims will allow user to quickly navigate and find claims made in the past, ultimately saving time.
\end{enumerate}
\subsection{Learning Requirements}

\begin{enumerate}
  \item Users will be provided instructions and a follow along process when making their first claim \\
  Rationale: Walking the user through the reimbursement request process would effectively and efficiently teach the user on how to create requests in the future, saving time and minimizing confusion.
  
\end{enumerate}
\subsection{Understandability and Politeness Requirements}

\begin{enumerate}
  
  \item There will be small help pop-ups available to users that will provide brief explanations of certain terminology or brief descriptions of what to input. \\
  Rationale: Providing such information would allow users to understand what is being asked without having to determine it themselves.
  \item Any error messages on the platform will provide information regarding why the error is showing. \\
  Rationale: Having detailed error messages will help users to quickly troubleshoot the problem and fix their mistakes.
\end{enumerate}
\subsection{Accessibility Requirements}

\begin{enumerate}
  
  \item The product shall have an option for increased font sizes for those with visual impairments. \\
  Rationale: Allowing an adjustable font size would accomodate for users who have bad vision.
  \item The product shall have an option for adjusting colour themes and will avoid colour combinations that are not supported for those with colour blindness. \\
  Rationale: Allowing customizable colour themes would accomodate for users who are colour blind.
\end{enumerate}
\section{Performance Requirements}

\subsection{Speed and Latency Requirements}
The platform must ensure a responsive experience for users to maintain workflow efficiency and user satisfaction. Specific speed and latency requirements include:

\begin{itemize}
    \item Any interface between a user and the automated system shall have a response time that ensures a smooth user experience.
    \item The response shall be fast enough to avoid interrupting the user’s flow of thought during the submission and review processes.
    \item The system shall process and save repetitive information quickly to enhance user experience.
    \item The product shall update status parameters in a timely manner to reflect changes accurately.
    \item The system shall generate reports promptly to facilitate timely decision-making by financial auditors.
\end{itemize}

\subsection{Safety-Critical Requirements}
Given the financial nature of the platform and its handling of sensitive information, safety-critical requirements focus on data security and user privacy. These requirements include:

\begin{itemize}
    \item The platform shall comply with relevant data protection and user privacy standards.
    \item The system shall ensure that no sensitive user data is exposed during the submission or review of expense reports.
    \item User authentication processes must prevent unauthorized access to financial information.
    \item Regular security audits must be performed to ensure compliance with established safety standards.
\end{itemize}

\subsection{Precision or Accuracy Requirements}
To ensure accurate financial reporting and user trust, precision requirements for monetary transactions and data entries are essential. These requirements include:

\begin{itemize}
    \item All monetary amounts shall be accurate to a standard level of precision.
    \item The system shall ensure the accuracy of expense calculations to prevent discrepancies in reporting.
    \item The product shall validate and display error messages for any inaccurate entries immediately upon submission.
\end{itemize}

\subsection{Robustness or Fault-Tolerance Requirements}
The platform must demonstrate robustness to ensure continuous operation even during unexpected failures or high usage. Specific robustness requirements include:

\begin{itemize}
    \item The product shall continue to operate in local mode whenever it loses its connection to the central server, allowing users to save entries temporarily.
    \item The system shall provide emergency operation capabilities in case of power or network failures, ensuring that critical processes can still be completed.
    \item Automatic backups of all financial data shall occur regularly to prevent data loss.
\end{itemize}

\subsection{Capacity Requirements}
The platform must accommodate a significant number of users and transactions, ensuring it can handle the expected load. These requirements include:

\begin{itemize}
    \item The product shall cater to a large number of simultaneous users during peak usage hours, ensuring that the system remains responsive.
    \item The platform must handle a high volume of expense report submissions without degradation in performance.
    \item The database shall be capable of storing financial records for an extended period, accommodating historical data for auditing purposes.
\end{itemize}

\subsection{Scalability or Extensibility Requirements}
To meet future growth and increased demand, the platform must be designed for scalability. Key requirements include:

\begin{itemize}
    \item The product shall be capable of scaling to accommodate a growing number of users and transactions over time.
    \item The platform must handle increasing transaction volumes as the organization grows.
    \item The system architecture should be modular to facilitate the addition of new features without requiring significant redesign.
\end{itemize}

\subsection{Longevity Requirements}
The longevity of the platform is crucial for ensuring a return on investment and sustainability. Requirements include:

\begin{itemize}
    \item The product shall be expected to operate effectively for an extended period within the defined maintenance budget.
    \item The system must be designed with upgrade paths to accommodate future technology advancements without complete replacement.
    \item All components of the system must be built with durability and support in mind, with the expectation of routine maintenance.
\end{itemize}

\section{Operational and Environmental Requirements}
\subsection{Expected Physical Environment}
\begin{enumerate}
  \item The system will be in an office room.
  \item The system will be located in the McMaster Campus. 
  \item The system will be useable in any natural environment for a laptop/phone/desktop. 
\end{enumerate}
\subsection{Wider Environment Requirements}
\begin{enumerate}
  \item The system will make the MES limit the use of Google Forms/Excel for reimbursements and adopt the application for these requests.
\end{enumerate}

\subsection{Requirements for Interfacing with Adjacent Systems}
\begin{enumerate}
  \item The app will be useable on a desktop, laptop or phone.
  \item The app will be running on Windows, Mac and Linux. 
  \item The app will work on the last 4 versions of Chrome, Firefox and Edge. 
\end{enumerate}
\subsection{Productization Requirements}
\begin{enumerate}
  \item N/A (this product is not for sale) 
\end{enumerate}
\subsection{Release Requirements}
\begin{enumerate}
  \item The product will follow an agile software development lifestyle.
  \item The product will only have maintenance is a crucial bug is found. 
  \item Products must be tested before each release. 
  \item A bug found will be made into a ticket which will be given a deadline to solve by, and then a new release build will be made.  
\end{enumerate}

\section{Maintainability and Support Requirements}
\subsection{Maintenance Requirements}

\begin{enumerate}
  
  \item If any changes to the reimbursement process is made in the future or if the number of student groups increase, the software will be easily extensible to allow for future development. \\
  Rationale: Keeping the program extensible will make the program easy to maintain and add features later on in the future.
  \item Any general maintenance issues should be resolvable by hired software developers after being provided all necessary documentation. \\
  Rationale: The system should be well documented to allow for easy understanding of the software.
\end{enumerate}
\subsection{Supportability Requirements}

\begin{enumerate}
  
  \item The solution will be supported on all platforms (Windows, Linux, Mac). \\
  Rationale: Having a wide range of platforms will allow users from different operating systems to easily use the software.
  \item The service shall only run on MES authorized devices. These devices are expected to have at least 4GB ram, a modern CPU, and access to the internet. \\
  Rationale: The MES requires the solution to be ran on authorized devices due to security reasons.
\end{enumerate}
\subsection{Adaptability Requirements}
\begin{enumerate}
  
  \item Expected to run above Windows 10 or ideally be updated to latest windows/mac software. \\
  \item Expected to be designed for possible expansion (increase in users etc) \\
  \item Expected to run on provided MES laptops and to support all platforms (Mac, Windows, Linux)
\end{enumerate}





\section{Security Requirements}
\subsection{Access Requirements}

\begin{enumerate}
  \item Administrators must have full access to all system functionalities, including creating, modifying, and deleting records for all users, managing user roles, and viewing audit logs and financial reports.
  \item MES staff must have access to financial records, reimbursement requests, audit logs, and the ability to approve or reject reimbursement requests.
  \item Club financial managers must have access to their respective club's financial records, submission of reimbursement requests, and viewing of their club's budget tracking, but no access to financial data from other clubs.
  \item Only administrators and MES staff must be able to access sensitive data such as vendor contracts, budget allocations, and event financials.
  \item Club financial managers must not have access to sensitive financial information beyond their own club's data.
  \item Administrators and MES staff must be able to modify financial records, approve budgets, and manage payments, while club financial managers can only submit reimbursement requests.
  \item All access to information and functions, such as reimbursement approval and budget tracking, must be restricted according to user roles to protect the confidentiality of the financial data.
\end{enumerate}

\subsection{Integrity Requirements}

\begin{enumerate}
    \item The system must ensure that all financial data, such as budgets, expense reports, and reimbursement requests, remain accurate and have not been tampered with. This includes preventing inaccurate or unauthorized modifications to any financial records. This is to ensure that submitted data is valid and consistent with the system's requirements.
    \item The product must protect itself from attacks coming from external sources and from unintentional mistakes by authorized users. If an external attack occurs, the system should log the attempt and prevent any changes to financial data.
    \item The system shall prevent internal misuse, such as incorrect data entry, through validation rules and error handling mechanisms.
    \item To maintain data integrity, the system must perform automated backups of all important financial data. In the event of system failure, data corruption, or accidental deletion, these backups should allow the system to recover with minimal data loss. Recovery processes should be tested regularly to ensure reliability.
    \item The system shall maintain detailed logs of all actions taken within the platform, such as data modifications, approvals, or deletions. These logs will be immutable and can be used to restore data or identify the source of an error if data becomes corrupted or tampered with.
\end{enumerate}

\subsection{Privacy Requirements}

\begin{enumerate}
  \item The system must comply with all relevant privacy laws and regulations to ensure that personal data is handled in accordance with these laws.
  \item Before collecting any personal information, the system must inform users of the types of data collected, how it will be used, and who will have access to it.
  \item The system must obtain consent from users before collecting personal information. Users must have the ability to revoke data collection consent at any time, and the system must stop processing or storing the data on the user.
  \item Users must be able to view, edit, or request corrections to their personal data stored in the system.
  \item The system must notify users of any changes to its information or privacy policy. Users should have the opportunity to review and give feedback or consent to the changes before any further processing of their data.
  \item The system must protect all personal information with encryption or other security measures to ensure it cannot be accessed, modified, or deleted by unauthorized users. 
\end{enumerate}

\subsection{Audit Requirements}

\begin{enumerate}
  \item The system must retain detailed records of all user transactions, including financial submissions, approvals, modifications, and deletions, ensuring that all actions can be traced back to the responsible user.
  \item Audit logs must record the date and time of each transaction, the user involved, and the specific actions taken, ensuring full traceability of the system's usage.
  \item The audit logs must be immutable, ensuring that no user can modify or delete the audit records.
  \item The system must retain audit logs for a minimum period of one year, or as required by applicable financial regulations.
  \item The system must record login attempts, both successful and unsuccessful, to provide a complete history of user access to the system.
  \item Access to the audit logs must be restricted to authorized users, such as administrators and auditors, ensuring confidentiality of the audit data.
  \item The system must ensure that audit data is protected from unauthorized access or tampering, using security measures such as encryption and access control.
\end{enumerate}

\subsection{Immunity Requirements}

\begin{enumerate}
  \item The system must implement protective measures against unauthorized programs such as viruses, malwares, and spywares, ensuring that the system remains secure from threats.
  \item The system must regularly scan for potential security vulnerabilities and unauthorized software, using antivirus and anti-malware tools to detect threats.
  \item The system must be capable of eliminating any detected malicious software to prevent it from affecting the integrity of the system or its data.
  \item The system must provide real-time monitoring for suspicious activities that show attempted infections.
  \item The system must ensure that all software components and libraries used in the product are regularly updated with the latest security patches to reduce the risk of infection by known vulnerabilities.
\end{enumerate}

\section{Cultural Requirements}
\subsection{Cultural Requirements}
\begin{enumerate}
  \item The application must be available in English. \\
  Rationale: English is the primary language spoken by citizens in Ontario and thus shall be the primary method of communication.
  \item The application shall use British spelling. \\
  Rationale: As the application is for use within McMaster University, the predominant spelling method is using British spelling.
  \item The system must be designed to be inclusive and usable by people from diverse cultural, linguistic, and social backgrounds. This includes ensuring gender-neutral language and avoiding culturally biased terms. \\
  Rationale:  MES is a diverse community composed of students and staff from various cultural and linguistic backgrounds. The platform must reflect this diversity by avoiding assumptions about users' cultural norms or preferences.
  \item The platform should allow users to input and view dates and times in different time zones and formats (e.g., 24-hour vs 12-hour clock, day-month-year vs month-day-year formats). \\
  Rationale: MES includes international students and staff who may be accustomed to different time zones and date formats. Flexibility in how time and dates are presented enhances the usability for these diverse users.
\end{enumerate}

\section{Compliance Requirements}

\subsection{Legal Requirements}
The development of the McMaster Engineering Society (MES) Custom Financial Expense Reporting Platform must adhere to various legal requirements to ensure compliance with applicable laws and regulations.

\begin{itemize}
    \item \textbf{Data Protection Compliance:} Personal information shall be implemented in a manner that complies with the Personal Information Protection and Electronic Documents Act (PIPEDA) in Canada. This includes ensuring that user data is collected, stored, and processed securely, with consent obtained from users.
    \item \textbf{Financial Regulations:} The platform must comply with the Financial Transactions and Reports Analysis Centre of Canada (FINTRAC) regulations to prevent money laundering and the financing of terrorist activities.
    \item \textbf{Accessibility Compliance:} The product shall meet the requirements set forth by the Accessibility for Ontarians with Disabilities Act (AODA), ensuring that all users, regardless of their abilities, can access and utilize the platform effectively.
    \item \textbf{Fit Criterion:} A legal opinion must be obtained confirming that the product does not violate any applicable laws or regulations.
\end{itemize}

\subsection{Standards Compliance Requirements}
To ensure the quality and reliability of the platform, it must comply with various industry standards. These standards are critical for avoiding delays in deployment and ensuring user trust.

\begin{itemize}
    \item \textbf{Information Security Standards:} The product shall comply with the ISO/IEC 27001 standard for information security management systems, ensuring a systematic approach to managing sensitive company and customer information.
    \item \textbf{Software Development Standards:} The development process shall follow Agile methodologies, specifically adhering to the Agile Manifesto principles, which emphasize flexibility and collaboration.
    \item \textbf{Fit Criterion:} The appropriate standard-keeper (e.g., certification bodies or industry organizations) must certify that the platform adheres to the specified standards throughout its development and operation.
\end{itemize}

\section{Open Issues}
\begin{itemize}
  \item \textbf{Issue Number}: 001
  \item \textbf{Cross-reference}: Affects Functional Requirement 003.
  \item \textbf{Summary}: The team is deciding if notifications should be sent through email, SMS, or through in-app alerts.
  \item \textbf{Stakeholders Involved}: Project Team, MES IT Team.
  \item \textbf{Action}: Perform a survey with stakeholders to determine the preferred notification method.
  \item \textbf{Resolution}: Expected decision within one week.

  \bigskip

  \item \textbf{Issue Number}: 002
  \item \textbf{Cross-reference}: Affects Functional Requirement 004.
  \item \textbf{Summary}: It is not clear how long audit logs should be stored in the system.
  \item \textbf{Stakeholders Involved}: Project Team, MES IT Team.
  \item \textbf{Action}: Stakeholders will be meeting to determine data retention policies. 
  \item \textbf{Resolution}: Expected decision within two weeks.

  \bigskip

  \item \textbf{Issue Number}: 003
  \item \textbf{Cross-reference}: Affects Functional Requirement 002.
  \item \textbf{Summary}: It is not clear whether some requests should get multiple approvals from multiple people. For example, a reimbursement request showing high levels of spending needing to be reimbursed.
  \item \textbf{Stakeholders Involved}: Project Team, MES Finance Team.
  \item \textbf{Action}: Stakeholders will be meeting to define thresholds and cases for multiple approvals.
  \item \textbf{Resolution}: Expected decision within two weeks.

  \bigskip

  \item \textbf{Issue Number}: 004
  \item \textbf{Cross-reference}: Affects Look and Feel Requirements.
  \item \textbf{Summary}: It has not yet been determined how the platform's responsiveness will be prioritized across desktop, tablet, and mobile.
  \item \textbf{Stakeholders Involved}: Project Team.
  \item \textbf{Action}: Stakeholders will be meeting to perform UI/UX design mockups to determine design priorities.
  \item \textbf{Resolution}: Expected decision within two weeks.
\end{itemize}

\section{Off-the-Shelf Solutions}
\subsection{Ready-Made Products}
\begin{enumerate}
  \item There is no pre-existing solution for this problem, this problem is MES specific and has never been done before.
\end{enumerate}
\subsection{Reusable Components}
\begin{enumerate}
  \item The main library we will use is ReactJS, there will be no other components that can be used/reused. 
\end{enumerate}
\subsection{Products That Can Be Copied}
\begin{enumerate}
  \item Any budgeting apps UI can be copied and taken over for the front end of the application. With that frontend along with some changes it can be applied to our app. 
\end{enumerate}

\section{New Problems}

\subsection{Effects on the Current Environment}
The introduction of the McMaster Engineering Society (MES) Custom Financial Expense Reporting Platform will have several effects on the current implementation environment:

\begin{itemize}
    \item \textbf{Integration with Existing Processes:} The new platform will replace the cumbersome process involving Google Forms, PDFs, and spreadsheets. This transition may initially disrupt the workflow for users accustomed to the old system, necessitating training and adaptation.
    \item \textbf{Data Migration:} Historical data from existing systems must be migrated to the new platform, which could lead to data integrity issues if not managed correctly.
    \item \textbf{Resistance to Change:} Existing users may exhibit reluctance to adopt the new platform, impacting initial usage rates and acceptance.
    \item \textbf{Preservation of Critical Operations:} It is crucial that the new product does not interfere with ongoing financial operations, such as payment processing or compliance reporting.
\end{itemize}

The goal is to identify potential conflicts early to minimize disruption during implementation.

\subsection{Effects on the Installed Systems}
The new platform will interface with existing systems at MES, and understanding these interactions is critical:

\begin{itemize}
    \item \textbf{Interface Requirements:} The new system will need to interact with existing financial software for data exchange. Specifications for data formats and protocols must be clearly defined.
    \item \textbf{Compatibility:} Ensure that the platform is compatible with existing hardware and software, minimizing the risk of conflicts or failures during operation.
    \item \textbf{Legacy System Considerations:} Users may need access to legacy systems during the transition phase, requiring well-defined procedures for accessing both systems without confusion.
\end{itemize}

A model depicting these interfaces, along with data dictionary definitions, will help clarify these relationships.

\subsection{Potential User Problems}
The transition to a new system may lead to potential user problems, which should be anticipated:

\begin{itemize}
    \item \textbf{Loss of Familiarity:} Existing users may struggle with the new interface, leading to frustration and decreased productivity.
    \item \textbf{User Error:} Inexperienced users might make errors during the transition, especially in submitting requests or processing approvals, which could impact financial accuracy.
    \item \textbf{Support Needs:} Increased demand for technical support during the initial rollout phase may overwhelm the support team, leading to delays in assistance.
\end{itemize}

Identifying these potential issues early allows for proactive measures to be taken.

\subsection{Limitations in the Anticipated Implementation Environment That May Inhibit the New Product}
Several limitations in the anticipated implementation environment may affect the success of the new platform:

\begin{itemize}
    \item \textbf{Server Capacity:} The current server infrastructure may not support the expected load during peak usage times. Upgrading server capacity may be necessary to accommodate the increased demand.
    \item \textbf{Power Requirements:} If the new platform relies on additional hardware or services, ensuring that power capabilities meet these needs is critical to avoid downtime.
    \item \textbf{User Bandwidth:} Users may require sufficient internet bandwidth to utilize the platform effectively, which may not be available in all cases.
\end{itemize}

Addressing these limitations early in the development process will facilitate a smoother implementation.

\subsection{Follow-Up Problems}
As the project progresses, several follow-up problems may emerge:

\begin{itemize}
    \item \textbf{Increased Demand:} If the new platform proves popular, there may be an unexpected surge in users, leading to scalability issues.
    \item \textbf{Compliance Challenges:} New legal regulations may emerge after implementation, necessitating additional features or modifications to ensure compliance.
    \item \textbf{Resource Allocation:} Ensuring that the necessary resources, including personnel for support and development, are available will be critical to the ongoing success of the platform.
\end{itemize}

Identifying these potential follow-up problems allows for strategic planning and resource allocation to mitigate risks.


\section{Tasks}
\subsection{Project Planning}
\begin{enumerate}
  \item The lifestyle we intend to use and develop this software is an Agile software development lifestyle.
\end{enumerate}

\subsection{Planning of the Development Phases}
\begin{longtable}{| m{3cm} | m{10cm} |}
  \hline
  \textbf{Step} & \textbf{Description} \\
  \hline
  \textbf{1. Requirements Gathering} & Stakeholders and the team collaborate to define high-level requirements and user stories. This is an ongoing process, with requirements being added or modified during the development cycle. \\
  \hline
  \textbf{2. Sprint Planning} & The team selects the user stories or tasks to be completed during the upcoming sprint. A sprint typically lasts 1 to 4 weeks. \\
  \hline
  \textbf{3. Design and Architecture} & Before starting the implementation, the team defines a general approach for the architecture and design of the system. The design may evolve as new requirements are discovered. \\
  \hline
  \textbf{4. Development} & The team works on the selected tasks, developing the functionality and features according to the user stories. Developers and testers collaborate closely. \\
  \hline
  \textbf{5. Testing} & Continuous testing is done throughout the sprint. Unit tests, integration tests, and functional tests are performed to ensure that the software meets the requirements. \\
  \hline
  \textbf{6. Review and Retrospective} & At the end of the sprint, the team holds a sprint review to showcase the completed work. A retrospective is also held to discuss improvements for the next sprint. \\
  \hline
  \textbf{7. Release Planning} & The team decides when to release new features to production. This can happen after each sprint or after several sprints, depending on the project's needs. \\
  \hline
  \textbf{8. Continuous Feedback} & The team gathers feedback from stakeholders and users to refine the product in subsequent sprints, ensuring the software evolves according to user needs. \\
  \hline
  \end{longtable}

\section{Migration to the New Product}
\subsection{Requirements for Migration to the New Product}
N/A - There are no current requirements to migrate data from the old product to the new product since no existing data will be imported into the new MES-ERP system. Historical reimbursement data from previous systems will be considered as legacy information and will remain in the old system, which will be referenced when necessary.

\subsection{Data That Has to be Modified or Translated for the New System}
N/A - No existing data will be modified or translated. Should it become necessary to import historical data in the future, this process will be considered during a potential extension of the system, as detailed in Section 25.

\section{Costs}

The cost estimation for the McMaster Engineering Society Custom Financial Expense Reporting Platform is based on factors such as functional and non-functional requirements, the difficulty of the business use cases, and the expected development effort.

\subsection{Labor Costs}
The total estimated labor cost is based on the following:
\begin{itemize}
    \item \textbf{Development Hours}: The estimated time required for developing the platform is between 400 to 500 hours. This estimate accounts for the implementation of core features such as reimbursement submissions, approval workflows, and financial tracking and integration with third-party services for sms notifications, and UI/UX design.
    \item \textbf{Testing Hours}: Testing the platform, including unit testing and integration testing, is expected to take 50 to 70 hours. This estimate is based on functional requirements and the need to ensure stability across many use cases.
\end{itemize}

\subsection{Monetary Costs}
While the development and testing are measured in hours, the following are expected to have real monetary costs:
\begin{itemize}
    \item \textbf{SMS Notification Costs}: Since the system will send SMS notifications, each SMS sent will have a cost. Based on typical SMS rates, the estimated cost is \$0.01 to \$0.05 per SMS, depending on the service provider. For 1,000 notifications, the cost would be in the range of \$10 to \$50.
    \item \textbf{Server Operating Costs}: The system will be hosted on a local server provided by MES. Although there is no cloud storage cost, running the server will have electricity and maintenance costs. These are expected to be minimal and can be estimated at approximately \$10 to \$20 per month.
\end{itemize}

\subsection{Additional Considerations}
\begin{itemize}
    \item \textbf{Maintenance Costs}: After development, maintenance will be required for system updates, bug fixes, and adding new features. This will likely require an extra 5-10 hours per month.
    \item \textbf{Backup Plan}: An extra 10-20 hours is set aside for unexpected issues or changes in requirements during development.
\end{itemize}

\section{Waiting Room}
\begin{enumerate}
  \item A mobile app version of the platform should be developed for easier on-the-go access to financial tracking and submissions. \\
  Rationale: A dedicated mobile app would improve accessibility for users who need to submit expenses or view budgets while away from their desktops. Although this would be beneficial, it is not critical for the platform's core functionality as responsive design already ensures usability on mobile devices.
  \item Implement machine learning-based expense categorization to automatically categorize expenses based on historical data. \\
  Rationale: Automated categorization of expenses would reduce manual entry and improve accuracy, providing an intelligent system that learns from user inputs. This is a technically complex feature that could be revisited after the platform's initial release.
  \item Provide users with customizable dashboards where they can select and prioritize the financial metrics most relevant to them. \\
  Rationale: Customizable dashboards allow users to personalize their experience and focus on the data that matters most to them. While this is an attractive feature, it can be delayed until after the core dashboard functionalities are developed.
  \item Enable multi-language functionality, allowing users to switch between different languages for a more inclusive experience. \\
  Rationale: MES is a diverse community with users from various linguistic backgrounds. Multi-language support would enhance accessibility and usability for all users. However, this feature is not critical for the platform's initial release and can be considered in future updates.
  \item The system should include functionality to import historical reimbursement data from the previous system, allowing all past transactions and records to be accessible within the new MES-ERP platform. \\
  Rationale: Historical data migration will allow users to reference past financial transactions within the new system, ensuring continuity and easy access to older records. This feature, while not critical to the initial deployment, adds value by consolidating all financial data into a single platform which reduces the need for referencing multiple systems.
\end{enumerate}

\section{Ideas for Solution}
A solution for this will be a web-based finance and accounting system using Typescript, NextJS, React, and Git to address the MES's challenges in managing reimbursement requests for 60 student groups. This system will streamline financial operations by allowing students to submit reimbursement requests with modular receipt uploads, real-time status tracking, and automated notifications. It will also feature live ledger tracking, custom budget creation, and support for multiple approval levels, ensuring transparency and reducing delays in request processing. This solution will greatly increase the efficiency of the old system the MES uses (excel files). Typescript will be used as the primary programming language. NextJS will serve as the framework for both front-end and back-end. React will also be used to create a responsive and interactive experience for the app. Git will manage version control, allowing the development team to share code, track changes, and collaborate efficiently without overwriting each other’s work. 
\newpage{}
\section*{Appendix --- Reflection}

The information in this section will be used to evaluate the team members on the
graduate attribute of Lifelong Learning.  Please answer the following questions:

\begin{enumerate}
  \item What knowledge and skills will the team collectively need to acquire to
  successfully complete this capstone project?  Examples of possible knowledge
  to acquire include domain specific knowledge from the domain of your
  application, or software engineering knowledge, mechatronics knowledge or
  computer science knowledge.  Skills may be related to technology, or writing,
  or presentation, or team management, etc.  You should look to identify at
  least one item for each team member.
  \item For each of the knowledge areas and skills identified in the previous
  question, what are at least two approaches to acquiring the knowledge or
  mastering the skill?  Of the identified approaches, which will each team
  member pursue, and why did they make this choice?
\end{enumerate}

\end{document}