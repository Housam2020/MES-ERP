\documentclass{article}

\usepackage{tabularx}
\usepackage{booktabs}

\title{Problem Statement and Goals\\\progname}

\author{\authname}

\date{}

%% Comments

\usepackage{color}

\newif\ifcomments\commentstrue %displays comments
%\newif\ifcomments\commentsfalse %so that comments do not display

\ifcomments
\newcommand{\authornote}[3]{\textcolor{#1}{[#3 ---#2]}}
\newcommand{\todo}[1]{\textcolor{red}{[TODO: #1]}}
\else
\newcommand{\authornote}[3]{}
\newcommand{\todo}[1]{}
\fi

\newcommand{\wss}[1]{\authornote{blue}{SS}{#1}} 
\newcommand{\plt}[1]{\authornote{magenta}{TPLT}{#1}} %For explanation of the template
\newcommand{\an}[1]{\authornote{cyan}{Author}{#1}}

%% Common Parts

\newcommand{\progname}{ProgName} % PUT YOUR PROGRAM NAME HERE
\newcommand{\authname}{Team \#, Team Name
\\ Student 1 name
\\ Student 2 name
\\ Student 3 name
\\ Student 4 name} % AUTHOR NAMES                  

\usepackage{hyperref}
    \hypersetup{colorlinks=true, linkcolor=blue, citecolor=blue, filecolor=blue,
                urlcolor=blue, unicode=false}
    \urlstyle{same}
                                


\begin{document}

\maketitle

\begin{table}[hp]
\caption{Revision History} \label{TblRevisionHistory}
\begin{tabularx}{\textwidth}{llX}
\toprule
\textbf{Date} & \textbf{Developer(s)} & \textbf{Change}\\
\midrule
Sep 24th, 2024 & Omar & Problem Statment and all subsections \\
Sep 24th, 2024 & Rachid, Sufyan & Sections 2, 3, and 4 \\
Sep 24th, 2024 & Omar, Rachid, Sufyan, Housam, Taaha & Appendix \\
\bottomrule
\end{tabularx}
\end{table}

\section{Problem Statement}

\subsection{Problem Statement}

\hspace{0.5cm} The McMaster Engineering Society is looking to develop a finance and accounting system that will streamline the financial operations of 60 student groups. Currently, the financial operations involve inefficient, fragmented processes that make it difficult to track budgets, manage expenses, and process reimbursement requests. The platform created will intake reimbursement receipts to help track budgets, manage expense reporting, and streamline reimbursement requests in an orderly fashion. By offering custom budget creation, live ledger tracking, modular receipt submission, and multiple approval levels, the system will provide a single, accessible platform by outputting reimbursement requests and responses for all financial matters.

\subsection{Problem}

\hspace{0.5cm} The main problem for the MES (McMaster Engineering Society) is 
the large throughput of the reimbursement reuqests and the lack of organization of such requests. 
This makes them lose track of the requests and the large wait times due to the large throughput can lose 
them money.  

\subsection{Inputs and Outputs}

\begin{itemize}
    \item Inputs: Reimbursement receipts provided to the McMaster Engineering Society (MES).
    \item Outputs: Requests for all financial matters and responses back to the student groups.
\end{itemize}

\subsection{Stakeholders}

\hspace{0.5cm} The stakeholders for this project would be the MES and the student groups that have reimbursement requests. 

\subsection{Environment}

\hspace{0.5cm} We intend for our solution to be usable on all platforms (Windows, Linux, and Mac), but not on mobile devices; it will be hosted on a laptop given by the MES. The laptop hosting the information should be a more modern laptop with at least 4 GB of RAM, a modern CPU, and access to the internet. The frontend of the program will be programmed in React, with the backend being in NextJS.

\section{Goals}

\begin{enumerate}
    \item Custom Budget Creation and Tracking \\
    Users will be able to create and manage budgets for different categories and transactions will be tracked in real time. This allows users to gain comprehensive financial oversight, which helps prevent going over budget. Custom budgets created should be categorized correctly 95\% of the time.
    \item Live Ledger Tracking \\
    A ledger updated continuously will be shown and will reflect all financial transactions such as expenses and budget allocations. This will allow users to monitor their financial standing at any time. The ledger should accurately show 100\% of processed transactions without significant delay.
    \item Modular Receipt Submission \\
    Users will be able to submit receipts in different file formats and will be allowed to automatically associate these with specific transactions. This simplifies the reimbursement process by having all receipts in one place. All uploaded receipts must be linked to transactions.
    \item Multi-level Approval Workflows \\
    A multi-step approval process will be implemented for expense reports and reimbursement requests. This provides a structured review process that will reduce errors and ensure compliance with policies. At least 90\% of expense reports must be approved or rejected within the defined time frame.
    \item Automated Notifications for Expense Tracking \\
    Automated email or SMS notification will be sent for budget updates, expense approvals, and reimbursements. This ensures that users will stay informed about the status of their financial requests without needing to manually follow-up. 95\% of users must receive notifications within 10 minutes of a status change in their submissions.
\end{enumerate}

\section{Stretch Goals}

\begin{enumerate}
    \item Customizable Approval Workflows \\
    Allow custom approval workflows that enable different student groups to customize the approval workflows according to their unique needs for expense and budget approvals. This will provide flexibility, allowing for the platform to scale and conform to the diverse requirements of various teams. This will be considered successful if 95\% of users are able to create workflows well suited to their needs without errors or complaints.
    \item Audit and Compliance Tracking \\
    Implement detailed audit logs for all transactions, approvals, and modifications to ensure accountability and compliance with financial regulations. This will ensure that all financial activities are traceable and meet compliance standards, reducing risks and delays during audit processes. The audit log should record 100\% of relevant activities, with full traceability and no gaps in tracking.
    \item User Dashboard with Financial Overview \\
    Develop a dashboard that provides users with real-time summaries of their budgets, expenses, pending reimbursements, and financial approvals. This will offer users a comprehensive view of their financial standing at any given time which will help gauge future decisions and reflection on past data. The dashboard accurately displays data for 100\% of transactions, with a refresh rate under 5 minutes.
    \item Machine Learning Based Expense Categorization \\
    Train and implement a machine learning algorithm to automatically categorize expenses based on historical reimbursement data. This will reduce the burden of manual input and repetitiveness on users. The trained model will be considered a success if it can achieve an accuracy rate of at least 85\%.
\end{enumerate}

\section{Challenge Level and Extras}
\subsection{Challenge Level: General}

\hspace{0.5cm}The general challenge level comes from the project requiring to create multiple complex features such as 
real-time ledger tracking, modular receipt submissions, multi-level approval workflows, and automated 
notifications all while creating a well designed user interface that allows users to immediately find 
and use these features. Another challenge also comes from integrating TypeScript and Next.js with existing 
MES applications. This must be done while maintaining stability and performance. 

\subsection{Extras}

\begin{enumerate}
    \item Useability Testing \\
    Conduct formal usability testing with a variety of stakeholders (students, financial staff) to refine the interface and ensure a smooth user experience.
    \item User Documentation \\
    Create comprehensive user documentation in the form of written documentation and video tutorials that guides end-users through every step of submitting expenses, reviewing budgets, and navigating the platform.
\end{enumerate}

\newpage{}

\section*{Appendix --- Reflection}

The purpose of reflection questions is to give you a chance to assess your own
learning and that of your group as a whole, and to find ways to improve in the
future. Reflection is an important part of the learning process.  Reflection is
also an essential component of a successful software development process.  

Reflections are most interesting and useful when they're honest, even if the
stories they tell are imperfect. You will be marked based on your depth of
thought and analysis, and not based on the content of the reflections
themselves. Thus, for full marks we encourage you to answer openly and honestly
and to avoid simply writing ``what you think the evaluator wants to hear.''

Please answer the following questions.  Some questions can be answered on the
team level, but where appropriate, each team member should write their own
response:


\begin{enumerate}
    \item What went well while writing this deliverable? \\
    \textbf{Omar, muhammao, 400325041} \\
    What went well for this deliverable were the team meetings we had throughout the project. These meetings helped me understand what was needed for the project as a whole and for the deliverable itself. The assistance provided by my teammates also helped me overcome issues with Git, which smoothed out the process. \\

    \textbf{Housam Alamour, 400317089} \\
    I believe this deliverable was instrumental in organizing the team structure and ensuring that everyone knows their roles in the team and what is expected of them. What went really well was being able to establish a consistent line of communication with all the group members and having the first encounter where we all together well. We would each work on some part of the deliverable then receive valuable feedback to continue to fix it. This has set a very solid foundation for collaborating on the rest of the project. \\

    \textbf{Sufyan, motalas, 400307042} \\
    While writing this deliverable, we decided on first writing down all the information in Google Docs and then have a person push the changes that should be added to the latex document. This allowed us to completely avoid conflicts and saved us a lot of time. We co-authored commits to properly credit everyone's work. \\

    \textbf{Rachid, khneissr, 400300627} \\
    We decided to work on this deliverable in person together and quickly realized that we need to set up our development environment to compile latex into pdf documents. We worked together to get it working and quickly solved problems anyone had. This greatly minimized time spent setting up and allowed us to spend more time completing the work. \\

    \textbf{Taaha, Atif, 400322524} \\
    For this deliverable, we successfully divided the tasks evenly and efficiently among all group members. This enabled everyone to start working simultaneously without interfering with each other's progress. Additionally, our communication was strong, allowing us to quickly assist one another whenever someone encountered challenges.
    \item What pain points did you experience during this deliverable, and how
    did you resolve them? (Done as group) \\
    The pain points for this deliverable were the issues encountered while installing LaTeX on VSCode. We all agreed that the entire team faced the following errors. We faced some errors with my environment settings, and setting it up took longer than we would have liked. Another challenge was finding the correct information from the lectures that was needed for each section of work (environment). However, once we found the relevant sections, the work was completed smoothly. Another pain point was ensuring everyone understood their role and were working on different parts of the project to ensure there was not duplicated effort. Upon encountering these issues, our team was able to discuss and resolve them in a way that ensures that they will never happen in the future, saving us time and effort later on.
    \item How did you and your team adjust the scope of your goals to ensure
    they are suitable for a Capstone project (not overly ambitious but also of
    appropriate complexity for a senior design project)? (Done as group)\\
    We adjusted our goals by ensuring that the scope of the project was not too large. The MES covers a broad range of tasks that the project could have included, but we took into account everyone’s schedules and what our project supervisor wanted to achieve with our capstone project. Using this information, we accepted a level of work that we found suitable for our team. This work included only agreeing to work first on including "Streamlining reimbursement requests, Payment requests, Intramural funding applications". We also agreed that once these initial, baseline features had been implemented we could move to tackling the other parts of the project.
\end{enumerate}  

\end{document}