\documentclass{article}

\usepackage{tabularx}
\usepackage{booktabs}

\title{Problem Statement and Goals\\\progname}

\author{\authname}

\date{}

%% Comments

\usepackage{color}

\newif\ifcomments\commentstrue %displays comments
%\newif\ifcomments\commentsfalse %so that comments do not display

\ifcomments
\newcommand{\authornote}[3]{\textcolor{#1}{[#3 ---#2]}}
\newcommand{\todo}[1]{\textcolor{red}{[TODO: #1]}}
\else
\newcommand{\authornote}[3]{}
\newcommand{\todo}[1]{}
\fi

\newcommand{\wss}[1]{\authornote{blue}{SS}{#1}} 
\newcommand{\plt}[1]{\authornote{magenta}{TPLT}{#1}} %For explanation of the template
\newcommand{\an}[1]{\authornote{cyan}{Author}{#1}}

%% Common Parts

\newcommand{\progname}{ProgName} % PUT YOUR PROGRAM NAME HERE
\newcommand{\authname}{Team \#, Team Name
\\ Student 1 name
\\ Student 2 name
\\ Student 3 name
\\ Student 4 name} % AUTHOR NAMES                  

\usepackage{hyperref}
    \hypersetup{colorlinks=true, linkcolor=blue, citecolor=blue, filecolor=blue,
                urlcolor=blue, unicode=false}
    \urlstyle{same}
                                


\begin{document}

\maketitle

\begin{table}[hp]
\caption{Revision History} \label{TblRevisionHistory}
\begin{tabularx}{\textwidth}{llX}
\toprule
\textbf{Date} & \textbf{Developer(s)} & \textbf{Change}\\
\midrule
Sep 24th, 2024 & Omar & Problem Statment and all subsections \\
Date2 & Name(s) & Description of changes\\
... & ... & ...\\
\bottomrule
\end{tabularx}
\end{table}

\section{Problem Statement}

\subsection{Problem Statement}

\hspace{0.5cm} The McMaster Engineering Society is looking to develop a finance and accounting system that will streamline the financial operations of 60 student groups. Currently, the financial operations involve inefficient, fragmented processes that make it difficult to track budgets, manage expenses, and process reimbursement requests. The platform created will intake reimbursement receipts to help track budgets, manage expense reporting, and streamline reimbursement requests in an orderly fashion. By offering custom budget creation, live ledger tracking, modular receipt submission, and multiple approval levels, the system will provide a single, accessible platform by outputting reimbursement requests and responses for all financial matters.

\subsection{Problem}

\hspace{0.5cm} The main problem for the MES (McMaster Engineering Society) is 
the large throughput of the reimbursement reuqests and the lack of organization of such requests. 
This makes them lose track of the requests and the large wait times due to the large throughput can lose 
them money.  

\subsection{Inputs and Outputs}

\begin{itemize}
    \item Inputs: Reimbursement receipts provided to the McMaster Engineering Society (MES).
    \item Outputs: Requests for all financial matters and responses back to the student groups.
\end{itemize}

\subsection{Stakeholders}

\hspace{0.5cm} The stakeholders for this project would be the MES and the student groups that have reimbursement requests. 

\subsection{Environment}

\hspace{0.5cm} We intend for our solution to be usable on all platforms (Windows, Linux, and Mac), but not on mobile devices; it will be hosted on a laptop given by the MES. The laptop hosting the information should be a more modern laptop with at least 4 GB of RAM, a modern CPU, and access to the internet. The frontend of the program will be programmed in React, with the backend being in NextJS.

\section{Goals}

\section{Stretch Goals}

\section{Challenge Level and Extras}

\wss{State your expected challenge level (advanced, general or basic).  The
challenge can come through the required domain knowledge, the implementation or
something else.  Usually the greater the novelty of a project the greater its
challenge level.  You should include your rationale for the selected level.
Approval of the level will be part of the discussion with the instructor for
approving the project.  The challenge level, with the approval (or request) of
the instructor, can be modified over the course of the term.}

\wss{Teams may wish to include extras as either potential bonus grades, or to
make up for a less advanced challenge level.  Potential extras include usability
testing, code walkthroughs, user documentation, formal proof, GenderMag
personas, Design Thinking, etc.  Normally the maximum number of extras will be
two.  Approval of the extras will be part of the discussion with the instructor
for approving the project.  The extras, with the approval (or request) of the
instructor, can be modified over the course of the term.}

\newpage{}

\section*{Appendix --- Reflection}

\wss{Not required for CAS 741}

The purpose of reflection questions is to give you a chance to assess your own
learning and that of your group as a whole, and to find ways to improve in the
future. Reflection is an important part of the learning process.  Reflection is
also an essential component of a successful software development process.  

Reflections are most interesting and useful when they're honest, even if the
stories they tell are imperfect. You will be marked based on your depth of
thought and analysis, and not based on the content of the reflections
themselves. Thus, for full marks we encourage you to answer openly and honestly
and to avoid simply writing ``what you think the evaluator wants to hear.''

Please answer the following questions.  Some questions can be answered on the
team level, but where appropriate, each team member should write their own
response:


\begin{enumerate}
    \item What went well while writing this deliverable? 
    \item What pain points did you experience during this deliverable, and how
    did you resolve them?
    \item How did you and your team adjust the scope of your goals to ensure
    they are suitable for a Capstone project (not overly ambitious but also of
    appropriate complexity for a senior design project)?
\end{enumerate}  

\end{document}