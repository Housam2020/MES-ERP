\documentclass[12pt, titlepage]{article}

\usepackage{amsmath, mathtools}

\usepackage[round]{natbib}
\usepackage{amsfonts}
\usepackage{amssymb}
\usepackage{graphicx}
\usepackage{colortbl}

\usepackage{hyperref}
\usepackage{xr}
\externaldocument{../../../SRS/SRS}

\usepackage{longtable}
\usepackage{xfrac}
\usepackage{tabularx}
\usepackage{float}
\usepackage{siunitx}
\usepackage{booktabs}
\usepackage{multirow}
\usepackage[section]{placeins}
\usepackage{caption}
\usepackage{fullpage}
\usepackage{color}
\usepackage[normalem]{ulem}
\usepackage{url}

\hypersetup{
bookmarks=true,     % show bookmarks bar? - Set only once if possible
colorlinks=true,       % false: boxed links; true: colored links
linkcolor=red,          % color of internal links (change box color with linkbordercolor)
citecolor=blue,      % color of links to bibliography
filecolor=magenta,  % color of file links
urlcolor=cyan,          % color of external links
pdftex, % Explicitly set driver
breaklinks=true % Allow links to break across lines
}

\usepackage{array}


%% Comments

\usepackage{color}

\newif\ifcomments\commentstrue %displays comments
%\newif\ifcomments\commentsfalse %so that comments do not display

\ifcomments
\newcommand{\authornote}[3]{\textcolor{#1}{[#3 ---#2]}}
\newcommand{\todo}[1]{\textcolor{red}{[TODO: #1]}}
\else
\newcommand{\authornote}[3]{}
\newcommand{\todo}[1]{}
\fi

\newcommand{\wss}[1]{\authornote{blue}{SS}{#1}} 
\newcommand{\plt}[1]{\authornote{magenta}{TPLT}{#1}} %For explanation of the template
\newcommand{\an}[1]{\authornote{cyan}{Author}{#1}}

%% Common Parts

\newcommand{\progname}{ProgName} % PUT YOUR PROGRAM NAME HERE
\newcommand{\authname}{Team \#, Team Name
\\ Student 1 name
\\ Student 2 name
\\ Student 3 name
\\ Student 4 name} % AUTHOR NAMES                  

\usepackage{hyperref}
    \hypersetup{colorlinks=true, linkcolor=blue, citecolor=blue, filecolor=blue,
                urlcolor=blue, unicode=false}
    \urlstyle{same}
                                


\providecommand{\mref}[1]{M\ref{#1}}
\providecommand{\SecMD}{5}

\begin{document}

\title{Module Interface Specification for \progname{}}

\author{\authname}

\date{\today}

\maketitle

\pagenumbering{roman}

\section{Revision History}

\begin{tabularx}{\textwidth}{p{3cm}p{2cm}X}
\toprule {\bf Date} & {\bf Version} & {\bf Notes}\\
\midrule
Date 1 & 1.0 & Notes\\
Date 2 & 1.1 & Notes\\
\bottomrule
\end{tabularx}

\newpage % Added page break

\section{Symbols, Abbreviations and Acronyms}

See SRS Documentation. % Simplified reference

\textcolor{red}{Additional symbols used in this MIS:}
\renewcommand{\arraystretch}{1.2}
\begin{tabular}{l l}
  \toprule
  \textbf{Symbol} & \textbf{Description}\\
  \midrule
  $\mathbb{Z}$ & Set of integers \\
  $\mathbb{R}$ & Set of real numbers \\
  $\mathbb{N}$ & Set of natural numbers (positive integers) \\
  $\mathbb{N}_0$ & Set of non-negative integers (0, 1, 2, ...) \\ % Added N_0
  seq of T & Sequence containing elements of type T \\
  T $\times$ U & Tuple containing elements of type T and U \\
  := & Assignment \\
  (c $\Rightarrow$ r) & Conditional rule (if condition c holds, result r applies) \\
  Map(K $\rightarrow$ V) & Map from keys of type K to values of type V \\ % Added Map
  Maybe<T> & Type T or null/nothing \\ % Added Maybe
  Partial<T> & A record where some fields of Type T may be present \\ % Added Partial
  \texttt{Blob} & \textcolor{red}{Represents binary data (e.g., file content)} \\ % Added Blob
  \texttt{Timestamp} & \textcolor{red}{String representing ISO 8601 date-time} \\ % Added Timestamp
  \bottomrule
\end{tabular}

% Removed \wss comment

\newpage

\tableofcontents

\newpage

\pagenumbering{arabic}

\section{Introduction}

The following document details the Module Interface Specifications (MIS) for the McMaster Engineering Society Custom Financial Expense Reporting Platform (MES-ERP). This platform is designed to streamline financial expense management for the McMaster Engineering Society (MES), providing an efficient and user-friendly solution for submitting, approving, and tracking reimbursement requests.

The MES-ERP aims to address the unique financial management needs of the MES by integrating expense tracking, budget management, and policy compliance into a cohesive platform. The system ensures accurate and efficient handling of financial requests while maintaining compliance with organizational policies and university regulations.

Complementary documents to this MIS include the System Requirements Specification (SRS) and the Module Guide (MG), which provide additional context and design details. The complete documentation and implementation of the MES-ERP can be found at \url{https://github.com/Housam2020/MES-ERP}.


\section{Notation}

The structure of the MIS for modules comes from \citet{HoffmanAndStrooper1995},
with the addition that template modules have been adapted from
\cite{GhezziEtAl2003}. The mathematical notation comes from Chapter 3 of
\citet{HoffmanAndStrooper1995}. For instance, the symbol \texttt{:=} is used for assignment and conditional rules follow the form $(c_1 \Rightarrow r_1 | c_2 \Rightarrow r_2 | ... | c_n \Rightarrow r_n )$.

The following table summarizes the primitive data types used by \progname{}.

\begin{center}
\renewcommand{\arraystretch}{1.2}
\noindent
\begin{tabular}{l l p{7.5cm}}
\toprule
\textbf{Data Type} & \textbf{Notation} & \textbf{Description}\\
\midrule
character & char & a single symbol or digit\\
integer & $\mathbb{Z}$ & a number without a fractional component in $(-\infty, \infty$) \\
natural number & $\mathbb{N}$ & a number without a fractional component in $[1, \infty$) \\
real & $\mathbb{R}$ & any number in $(-\infty, \infty$)\\
boolean & Bool & True or False value \\
string & String & A sequence of characters \\
\bottomrule
\end{tabular}
\end{center}

\noindent
The specification of \progname \ uses some derived data types: sequences, strings, and
tuples. Sequences are lists filled with elements of the same data type (e.g., seq of $\mathbb{R}$). Strings
are sequences of characters. Tuples contain a list of values, potentially of
different types (e.g., String $\times$ $\mathbb{R}$). In addition, \progname \ uses functions, which
are defined by the data types of their inputs and outputs (e.g., $f : \mathbb{R} \rightarrow \text{Bool}$). Local functions are
described by giving their type signature followed by their specification. Object types may be defined using record notation, e.g., \texttt{Type = record of field1: String, field2: $\mathbb{R}$ end}.

\section{Timeline}

This section outlines the timeline for the implementation of the project. The timeline includes the development of all modules, testing, and deployment phases. Tasks are divided by modules, specifying responsibilities and key milestones.

\subsection{Development Timeline}
\begin{tabularx}{\textwidth}{|l|l|X|}
\hline
\textbf{Week} & \textbf{Task} & \textbf{Details} \\
\hline
Week 1 & Initial Planning & Team meeting to finalize requirements and review the SRS. Assign responsibilities for each module. \\
\hline
Week 2 & User Authentication and Profile Management Module & Development of secure login, roles, and basic profile updates. Begin unit testing for authentication. \\
\hline
Week 3 & Expense Submission and Tracking Module & Implement submission forms for expenses, including receipt uploads and status tracking. Start unit testing. \\
\hline
Week 4 & Budget and Funding Management Module & Develop logic for fetching budgets, validating funds, and updating department budgets. Integrate with the database module. \\
\hline
Week 5 & Approval Workflow and Review Module & Implement dynamic routing rules and approval workflows. Integrate notifications for pending approvals. Conduct unit testing. \\
\hline
\textcolor{red}{Week 6} & \textcolor{red}{Notifications \& Compliance} & \textcolor{red}{Build Notifications (\mref{mNotify}) and Policy/Compliance (\mref{mCompliance}) modules. Integrate notifications for key events.} \\
\hline
\textcolor{red}{Week 7} & \textcolor{red}{Continued Modules} & \textcolor{red}{Continue development on BH modules as needed.} \\
\hline
Week 8 & Reporting and Analytics Module & Develop functionality for generating reports \textcolor{red}{(analytics views)} and tracking usage statistics. Validate with sample data. \\
\hline
Week 9 & \textcolor{red}{Admin Panel Development} & \textcolor{red}{Begin implementation of Administrator and Configuration Panel Module (\mref{mAdminPanel}).} \\
\hline
Week 10 & \textcolor{red}{Admin Panel Continued} & \textcolor{red}{Continue implementation of Administrator and Configuration Panel Module (\mref{mAdminPanel}).} \\
\hline
Week 11 & Administrator and Configuration Panel Module & Build an admin interface for managing roles, \textcolor{red}{groups,} and system logs. Ensure ease of use. \\
\hline
Week 12 & Testing and Integration & Conduct full system testing, including integration testing across all modules. Address any bugs or performance issues. \\
\hline
Week 13 & Final Review & Review system functionality against requirements. Conduct user \textcolor{red}{acceptance} testing and gather feedback for final improvements. \\
\hline
Week 14 & Deployment & Deploy the system on university servers. Ensure documentation is complete and ready for handover. \\
\hline
\end{tabularx}

\subsection{Testing and Verification}
Testing will be conducted in multiple phases:
\begin{itemize}
    \item Unit Testing: Conducted during the implementation of each module (Weeks 2–11).
    \item Integration Testing: Performed once modules are integrated (Week 12).
    \item System Testing: Comprehensive testing of the entire system to ensure functionality and performance (Week 12–13).
    \item User Acceptance Testing: Gather feedback from end-users during Week 13 to identify potential areas for improvement.
\end{itemize}

\subsection{Responsibilities}
The following responsibilities are assigned to team members:
\begin{itemize}
    \item Module Implementation: Each team member is responsible for implementing the modules assigned to them during the initial planning phase.
    \item Documentation: All team members contribute to the MIS and ensure consistency with the SRS and MG.
    \item Testing: Shared responsibility for writing and executing test cases, with module developers performing unit tests.
    \item Deployment: Coordinated by the team lead, with support from all team members for configuration and setup.
\end{itemize}

\section{Module Decomposition}
\textcolor{red}{This section provides the Module Interface Specifications (MIS) for the modules identified in the Module Guide (MG) Section~\ref{SecMD}. The decomposition follows the HH, BH, and SD layers defined in the MG.}

\newpage
~\newpage

% --- MIS of Database Module ---
\section{MIS of Database Module (\mref{mDB})}

\subsection{Module}
Database Interaction Layer

\subsection{Uses}
None (Hardware-Hiding Module).

\subsection{Syntax}
\subsubsection{Exported Constants}
\begin{itemize}
    \item \texttt{DATABASE\_URL}: String - The URL for the database connection.
    \item \texttt{MAX\_BATCH\_SIZE}: $\mathbb{N}$ - The maximum number of records to fetch in a single query. \textcolor{red}{(e.g., 1000)}
    \item \texttt{MAX\_CONNECTIONS}: $\mathbb{N}$ - The maximum number of concurrent database connections allowed by the pool. \textcolor{red}{(e.g., 10)}
    \item \texttt{DEFAULT\_TIMEOUT}: $\mathbb{N}$ - The default timeout \textcolor{red}{in milliseconds} for database queries. \textcolor{red}{(e.g., 30000)}
\end{itemize}

\subsubsection{Exported Access Programs}
\textcolor{red}{Note: `Object` type represents a structured record or JSON-like object. `Array` represents a sequence.}
\begin{center}
    \scriptsize
    \begin{tabular}{|p{3cm}|p{4cm}|p{4cm}|p{4cm}|}
        \hline
        \textbf{Name} & \textbf{Input} & \textbf{Output} & \textbf{Exceptions} \\
        \hline
        query & queryString: String, params: Object & results: Array of Object & \textcolor{red}{DBQueryError} \\
        \hline
        insert & collection: String, document: Object & documentID: String & \textcolor{red}{DBInsertError} \\
        \hline
        update & collection: String, filter: Object, updates: Object & modifiedCount: $\mathbb{N}_0$ & \textcolor{red}{DBUpdateError} \\
        \hline
        delete & collection: String, filter: Object & deletedCount: $\mathbb{N}_0$ & \textcolor{red}{DBDeleteError} \\
        \hline
        transaction & operations: Array of \newline \{ type: String, \newline collection: String, \newline ...params \} & results: Array of Object & \textcolor{red}{DBTransactionError} \\
        \hline
    \end{tabular}
\end{center}

\subsection{Semantics}
\subsubsection{State Variables}
\begin{itemize}
    \item \textcolor{red}{\texttt{dbState}: \{Connected, Disconnected, Error\} - Represents the current connection status to the database server.}
\end{itemize}

\subsubsection{Environment Variables}
\begin{itemize}
    \item \textcolor{red}{\texttt{DatabaseServer}: Represents the external database system (e.g., Supabase PostgreSQL instance) holding the persisted data.}
\end{itemize}

\subsubsection{Assumptions}
\begin{itemize}
    \item \texttt{DatabaseServer} is accessible. \textcolor{red}{A1}
    \item Database schema (table structures, constraints) is predefined and consistent with the application's expectations. \textcolor{red}{A2}
    \item Authentication credentials (\textcolor{red}{e.g., API keys}) are securely managed externally (e.g., environment variables) and provided correctly during initialization. \textcolor{red}{A3}
    \item \textcolor{red}{Network connectivity exists between the application server and the \texttt{DatabaseServer}. A4}
\end{itemize}

\subsubsection{Access Routine Semantics}
\noindent \texttt{query(queryString, params)}:
\begin{itemize}
    \item \textbf{Transition}: \textcolor{red}{$(\texttt{dbState} = \text{Connected}) \implies$ Executes the database query specified by \texttt{queryString} with bound \texttt{params} against the \texttt{DatabaseServer}.}
    \item \textbf{Output}: \textcolor{red}{$\texttt{out} := \texttt{results}$ (Array of Object representing rows matching the query).}
    \item \textbf{Exceptions}: \textcolor{red}{$(\texttt{dbState} \neq \text{Connected}) \lor (\text{Query execution fails on \texttt{DatabaseServer}}) \implies \texttt{exc} := \text{DBQueryError}$}
    \item \textbf{Precondition}: \texttt{queryString} is valid SQL syntax for the target database. \texttt{params} structure matches placeholders in \texttt{queryString}. \textcolor{red}{(Uses \texttt{sanitizeQuery})}
    \item \textbf{Postcondition}: Returned \texttt{results} are consistent with the state of the \texttt{DatabaseServer} at the time of query execution.
\end{itemize}

\noindent \texttt{insert(collection, document)}:
\begin{itemize}
    \item \textbf{Transition}: \textcolor{red}{$(\texttt{dbState} = \text{Connected}) \land (\texttt{validateSchema}(\texttt{collection}, \texttt{document})) \implies$ Persists the \texttt{document} into the specified \texttt{collection} on the \texttt{DatabaseServer}.}
    \item \textbf{Output}: \textcolor{red}{$\texttt{out} := \texttt{documentID}$ (String representing the unique ID assigned by the \texttt{DatabaseServer}).}
    \item \textbf{Exceptions}: \textcolor{red}{$(\texttt{dbState} \neq \text{Connected}) \lor (\neg \texttt{validateSchema}(\texttt{collection}, \texttt{document})) \lor (\text{Insertion fails on \texttt{DatabaseServer}}) \implies \texttt{exc} := \text{DBInsertError}$}
    \item \textbf{Precondition}: \texttt{document} conforms to the expected schema for the \texttt{collection}. \textcolor{red}{(Uses \texttt{validateSchema})}
    \item \textbf{Postcondition}: \texttt{DatabaseServer} state reflects the newly inserted \texttt{document}.
\end{itemize}

\noindent \texttt{update(collection, filter, updates)}:
\begin{itemize}
    \item \textbf{Transition}: \textcolor{red}{$(\texttt{dbState} = \text{Connected}) \implies$ Applies the specified \texttt{updates} to documents in the \texttt{collection} on the \texttt{DatabaseServer} that match the \texttt{filter} criteria.}
    \item \textbf{Output}: \textcolor{red}{$\texttt{out} := \texttt{modifiedCount}$ ($\mathbb{N}_0$ representing the number of documents updated).}
    \item \textbf{Exceptions}: \textcolor{red}{$(\texttt{dbState} \neq \text{Connected}) \lor (\text{Update operation fails on \texttt{DatabaseServer}}) \implies \texttt{exc} := \text{DBUpdateError}$}
    \item \textbf{Precondition}: \texttt{filter} and \texttt{updates} are valid for the database schema.
    \item \textbf{Postcondition}: Matching documents on the \texttt{DatabaseServer} are updated.
\end{itemize}

\noindent \texttt{delete(collection, filter)}:
\begin{itemize}
    \item \textbf{Transition}: \textcolor{red}{$(\texttt{dbState} = \text{Connected}) \implies$ Removes documents from the \texttt{collection} on the \texttt{DatabaseServer} that match the \texttt{filter} criteria.}
    \item \textbf{Output}: \textcolor{red}{$\texttt{out} := \texttt{deletedCount}$ ($\mathbb{N}_0$ representing the number of documents removed).}
    \item \textbf{Exceptions}: \textcolor{red}{$(\texttt{dbState} \neq \text{Connected}) \lor (\text{Deletion fails on \texttt{DatabaseServer}}) \implies \texttt{exc} := \text{DBDeleteError}$}
    \item \textbf{Precondition}: \texttt{filter} is specified to prevent accidental mass deletion.
    \item \textbf{Postcondition}: Matching documents are removed from the \texttt{DatabaseServer}.
\end{itemize}

\noindent \texttt{transaction(operations)}:
\begin{itemize}
    \item \textbf{Transition}: \textcolor{red}{$(\texttt{dbState} = \text{Connected}) \implies$ Executes the sequence of \texttt{operations} (inserts, updates, deletes) as an atomic transaction on the \texttt{DatabaseServer}. All operations succeed or all are rolled back.}
    \item \textbf{Output}: \textcolor{red}{$\texttt{out} := \texttt{results}$ (Array of outputs corresponding to each operation if successful).}
    \item \textbf{Exceptions}: \textcolor{red}{$(\texttt{dbState} \neq \text{Connected}) \lor (\text{Any operation is invalid}) \lor (\text{Transaction fails on \texttt{DatabaseServer}}) \implies \texttt{exc} := \text{DBTransactionError}$}
    \item \textbf{Precondition}: All individual \texttt{operations} within the sequence are valid.
    \item \textbf{Postcondition}: \texttt{DatabaseServer} state reflects the successful completion of all operations, or remains unchanged if any operation failed.
\end{itemize}

\subsubsection{Local Functions}
\begin{itemize}
    \item \texttt{validateSchema(collection: String, document: Object) $\rightarrow$ Bool}: \textcolor{red}{Checks if the structure and data types of \texttt{document} match the expected schema for the given \texttt{collection}. (Implementation details hidden).}
    \item \texttt{sanitizeQuery(query: String) $\rightarrow$ String}: \textcolor{red}{Escapes potentially harmful characters in \texttt{query} to prevent SQL injection attacks. (Implementation details hidden).}
\end{itemize}

% --- MIS of User Authentication Module ---
\section{MIS of User Authentication \& Profile Management Module (\mref{mUserAuth})}

\subsection{Module}
User Authentication \& Profile Management

\subsection{Uses}
\begin{itemize}
    \item \textcolor{red}{Database Interaction Layer (\mref{mDB})}
\end{itemize}

\subsection{Syntax}
\subsubsection{Exported Constants}
\begin{itemize}
    \item \texttt{SESSION\_TIMEOUT}: $\mathbb{N}$ - The duration \textcolor{red}{(in seconds)} before an inactive session expires. \textcolor{red}{(e.g., 3600)}
    \item \texttt{MAX\_LOGIN\_ATTEMPTS}: $\mathbb{N}$ - Maximum number of failed login attempts before account lockout. \textcolor{red}{(e.g., 5)}
    \item \texttt{LOCKOUT\_DURATION}: $\mathbb{N}$ - Duration \textcolor{red}{(in seconds)} of account lockout after exceeding \texttt{MAX\_LOGIN\_ATTEMPTS}. \textcolor{red}{(e.g., 600)}
\end{itemize}

\subsubsection{Exported Types} % Added Types
\textcolor{red}{Type \texttt{Credentials} = record of email: String, password: String end} \\
\textcolor{red}{Type \texttt{ProfileData} = record of userID: String, email: String, fullName: String, phoneNum: String, /* etc. */ end} \\
\textcolor{red}{Type \texttt{Permissions} = record of allowedActions: seq of String end} \\
\textcolor{red}{Type \texttt{Timestamp} = String representing ISO 8601 date-time}

\subsubsection{Exported Access Programs}
\begin{center}
    \scriptsize
    \begin{tabular}{|p{3cm}|p{4cm}|p{4cm}|p{4cm}|}
        \hline
        \textbf{Name} & \textbf{Input} & \textbf{Output} & \textbf{Exceptions} \\
        \hline
        authenticate & credentials: \textcolor{red}{Credentials} & sessionToken: String & AuthenticationFailed \\
        \hline
        getProfile & userID: String & profileData: \textcolor{red}{ProfileData} & UserNotFound \\
        \hline
        updateProfile & userID: String, updates: \textcolor{red}{Partial<ProfileData>} & confirmation: Bool & InvalidProfileData \\
        \hline
        validateSession & sessionToken: String & \textcolor{red}{(isValid: Bool, userID: Maybe<String>)} & \textcolor{red}{None} \\ % Changed output and exceptions
        \hline
        \textcolor{red}{getUserPermissions} & \textcolor{red}{userID: String} & \textcolor{red}{permissions: Permissions} & \textcolor{red}{UserNotFound} \\
        \hline
        \textcolor{red}{registerUser} & \textcolor{red}{credentials: Credentials} & \textcolor{red}{userID: String} & \textcolor{red}{RegistrationFailed} \\
        \hline
        \textcolor{red}{signOut} & \textcolor{red}{sessionToken: String} & \textcolor{red}{confirmation: Bool} & \textcolor{red}{None} \\ % Changed exceptions
        \hline
    \end{tabular}
\end{center}

\subsection{Semantics}
\subsubsection{State Variables}
\begin{itemize}
    \item \texttt{sessions}: \textcolor{red}{Map (String $\rightarrow$ record of userID: String, expiry: Timestamp end)} - A mapping of active session tokens to user information and expiration times.
    \item \texttt{loginAttempts}: \textcolor{red}{Map (String $\rightarrow$ record of count: $\mathbb{N}_0$, lockoutExpiry: Timestamp end)} - A record of failed login attempts per user \textcolor{red}{(keyed by email or IP)}.
\end{itemize}

\subsubsection{Environment Variables}
\begin{itemize}
    \item \textcolor{red}{\texttt{AuthProvider}: Represents the external authentication service (Supabase Auth) which manages credentials and issues tokens.}
\end{itemize}

\subsubsection{Assumptions}
\begin{itemize}
    \item The \textcolor{red}{\texttt{AuthProvider}} is available and functioning correctly. \textcolor{red}{A5}
    \item \textcolor{red}{User emails are unique identifiers for authentication.} \textcolor{red}{A6}
    \item \textcolor{red}{Input data for profile updates is validated by the caller or within \texttt{updateProfile}. A7}
    \item \textcolor{red}{The Database module (\mref{mDB}) is available for retrieving/storing non-credential profile data. A8}
\end{itemize}

\subsubsection{Access Routine Semantics}
\noindent \texttt{authenticate(credentials)}:
\begin{itemize}
    \item \textbf{Transition}: \textcolor{red}{Verifies \texttt{credentials} against \texttt{AuthProvider}. If valid and user not locked out, generates a new session token, updates \texttt{sessions} map, resets \texttt{loginAttempts} for the user. If invalid, increments \texttt{loginAttempts}; if count exceeds \texttt{MAX\_LOGIN\_ATTEMPTS}, sets lockout expiry.}
    \item \textbf{Output}: \textcolor{red}{(Authentication succeeds) $\implies \texttt{out} := \texttt{sessionToken}$ (String)}
    \item \textbf{Exceptions}: \textcolor{red}{(Authentication fails $\lor$ User locked out) $\implies \texttt{exc} := \text{AuthenticationFailed}$}
\end{itemize}

\noindent \texttt{getProfile(userID)}:
\begin{itemize}
    \item \textbf{Transition}: \textcolor{red}{Calls \mref{mDB}::query to retrieve profile data for \texttt{userID} from the \texttt{'users'} table.}
    \item \textbf{Output}: \textcolor{red}{(User found in DB) $\implies \texttt{out} := \texttt{profileData}$ (ProfileData)}
    \item \textbf{Exceptions}: \textcolor{red}{(User not found in DB $\lor$ DB query fails) $\implies \texttt{exc} := \text{UserNotFound}$}
\end{itemize}

\noindent \texttt{updateProfile(userID, updates)}:
\begin{itemize}
    \item \textbf{Transition}: \textcolor{red}{Validates \texttt{updates} (A7). Calls \mref{mDB}::update to apply changes to the \texttt{'users'} table for the given \texttt{userID}.}
    \item \textbf{Output}: \textcolor{red}{(Update succeeds) $\implies \texttt{out} := \text{True}$}
    \item \textbf{Exceptions}: \textcolor{red}{(Invalid \texttt{updates} $\lor$ DB update fails) $\implies \texttt{exc} := \text{InvalidProfileData}$}
\end{itemize}

\noindent \texttt{validateSession(sessionToken)}:
\begin{itemize}
    \item \textbf{Transition}: \textcolor{red}{Checks if \texttt{sessionToken} exists in \texttt{sessions} map and if its expiry time has not passed.}
    \item \textbf{Output}: \textcolor{red}{(Session is valid) $\implies \texttt{out} := (\text{True}, \texttt{userID})$}. \textcolor{red}{(Session invalid or expired) $\implies \texttt{out} := (\text{False}, \text{null})$}.
    \item \textbf{Exceptions}: None \textcolor{red}{(returns validity status)}.
\end{itemize}

\noindent \textcolor{red}{\texttt{getUserPermissions(userID)}}:
\begin{itemize}
    \item \textbf{Transition}: \textcolor{red}{Calls \mref{mDB}::query to retrieve all roles associated with \texttt{userID} from \texttt{'user\_roles'}, then queries \texttt{'role\_permissions'} and \texttt{'permissions'} tables to aggregate all permission names.}
    \item \textbf{Output}: \textcolor{red}{(User found and permissions retrieved) $\implies \texttt{out} := \texttt{permissions}$ (Permissions)}
    \item \textbf{Exceptions}: \textcolor{red}{(User not found $\lor$ DB query fails) $\implies \texttt{exc} := \text{UserNotFound}$}
\end{itemize}

% --- CORRECTED BRACES AROUND \textcolor ---
\textcolor{red}{
\noindent \texttt{registerUser(credentials)}:
\begin{itemize}
    \item \textbf{Transition}: Calls \texttt{AuthProvider} to create a new user with given \texttt{credentials}. If successful, calls \mref{mDB}::insert to create a corresponding record in the \texttt{'users'} table and assigns a default role via \mref{mDB}::insert into \texttt{'user\_roles'}.
    \item \textbf{Output}: (Registration succeeds) $\implies \texttt{out} := \texttt{userID}$ (String)
    \item \textbf{Exceptions}: (Registration fails at AuthProvider $\lor$ DB insert fails) $\implies \texttt{exc} := \text{RegistrationFailed}$
\end{itemize}
} % <--- Added closing brace

\textcolor{red}{
\noindent \texttt{signOut(sessionToken)}:
\begin{itemize}
    \item \textbf{Transition}: Removes the entry corresponding to \texttt{sessionToken} from the \texttt{sessions} map. May optionally call \texttt{AuthProvider} to invalidate the token externally if applicable.
    \item \textbf{Output}: (Sign out successful or token not found) $\implies \texttt{out} := \text{True}$
    \item \textbf{Exceptions}: None (fails silently or returns True).
\end{itemize}
} % <--- Added closing brace

\subsubsection{Local Functions}
None.

% --- MIS of Expense Submission Module ---
\section{MIS of Expense Submission \& Tracking Module (\mref{mExpenseSub})}
\subsection{Module}
Expense Submission \& Tracking

\subsection{Uses}
\begin{itemize}
    \item \textcolor{red}{Database Interaction Layer (\mref{mDB})}
    \item \textcolor{red}{Data Validation Module (\mref{mValidation})}
    \item \textcolor{red}{Policy \& Compliance Management Module (\mref{mCompliance})} % Added for logging
\end{itemize}

\subsection{Syntax}
\subsubsection{Exported Constants}
\begin{itemize}
    \item \texttt{MAX\_RECEIPT\_SIZE}: $\mathbb{N}$ - The maximum file size (in \textcolor{red}{bytes}) for uploaded receipts. \textcolor{red}{(e.g., 5242880 for 5MB)}
    \item \texttt{ALLOWED\_FILE\_TYPES}: seq of String - List of accepted file formats for receipts \textcolor{red}{["pdf", "jpg", "png", "jpeg"]}.
    \item \texttt{EXPENSE\_CATEGORIES}: seq of String - Predefined expense categories ["conference", "travel", "supplies", "materials", \dots]. \textcolor{red}{(Note: May be dynamically loaded from DB instead)}
\end{itemize}

\subsubsection{Exported Types} % Added Types
\textcolor{red}{Type \texttt{ExpenseDetails} = record of group\_id: String, amount\_requested\_cad: $\mathbb{R}$, budget\_line: String, /* etc. */ end} \\
\textcolor{red}{Type \texttt{Attachment} = record of fileName: String, fileData: \texttt{Blob}, /* etc. */ end} \\ % Assuming Blob type exists
\textcolor{red}{Type \texttt{ExpenseStatus} = record of status: String, timestamp: Timestamp, /* etc. */ end} \\
\textcolor{red}{Type \texttt{ExpenseSummary} = record of requestID: String, submitterName: String, amount: $\mathbb{R}$, status: String end}

\subsubsection{Exported Access Programs}
\begin{center}
    \scriptsize
    \begin{tabular}{|p{3cm}|p{4cm}|p{4cm}|p{4cm}|}
        \hline
        \textbf{Name} & \textbf{Input} & \textbf{Output} & \textbf{Exceptions} \\
        \hline
        submitExpense & expenseDetails: \textcolor{red}{ExpenseDetails}, attachments: seq of \textcolor{red}{Attachment} & requestID: String & InvalidExpense \\
        \hline
        uploadAttachment & requestID: String, file: \textcolor{red}{Attachment} & fileID: String & FileUploadFailed \\
        \hline
        getExpenseStatus & requestID: String & status: \textcolor{red}{ExpenseStatus} & RequestNotFound \\
        \hline
        updateExpenseDetails & requestID: String, updates: \textcolor{red}{Partial<ExpenseDetails>} & confirmation: Bool & InvalidUpdate \\
        \hline
        searchExpenses & filters: Object & expenseList: seq of \textcolor{red}{ExpenseSummary} & InvalidSearch \\
        \hline
    \end{tabular}
\end{center}

\subsection{Semantics}
\subsubsection{State Variables}
\begin{itemize}
    \item None \textcolor{red}{(This module is largely stateless; request data is persisted via the DB module).}
\end{itemize}

\subsubsection{Environment Variables}
\begin{itemize}
    \item \textcolor{red}{\texttt{FileStorageProvider}: Represents the external service (e.g., Supabase Storage) used for persisting uploaded files.}
\end{itemize}

\subsubsection{Assumptions}
\begin{itemize}
    \item All monetary values \textcolor{red}{provided in \texttt{amount\_requested\_cad} are assumed correct after potential currency conversion if applicable. A9}
    \item \texttt{FileStorageProvider} is available and has sufficient capacity. \textcolor{red}{A10}
    \item Users have necessary permissions \textcolor{red}{(e.g., \texttt{create\_requests})} to submit expenses, verified by the calling context (e.g., middleware or UI). \textcolor{red}{A11}
\end{itemize}

\subsubsection{Access Routine Semantics}
\noindent \texttt{submitExpense(expenseDetails, attachments)}:
\begin{itemize}
    \item \textbf{Transition}: \textcolor{red}{Calls \mref{mValidation}::validate(\texttt{expenseDetails}). If valid, generates a unique \texttt{requestID}. Persists \texttt{expenseDetails} (with \texttt{requestID}, \texttt{userID} from context, \texttt{status}='Pending') via \mref{mDB}::insert into \texttt{payment\_requests}. For each attachment, calls \texttt{uploadAttachment}. Calls \mref{mCompliance}::logAction(type='SubmitRequest', ...).}
    \item \textbf{Output}: \textcolor{red}{(Submission successful) $\implies \texttt{out} := \texttt{requestID}$ (String)}
    \item \textbf{Exceptions}: \textcolor{red}{(Validation fails $\lor$ DB insert fails $\lor$ Any attachment upload fails) $\implies \texttt{exc} := \text{InvalidExpense}$}
    \item \textbf{Precondition}: \texttt{expenseDetails} contains required fields (e.g., amount, group\_id, budget\_line). User context provides \texttt{userID}.
    \item \textbf{Postcondition}: Expense request record created in DB with 'Pending' status. Associated attachments stored via \texttt{FileStorageProvider}. Audit log updated.
\end{itemize}

\noindent \texttt{uploadAttachment(requestID, file)}:
\begin{itemize}
    \item \textbf{Transition}: \textcolor{red}{Calls \mref{mValidation}::validateFile(\texttt{file}, \texttt{MAX\_RECEIPT\_SIZE}, \texttt{ALLOWED\_FILE\_TYPES}). If valid, generates a unique \texttt{fileID}. Stores \texttt{file.fileData} via \texttt{FileStorageProvider}. Calls \mref{mDB}::insert or update to link \texttt{fileID} and metadata (\texttt{file.fileName}) to the request identified by \texttt{requestID} (e.g., in an \texttt{attachments} table or JSONB field).} \textcolor{red}{Optionally calls internal OCR function (\texttt{\_extractAmountFromReceipt}) on \texttt{file.fileData}.}
    \item \textbf{Output}: \textcolor{red}{(Upload successful) $\implies \texttt{out} := \texttt{fileID}$ (String)}
    \item \textbf{Exceptions}: \textcolor{red}{(Validation fails $\lor$ File storage fails $\lor$ DB update fails) $\implies \texttt{exc} := \text{FileUploadFailed}$}
    \item \textbf{Precondition}: \texttt{requestID} refers to an existing request. \texttt{file} contains valid data.
    \item \textbf{Postcondition}: File is stored externally. DB record links file to the request. \textcolor{red}{Potential update to request's amount based on OCR.}
\end{itemize}

\noindent \texttt{getExpenseStatus(requestID)}:
\begin{itemize}
    \item \textbf{Transition}: \textcolor{red}{Calls \mref{mDB}::query to retrieve the status and relevant tracking info for the request identified by \texttt{requestID} from \texttt{payment\_requests}.}
    \item \textbf{Output}: \textcolor{red}{(Request found) $\implies \texttt{out} := \texttt{status}$ (ExpenseStatus)}
    \item \textbf{Exceptions}: \textcolor{red}{(Request not found $\lor$ DB query fails) $\implies \texttt{exc} := \text{RequestNotFound}$}
    \item \textbf{Postcondition}: Status information is retrieved.
\end{itemize}

\noindent \texttt{updateExpenseDetails(requestID, updates)}:
\begin{itemize}
    \item \textbf{Transition}: \textcolor{red}{Fetches current request state via \mref{mDB}. Checks if request \texttt{status} allows editing (e.g., 'Pending', 'Rejected'). Calls \mref{mValidation}::validate(\texttt{updates}). If valid and editable, calls \mref{mDB}::update to apply changes to \texttt{payment\_requests} for \texttt{requestID}. Calls \mref{mCompliance}::logAction(type='UpdateRequest', ...).}
    \item \textbf{Output}: \textcolor{red}{(Update successful) $\implies \texttt{out} := \text{True}$}
    \item \textbf{Exceptions}: \textcolor{red}{(Request not found $\lor$ Request not editable $\lor$ Validation fails $\lor$ DB update fails) $\implies \texttt{exc} := \text{InvalidUpdate}$}
    \item \textbf{Precondition}: \texttt{requestID} exists. Request is in an editable state. User context provides permissions.
    \item \textbf{Postcondition}: Expense details updated in DB. Audit log updated.
\end{itemize}

\noindent \texttt{searchExpenses(filters)}:
\begin{itemize}
    \item \textbf{Transition}: \textcolor{red}{Constructs a query based on \texttt{filters} (e.g., status, group\_id, date range). Calls \mref{mDB}::query on \texttt{payment\_requests} (potentially joining with \texttt{users}, \texttt{groups}) to retrieve matching requests.}
    \item \textbf{Output}: \textcolor{red}{(Query successful) $\implies \texttt{out} := \texttt{expenseList}$ (seq of ExpenseSummary)}
    \item \textbf{Exceptions}: \textcolor{red}{(Invalid \texttt{filters} $\lor$ DB query fails) $\implies \texttt{exc} := \text{InvalidSearch}$}
    \item \textbf{Postcondition}: List of matching expense summaries is returned.
\end{itemize}

\subsubsection{Local Functions}
\begin{itemize}
    \item \textcolor{red}{\texttt{\_extractAmountFromReceipt(fileData: Blob) $\rightarrow$ Maybe<$\mathbb{R}$>}: Internal helper that attempts OCR on the receipt data and returns an extracted amount or null. (Uses Tesseract.js library).}
\end{itemize}

% --- MIS of Approval Module ---
\section{MIS of Approval Workflow and Review Module (\mref{mApproval})}

\subsection{Module}
Approval Workflow and Review

\subsection{Uses}
\begin{itemize}
    \item \textcolor{red}{Database Interaction Layer (\mref{mDB})}
    \item Budget and Funding Management Module (\mref{mBudgetMgmt})
    \item Notifications and Communication Module (\mref{mNotify})
    \item \textcolor{red}{Policy \& Compliance Management Module (\mref{mCompliance})} % For logging
\end{itemize}

\subsection{Syntax}
\subsubsection{Exported Constants}
None.

\subsubsection{Exported Types} % Added Types
\textcolor{red}{Type \texttt{RequestStatus} = String (e.g., "Pending", "Approved", "Rejected")} \\
\textcolor{red}{Type \texttt{UserDetails} = record of userID: String, roles: seq of String, groups: seq of String end}

\subsubsection{Exported Access Programs}
\begin{center}
    \scriptsize
    \begin{tabular}{|p{3cm}|p{4cm}|p{4cm}|p{4cm}|} % Adjusted width
        \hline
        \textbf{Name} & \textbf{Input} & \textbf{Output} & \textbf{Exceptions} \\
        \hline
        \textcolor{red}{getPendingRequests} & \textcolor{red}{approverDetails: UserDetails} & \textcolor{red}{pendingList: seq of ExpenseSummary} & \textcolor{red}{AccessDenied} \\
        \hline
        addNote & requestID: String, note: String, \textcolor{red}{userID: String} & confirmation: Bool & RequestNotFound \\
        \hline
        updateStatus & requestID: String, newStatus: \textcolor{red}{RequestStatus}, \textcolor{red}{approverID: String}, \textcolor{red}{comment: String (optional)} & confirmation: Bool & InvalidStatus \\
        \hline
    \end{tabular}
\end{center}

\subsection{Semantics}
\subsubsection{State Variables}
\begin{itemize}
    \item None \textcolor{red}{(Workflow state stored in DB via \mref{mDB}).}
\end{itemize}

\subsubsection{Environment Variables}
None. % Removed DB and Notification system as they are Used modules

\subsubsection{Assumptions}
\begin{itemize}
    \item \textcolor{red}{Approver's identity and permissions (\texttt{approverDetails} or \texttt{approverID}) are validated by the calling context. A12}
    \item \textcolor{red}{The Database (\mref{mDB}), Budget (\mref{mBudgetMgmt}), Notification (\mref{mNotify}), and Compliance (\mref{mCompliance}) modules are available and functioning correctly. A13}
    \item \textcolor{red}{Approval rules (e.g., which roles can approve which requests, thresholds for multi-level approval) are defined and accessible (potentially configuration or within this module's logic). A14}
\end{itemize}

\subsubsection{Access Routine Semantics}
\noindent \textcolor{red}{\texttt{getPendingRequests(approverDetails)}}:
\begin{itemize}
    \item \textbf{Transition}: \textcolor{red}{Queries \mref{mDB} for requests with status 'Pending'. Filters the list based on \texttt{approverDetails}' roles/groups and defined approval rules (A14).}
    \item \textbf{Output}: \textcolor{red}{(Requests found matching criteria) $\implies \texttt{out} := \texttt{pendingList}$ (seq of ExpenseSummary)}
    \item \textbf{Exceptions}: \textcolor{red}{(Approver lacks permission to view any requests) $\implies \texttt{exc} := \text{AccessDenied}$}
\end{itemize}

\noindent \texttt{addNote(requestID, note, userID)}:
\begin{itemize}
    \item \textbf{Transition}: \textcolor{red}{Calls \mref{mDB}::update to add the \texttt{note} (associated with \texttt{userID} and timestamp) to the specified \texttt{requestID}'s record (e.g., in a comments field/table). Calls \mref{mCompliance}::logAction(type='AddNote', ...).}
    \item \textbf{Output}: \textcolor{red}{(Update successful) $\implies \texttt{out} := \text{True}$}
    \item \textbf{Exceptions}: \textcolor{red}{(Request not found $\lor$ DB update fails) $\implies \texttt{exc} := \text{RequestNotFound}$}
\end{itemize}

\noindent \texttt{updateStatus(requestID, newStatus, approverID, comment)}:
\begin{itemize}
    \item \textbf{Transition}: \textcolor{red}{Fetches request via \mref{mDB}. Verifies \texttt{approverID} has permission to change status based on rules (A14) and request details (e.g., amount, group) using \texttt{\_canApprove}. If \texttt{newStatus} is 'Approved', may call \mref{mBudgetMgmt}::validateFunds. If validation passes (or status is 'Rejected'), calls \mref{mDB}::update to set the new status, record \texttt{approverID}, timestamp, and \texttt{comment}. Calls \mref{mCompliance}::logAction(type='UpdateStatus', ...). Calls \mref{mNotify}::sendNotification(requestSubmitterID, newStatus).}
    \item \textbf{Output}: \textcolor{red}{(Status update successful) $\implies \texttt{out} := \text{True}$}
    \item \textbf{Exceptions}: \textcolor{red}{(Request not found $\lor$ Approver lacks permission $\lor$ Budget validation fails for 'Approved' $\lor$ Invalid \texttt{newStatus} transition $\lor$ DB update fails) $\implies \texttt{exc} := \text{InvalidStatus}$}
\end{itemize}

\subsubsection{Local Functions}
\begin{itemize}
    \item \textcolor{red}{\texttt{\_canApprove(approverDetails: UserDetails, request: ExpenseDetails) $\rightarrow$ Bool}: Internal function checking if the approver has the necessary role/group permissions based on defined rules (A14) and request properties.}
\end{itemize}

% --- Budget Module Example ---
\section{MIS of Budget and Funding Management Module (\mref{mBudgetMgmt})}

\subsection{Module}
Budget and Funding Management

\subsection{Uses}
\begin{itemize}
    \item \textcolor{red}{Database Interaction Layer (\mref{mDB})}
    \item \textcolor{red}{Policy \& Compliance Management Module (\mref{mCompliance})} % For logging budget changes
\end{itemize}

\subsection{Syntax}
\subsubsection{Exported Constants}
None.

\subsubsection{Exported Types} % Added types
\textcolor{red}{Type \texttt{BudgetLine} = record of id: Maybe<$\mathbb{N}$>, group\_id: String, line\_label: String, amount: $\mathbb{R}$, line\_type: String, order\_index: $\mathbb{N}$ end} \\
\textcolor{red}{Type \texttt{GroupWithLines} = record of id: String, name: String, total\_budget: $\mathbb{R}$, lines: seq of BudgetLine end} \\
\textcolor{red}{Type \texttt{BudgetDetails} = record of groupID: String, totalAllocated: $\mathbb{R}$, currentSpent: $\mathbb{R}$, lines: seq of BudgetLine end}

\subsubsection{Exported Access Programs}
\begin{center}
    \scriptsize
    \begin{tabular}{|p{3cm}|p{4cm}|p{4cm}|p{4cm}|} % Adjusted width
        \hline
        \textbf{Name} & \textbf{Input} & \textbf{Output} & \textbf{Exceptions} \\
        \hline
        getBudget & groupID: String & budgetDetails: \textcolor{red}{BudgetDetails} & \textcolor{red}{GroupNotFound} \\
        \hline
        validateFunds & requestAmount: $\mathbb{R}$, groupID: String & \textcolor{red}{isSufficient: Bool} & \textcolor{red}{GroupNotFound} \\
        \hline
        \textcolor{red}{getOperatingBudget} & \textcolor{red}{userID: String} & \textcolor{red}{budgetData: seq of GroupWithLines} & \textcolor{red}{AccessDenied} \\ % Added userID input
        \hline
        \textcolor{red}{saveOperatingBudget} & \textcolor{red}{budgetData: seq of GroupWithLines}, \textcolor{red}{userID: String} & \textcolor{red}{confirmation: Bool} & \textcolor{red}{SaveFailed, AccessDenied} \\ % Added userID input
        \hline
    \end{tabular}
\end{center}

\subsection{Semantics}
\subsubsection{State Variables}
\begin{itemize}
    \item None \textcolor{red}{(Budget data stored in DB via \mref{mDB})}
\end{itemize}

\subsubsection{Environment Variables}
None.

\subsubsection{Assumptions}
\begin{itemize}
    \item \textcolor{red}{Budget structures (lines, amounts) stored in the database are accurate. A15}
    \item \textcolor{red}{Permissions for viewing/modifying budgets (\texttt{userID}) are checked by the caller. A16}
\end{itemize}

\subsubsection{Access Routine Semantics}
\noindent \texttt{getBudget(groupID)}:
\begin{itemize}
    \item \textbf{Transition}: \textcolor{red}{Calls \mref{mDB}::query to retrieve group details (\texttt{total\_budget}) and associated budget lines (\texttt{operating\_budget\_lines}) for the given \texttt{groupID}.} \textcolor{red}{Calculates \texttt{currentSpent} based on associated expense lines.}
    \item \textbf{Output}: \textcolor{red}{(Group found) $\implies \texttt{out} := \texttt{budgetDetails}$ (BudgetDetails)}
    \item \textbf{Exceptions}: \textcolor{red}{(Group not found $\lor$ DB query fails) $\implies \texttt{exc} := \text{GroupNotFound}$}
\end{itemize}

\noindent \texttt{validateFunds(requestAmount, groupID)}:
\begin{itemize}
    \item \textbf{Transition}: \textcolor{red}{Calls \texttt{getBudget(groupID)}. Compares \texttt{requestAmount} against (\texttt{totalAllocated} - \texttt{currentSpent}).}
    \item \textbf{Output}: \textcolor{red}{(\texttt{totalAllocated} - \texttt{currentSpent} >= \texttt{requestAmount}) $\implies \texttt{out} := \text{True}$}. \textcolor{red}{Else $\implies \texttt{out} := \text{False}$}.
    \item \textbf{Exceptions}: \textcolor{red}{(Group not found) $\implies \texttt{exc} := \text{GroupNotFound}$}
\end{itemize}

\noindent \textcolor{red}{\texttt{getOperatingBudget(userID)}}:
\begin{itemize}
    \item \textbf{Transition}: \textcolor{red}{Checks if \texttt{userID} has permission (A16). Calls \mref{mDB}::query to retrieve all groups and all operating budget lines, ordering them appropriately.}
    \item \textbf{Output}: \textcolor{red}{$\texttt{out} := \texttt{budgetData}$ (structured representation of groups and their lines).}
    \item \textbf{Exceptions}: \textcolor{red}{(User lacks permission $\lor$ DB query fails) $\implies \texttt{exc} := \text{AccessDenied}$}
\end{itemize}

\noindent \textcolor{red}{\texttt{saveOperatingBudget(budgetData, userID)}}:
\begin{itemize}
    \item \textbf{Transition}: \textcolor{red}{Checks if \texttt{userID} has permission (A16). Iterates through \texttt{budgetData}. For each group, updates its total via \mref{mDB}::update. For each line, determines if it's new, updated, or deleted, and calls appropriate \mref{mDB}::insert, \mref{mDB}::update, or \mref{mDB}::delete operations within a transaction (\mref{mDB}::transaction). Calls \mref{mCompliance}::logAction(type='SaveBudget', userID=userID, ...).}
    \item \textbf{Output}: \textcolor{red}{(Save successful) $\implies \texttt{out} := \text{True}$}
    \item \textbf{Exceptions}: \textcolor{red}{(User lacks permission) $\implies \texttt{exc} := \text{AccessDenied}$}. \textcolor{red}{(DB transaction fails) $\implies \texttt{exc} := \text{SaveFailed}$}
\end{itemize}

\subsubsection{Local Functions}
None.

% --- Placeholder for remaining modules ---
\section{MIS of Notifications \& Communication Module (\mref{mNotify})}
\dots \textit{\textcolor{red}{[Requires similar review/updates as above: Check Uses, Syntax, State/Env Vars, Assumptions, Access Semantics]}} \dots

\section{MIS of Reporting and Analytics Module (\mref{mReporting})}
\dots \textit{\textcolor{red}{[Requires similar review/updates]}} \dots

\section{MIS of Graphical User Interface (GUI) Module (\mref{mGUI})}
\dots \textit{\textcolor{red}{[Requires similar review/updates - Note: State vars here might represent UI state like active view/filters, Env var is the Browser Window]} \dots}

\section{MIS of Policy \& Compliance Management Module (\mref{mCompliance})}
\dots \textit{\textcolor{red}{[Requires similar review/updates - Focus on logging actions and potentially validating against rules]}} \dots

\section{MIS of Administrator and Configuration Panel Module (\mref{mAdminPanel})}
\dots \textit{\textcolor{red}{[Requires similar review/updates - Focus on operations like addRole, updateGroup, etc., using other modules]}} \dots

\section{MIS of Data Validation Module (\mref{mValidation})}
\dots \textit{\textcolor{red}{[Requires similar review/updates - Focus on validation functions for different data types/structures]} \dots}

\newpage

\bibliographystyle {plainnat}
\bibliography {../../../refs/References} % Correct relative path

\newpage

% --- Keep Appendix and Reflection as is ---
\appendix
\section*{Appendix}

The purpose of reflection questions is to give you a chance to assess your own
learning and that of your group as a whole, and to find ways to improve in the
future. Reflection is an important part of the learning process.  Reflection is
also an essential component of a successful software development process.  

Reflections are most interesting and useful when they're honest, even if the
stories they tell are imperfect. You will be marked based on your depth of
thought and analysis, and not based on the content of the reflections
themselves. Thus, for full marks we encourage you to answer openly and honestly
and to avoid simply writing ``what you think the evaluator wants to hear.''

Please answer the following questions.  Some questions can be answered on the
team level, but where appropriate, each team member should write their own
response:
 % Correct relative path

\begin{enumerate}
    \item \textbf{What went well while writing this deliverable?}  
    \newline
    Rachid: Collaborating as a team was smooth, and we were able to divide the modules effectively, which streamlined the writing process. \\
    Sufyan: Our TA meeting going over the modules and discussing it helped us understand the requirements better. \\
    Housam: Identifying each module’s secrets and responsibilities early on helped maintain clarity and reduced redundancy in our design process. \\
    Taaha: Understanding a double checking our modules over with the TA helped tremendously, ensuring we had the correct approach \\
    Omar: The TA meeting went over a lot of our confusion that we had, in addition the extra alloted time allowed us to flesh out this deliverable a bit more. \\
  
    \item \textbf{What pain points did you experience during this deliverable, and how did you resolve them?}  
    \newline
    Rachid: One challenge was ensuring consistency across modules. Regular team reviews and communication helped resolve any inconsistencies. \\
    Sufyan: Coordinating between the MG and MIS was a bit challenging, since we broke up who does each PDF. \\
    Housam: Aligning anticipated changes with module-level details was challenging but resolved through a thorough review of the SRS and MG. \\
    Taaha: Some of the questions were somewhat hard to understand what to do such as the ones requiring to create diagrams. \\
    Omar: A pain point I had was ensuring consistency between documents and ensuring all ideas align with eachother. \\
  
    \item \textbf{Which of your design decisions stemmed from speaking to your client(s) or a proxy (e.g., your peers, stakeholders, potential users)? For those that were not, why, and where did they come from?}  
    \newline
    GROUP: Many decisions, like integrating notifications, came directly from stakeholder feedback. Others, like modular decomposition, were based on best practices and team experience.
  
    \item \textbf{While creating the design doc, what parts of your other documents (e.g., requirements, hazard analysis, etc.), if any, needed to be changed, and why?}  
    \newline
    GROUP: The requirements document was updated to better align with the final design, specifically in the Reporting module to include export formats.
  
    \item \textbf{What are the limitations of your solution? Put another way, given unlimited resources, what could you do to make the project better? (LO\_ProbSolutions)}  
    \newline
    Rachid: With unlimited resources, we could enhance system scalability and user interface design, making it more robust and user-friendly. \\
    Sufyan: We could also improve the integration with external systems such as banking APIs but we are limited by Open Banking not being available. \\
    Housam: With unlimited resources, we could incorporate advanced machine learning algorithms for OCR-based receipt processing to enhance accuracy and reduce manual intervention. \\
    Taaha: Provided unlimited resources, we could integrate better support with online banking organizations to allow for easier transactions. \\
    Omar: We could increase the use cases of the application, we could allow users more options based off what the MES requires or we could refine our solution to be more user friendly. We could have additionaly made a mobile app as well as a web app. \\
  
    \item \textbf{Give a brief overview of other design solutions you considered. What are the benefits and tradeoffs of those other designs compared with the chosen design? From all the potential options, why did you select the documented design? (LO\_Explores)}  
    \newline
    Rachid: We considered alternative approaches for managing the workflow logic, such as using external libraries, but chose an in-house solution for simplicity and better control over implementation. \\
    Sufyan: We also considered using a third-party notification service, but opted for an in-house system to maintain data privacy and security. \\
    Housam: We debated between centralized compliance validation versus distributed validation across modules. While centralized validation offered simplicity, we chose the distributed approach to align better with modular decomposition and scalability goals.\\
    Taaha: We considered using external libraries to manage numerous functionality such as workflow process and notifications. \\
    Omar: A design solution I considered was making a mobile app instead of a web app, this in my opinion would have made the notifcaion process easier and more friendly to users. The tradeoff being the MES did not want this initially so we would not be making an appealing app to our main stakeholder. We decided to select a web app since that would appeal to all stakeholders and we are still able to notify users via email. 
  \end{enumerate}

\end{document}
