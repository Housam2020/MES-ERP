\documentclass[12pt, titlepage]{article}

\usepackage{amsmath, mathtools}

\usepackage[round]{natbib}
\usepackage{amsfonts}
\usepackage{amssymb}
\usepackage{graphicx}
\usepackage{colortbl}
\usepackage{xr}
\usepackage{hyperref}
\usepackage{longtable}
\usepackage{xfrac}
\usepackage{tabularx}
\usepackage{float}
\usepackage{siunitx}
\usepackage{booktabs}
\usepackage{multirow}
\usepackage[section]{placeins}
\usepackage{caption}
\usepackage{fullpage}

\hypersetup{
bookmarks=true,     % show bookmarks bar?
colorlinks=true,       % false: boxed links; true: colored links
linkcolor=red,          % color of internal links (change box color with linkbordercolor)
citecolor=blue,      % color of links to bibliography
filecolor=magenta,  % color of file links
urlcolor=cyan          % color of external links
}

\usepackage{array}

\externaldocument{../../SRS/SRS}

%% Comments

\usepackage{color}

\newif\ifcomments\commentstrue %displays comments
%\newif\ifcomments\commentsfalse %so that comments do not display

\ifcomments
\newcommand{\authornote}[3]{\textcolor{#1}{[#3 ---#2]}}
\newcommand{\todo}[1]{\textcolor{red}{[TODO: #1]}}
\else
\newcommand{\authornote}[3]{}
\newcommand{\todo}[1]{}
\fi

\newcommand{\wss}[1]{\authornote{blue}{SS}{#1}} 
\newcommand{\plt}[1]{\authornote{magenta}{TPLT}{#1}} %For explanation of the template
\newcommand{\an}[1]{\authornote{cyan}{Author}{#1}}

%% Common Parts

\newcommand{\progname}{ProgName} % PUT YOUR PROGRAM NAME HERE
\newcommand{\authname}{Team \#, Team Name
\\ Student 1 name
\\ Student 2 name
\\ Student 3 name
\\ Student 4 name} % AUTHOR NAMES                  

\usepackage{hyperref}
    \hypersetup{colorlinks=true, linkcolor=blue, citecolor=blue, filecolor=blue,
                urlcolor=blue, unicode=false}
    \urlstyle{same}
                                


\begin{document}

\title{Module Interface Specification for \progname{}}

\author{\authname}

\date{\today}

\maketitle

\pagenumbering{roman}

\section{Revision History}

\begin{tabularx}{\textwidth}{p{3cm}p{2cm}X}
\toprule {\bf Date} & {\bf Version} & {\bf Notes}\\
\midrule
Date 1 & 1.0 & Notes\\
Date 2 & 1.1 & Notes\\
\bottomrule
\end{tabularx}

~\newpage

\section{Symbols, Abbreviations and Acronyms}

See SRS Documentation at \wss{give url}

\wss{Also add any additional symbols, abbreviations or acronyms}

\newpage

\tableofcontents

\newpage

\pagenumbering{arabic}

\section{Introduction}

The following document details the Module Interface Specifications (MIS) for the McMaster Engineering Society Custom Financial Expense Reporting Platform (MES-ERP). This platform is designed to streamline financial expense management for the McMaster Engineering Society (MES), providing an efficient and user-friendly solution for submitting, approving, and tracking reimbursement requests.

The MES-ERP aims to address the unique financial management needs of the MES by integrating expense tracking, budget management, and policy compliance into a cohesive platform. The system ensures accurate and efficient handling of financial requests while maintaining compliance with organizational policies and university regulations.

Complementary documents to this MIS include the System Requirements Specification (SRS) and the Module Guide (MG), which provide additional context and design details. The complete documentation and implementation of the MES-ERP can be found at \url{https://github.com/Housam2020/MES-ERP}.


\section{Notation}

The structure of the MIS for modules comes from \citet{HoffmanAndStrooper1995},
with the addition that template modules have been adapted from
\cite{GhezziEtAl2003}.  The mathematical notation comes from Chapter 3 of
\citet{HoffmanAndStrooper1995}.  For instance, the symbol := is used for a
multiple assignment statement and conditional rules follow the form $(c_1
\Rightarrow r_1 | c_2 \Rightarrow r_2 | ... | c_n \Rightarrow r_n )$.

The following table summarizes the primitive data types used by \progname. 

\begin{center}
\renewcommand{\arraystretch}{1.2}
\noindent 
\begin{tabular}{l l p{7.5cm}} 
\toprule 
\textbf{Data Type} & \textbf{Notation} & \textbf{Description}\\ 
\midrule
character & char & a single symbol or digit\\
integer & $\mathbb{Z}$ & a number without a fractional component in (-$\infty$, $\infty$) \\
natural number & $\mathbb{N}$ & a number without a fractional component in [1, $\infty$) \\
real & $\mathbb{R}$ & any number in (-$\infty$, $\infty$)\\
\bottomrule
\end{tabular} 
\end{center}

\noindent
The specification of \progname \ uses some derived data types: sequences, strings, and
tuples. Sequences are lists filled with elements of the same data type. Strings
are sequences of characters. Tuples contain a list of values, potentially of
different types. In addition, \progname \ uses functions, which
are defined by the data types of their inputs and outputs. Local functions are
described by giving their type signature followed by their specification.

\section{Timeline}

This section outlines the timeline for the implementation of the project. The timeline includes the development of all modules, testing, and deployment phases. Tasks are divided by modules, specifying responsibilities and key milestones.

\subsection{Development Timeline}
\begin{tabularx}{\textwidth}{|l|l|X|}
\hline
\textbf{Week} & \textbf{Task} & \textbf{Details} \\ 
\hline
Week 1 & Initial Planning & Team meeting to finalize requirements and review the SRS. Assign responsibilities for each module. \\ 
\hline
Week 2 & User Authentication and Profile Management Module & Development of secure login, roles, and basic profile updates. Begin unit testing for authentication. \\ 
\hline
Week 3 & Expense Submission and Tracking Module & Implement submission forms for expenses, including receipt uploads and status tracking. Start unit testing. \\ 
\hline
Week 4 & Budget and Funding Management Module & Develop logic for fetching budgets, validating funds, and updating department budgets. Integrate with the database module. \\ 
\hline
Week 5 & Approval Workflow and Review Module & Implement dynamic routing rules and approval workflows. Integrate notifications for pending approvals. Conduct unit testing. \\ 
\hline
Week 6 & Disbursement and Payment Processing Module & Create logic for issuing payments and generating logs for auditing. Integrate with external financial systems. \\ 
\hline
Week 7 & Notifications and Communication Module & Build notifications for key events, such as request approvals or denials. Include dashboard alerts for overdue actions. \\ 
\hline
Week 8 & Reporting and Analytics Module & Develop functionality for generating reports and tracking usage statistics. Validate with sample data. \\ 
\hline
Week 9 & Policy and Compliance Management Module & Implement rule validation for requests and ensure compliance checks during submission. \\ 
\hline
Week 10 & Integration with Other University Systems Module & Create API connectors for university systems (SIS, finance) and test for reliability. \\ 
\hline
Week 11 & Administrator and Configuration Panel Module & Build an admin interface for managing roles, notifications, and system logs. Ensure ease of use. \\ 
\hline
Week 12 & Testing and Integration & Conduct full system testing, including integration testing across all modules. Address any bugs or performance issues. \\ 
\hline
Week 13 & Final Review & Review system functionality against requirements. Conduct user testing and gather feedback for final improvements. \\ 
\hline
Week 14 & Deployment & Deploy the system on university servers. Ensure documentation is complete and ready for handover. \\ 
\hline
\end{tabularx}

\subsection{Testing and Verification}
Testing will be conducted in multiple phases:
\begin{itemize}
    \item Unit Testing: Conducted during the implementation of each module (Weeks 2–11).
    \item Integration Testing: Performed once modules are integrated (Week 12).
    \item System Testing: Comprehensive testing of the entire system to ensure functionality and performance (Week 12–13).
    \item User Acceptance Testing: Gather feedback from end-users during Week 13 to identify potential areas for improvement.
\end{itemize}

\subsection{Responsibilities}
The following responsibilities are assigned to team members:
\begin{itemize}
    \item Module Implementation: Each team member is responsible for implementing the modules assigned to them during the initial planning phase.
    \item Documentation: All team members contribute to the MIS and ensure consistency with the SRS.
    \item Testing: Shared responsibility for writing and executing test cases, with module developers performing unit tests.
    \item Deployment: Coordinated by the team lead, with support from all team members for configuration and setup.
\end{itemize}

\section{Module Decomposition}
The modules in this system are divided into the following levels:

\begin{table}[h!]
    \centering
    \begin{tabular}{p{0.3\textwidth} p{0.6\textwidth}}
        \toprule
        \textbf{Level 1} & \textbf{Level 2} \\
        \midrule
        Hardware-Hiding Module & Database Module \\
        \midrule
        \multirow{6}{0.3\textwidth}{Behavior-Hiding Module} & Approval Workflow and Review \\
        & Budget and Funding Management \\
        & Reporting and Analytics \\
        & Policy and Compliance Management \\
        & Notifications and Communication \\
        & Disbursement and Payment Processing \\
        \midrule
        \multirow{3}{0.3\textwidth}{Software Decision Module} & GUI Module \\
        & Integration with Other University Systems \\
        & Software Design Decisions Module \\
        \bottomrule
    \end{tabular}
    \caption{Module Decomposition}
    \label{tab:module_decomposition}
\end{table}

\newpage
~\newpage

\section{MIS of Database Module}

\subsection{Module}
Database

\subsection{Uses}
None (Hardware-Hiding Module).

\subsection{Syntax}
\subsubsection{Exported Constants}
\begin{itemize}
    \item DATABASE\_URL: The URL for the database connection.
    \item MAX\_BATCH\_SIZE: The maximum number of records to fetch in a single query.
    \item MAX\_CONNECTIONS: The maximum number of concurrent database connections.
    \item DATABASE\_CREDENTIALS: The username and password for database access.
    \item DEFAULT\_TIMEOUT: The default timeout for database queries.
\end{itemize}

\subsubsection{Exported Access Programs}
\begin{center}
    \scriptsize
    \begin{tabular}{|p{3cm}|p{4cm}|p{4cm}|p{4cm}|}
        \hline
        \textbf{Name} & \textbf{Input} & \textbf{Output} & \textbf{Exceptions} \\
        \hline
        query & queryString (\texttt{String}), params (\texttt{Object}) & results (\texttt{Array}) & QueryExecutionException \\
        \hline
        insert & collection (\texttt{String}), document (\texttt{Object}) & documentID (\texttt{String}) & InsertionException \\
        \hline
        update & collection (\texttt{String}), filter (\texttt{Object}), updates (\texttt{Object}) & modifiedCount (\texttt{Number}) & UpdateException \\
        \hline
        delete & collection (\texttt{String}), filter (\texttt{Object}) & deletedCount (\texttt{Number}) & DeletionException \\
        \hline
        transaction & operations (\texttt{Array}) & results (\texttt{Array}) & TransactionException \\
        \hline
    \end{tabular}
\end{center}

\subsection{Semantics}
\subsubsection{State Variables}
\begin{itemize}
    %connections: Pool of active database connections
    %transactions: Active transactions and their states
    %collections: Schema and metadata for database collections
    \item connections: Pool of active database connections.
    \item collections: Schema and metadata for database collections.
    \item queries: Record of executed queries and their results.
    \item inserts: Record of inserted documents and their IDs.
    \item updates: Record of updated documents and their modification counts.
    \item deletions: Record of deleted documents and their deletion counts.
    \item transactions: Record of transaction operations and their results.
    \item collections: Record of database collections and their schemas.
\end{itemize}

\subsubsection{Environment Variables}
\begin{itemize}
    \item Database Server: Hosts the database and provides access to stored data.
    \item Database Client: Connects to the database server and executes queries.
    \item Configuration: Database connection and optimization settings.
\end{itemize}

\subsubsection{Assumptions}
\begin{itemize}
    \item Database server is accessible and responsive.
    \item Database schema is predefined and consistent across collections.
    \item Database queries are validated before execution.
    \item Authentication credentials are securely stored
\end{itemize}

\subsubsection{Access Routine Semantics}
\noindent \texttt{query(queryString)}:
\begin{itemize}
    \item \textbf{Transition}: Executes a database query with the provided parameters.
    \item \textbf{Output}: Returns query results or raises QueryExecutionException.
    \item \textbf{Precondition}: Query string is valid MongoDB syntax.
    \item \textbf{Postcondition}: Results are properly sanitized and formatted.
\end{itemize}

\noindent \texttt{insert(collection, document)}:
\begin{itemize}
    \item \textbf{Transition}: Inserts a new document into the specified collection.
    \item \textbf{Output}: Returns the unique identifier of the inserted document.
    \item \textbf{Exceptions}: InsertionException if the document is invalid or the operation fails.
    \item \textbf{Precondition}: Document matches the collection schema.
\end{itemize}

\noindent \texttt{update(collection, filter, updates)}:
\begin{itemize}
    \item \textbf{Transition}: Updates documents matching the filter criteria.
    \item \textbf{Output}: Returns the count of modified documents.
    \item \textbf{Exceptions}: UpdateException if the operation fails or the updates are invalid.
    \item \textbf{Precondition}: Update operation is valid for the collection schema.
    \item \textbf{Postcondition}: Matching documents are updated atomically.
\end{itemize}

\noindent \texttt{delete(collection, filter)}:
\begin{itemize}
    \item \textbf{Transition}: Removes documents matching the filter criteria.
    \item \textbf{Output}: Returns the count of deleted documents.
    \item \textbf{Precondition}: Filter is non-empty to prevent accidental collection deletion.
    \item \textbf{Postcondition}: Matching documents are permanently removed.
\end{itemize}

\noindent \texttt{transaction(operations)}:
\begin{itemize}
    \item \textbf{Transition}: Executes multiple operations in a single transaction.
    \item \textbf{Output}: Returns results of all operations or raises TransactionException.
    \item \textbf{Precondition}: All operations are valid and follow the collection schema.
    \item \textbf{Postcondition}: All operations succeed or all are rolled back.
\end{itemize}

\subsubsection{Local Functions}
\begin{itemize}
    \item \texttt{validateSchema(collection, document) -> Boolean}: Verifies document against collection schema.
    \item \texttt{sanitizeQuery(query) -> String}: Prevents injection attacks by sanitizing query strings.
    \item \texttt{manageConnections() -> void}: Monitors and manages database connection pool.
    \item \texttt{logQuery(query, results) -> void}: Records executed queries and results.
    \item \texttt{logInsert(collection, documentID) -> void}: Records inserted documents.
    \item \texttt{logUpdate(collection, modifiedCount) -> void}: Records updated documents.
    \item \texttt{logDelete(collection, deletedCount) -> void}: Records deleted documents.
    \item \texttt{logTransaction(operations) -> void}: Records transaction operations and results.
    \item \texttt{logCollection(collection, schema) -> void}: Records collection schema and metadata.
\end{itemize}

\section{MIS of User Authentication \& Profile Management Module}

\subsection{Module}
User Authentication \& Profile Management

\subsection{Uses}
\begin{itemize}
    \item Database Module
    \item Integration with Other University Systems Module
    \item Notifications and Communication Module
\end{itemize}

\subsection{Syntax}
\subsubsection{Exported Constants}
\begin{itemize}
    \item SESSION\_TIMEOUT: The duration (in minutes) before an inactive session expires.
    \item MAX\_LOGIN\_ATTEMPTS: Maximum number of failed login attempts before account lockout.
    \item LOCKOUT\_DURATION: Duration (in minutes) of account lockout after exceeding MAX\_LOGIN\_ATTEMPTS.
\end{itemize}

\subsubsection{Exported Access Programs}
\begin{center}
    \scriptsize
    \begin{tabular}{|p{3cm}|p{4cm}|p{4cm}|p{4cm}|}
        \hline
        \textbf{Name} & \textbf{Input} & \textbf{Output} & \textbf{Exceptions} \\
        \hline
        authenticate & credentials (\texttt{Object}) & sessionToken (\texttt{String}) & AuthenticationFailedException \\
        \hline
        getProfile & userID (\texttt{String}) & profileData (\texttt{Object}) & UserNotFoundException \\
        \hline
        updateProfile & userID (\texttt{String}), updates (\texttt{Object}) & confirmation (\texttt{Boolean}) & InvalidProfileDataException \\
        \hline
        validateSession & sessionToken (\texttt{String}) & isValid (\texttt{Boolean}) & InvalidSessionException \\
        \hline
        getRolePermissions & userID (\texttt{String}) & permissions (\texttt{Object}) & RoleNotFoundException \\
        \hline
    \end{tabular}
\end{center}

\subsection{Semantics}

\subsubsection{State Variables}
\begin{itemize}
    \item sessions: A mapping of active session tokens to user information and expiration times.
    \item profiles: A mapping of user IDs to their profile information and preferences
    \item loginAttempts: A record of failed login attempts per user.
\end{itemize}

\subsubsection{Environment Variables}
\begin{itemize}
    \item Identity Provider: The university's SSO system for authentication.
    \item User Database: Stores user profiles, roles, and authentication data.
\end{itemize}

\subsubsection{Assumptions}
\begin{itemize}
    \item The university's identity provider is available and functioning
    \item Users have unique identifiers across the system
    \item Profile data is validated before storage
\end{itemize}

\subsubsection{Access Routine Semantics}
\noindent \texttt{authenticate(credentials)}:
\begin{itemize}
    \item \textbf{Transition}: Validates credentials against the identity provider and creates a new session.
    \item \textbf{Output}: Returns a session token if successful.
    \item \textbf{Exceptions}: AuthenticationFailedException if the credentials are invalid.
\end{itemize}

\noindent \texttt{getProfile(userID)}:
\begin{itemize}
    \item \textbf{Output}: Returns the user's profile data.
    \item \textbf{Exceptions}: UserNotFoundException if the user ID does not exist.
\end{itemize}

\noindent \texttt{updateProfile(userID, updates)}:
\begin{itemize}
    \item \textbf{Transition}: Updates the specified user's profile with new information.
    \item \textbf{Output}: Returns \texttt{true} if successful.
    \item \textbf{Exceptions}: InvalidProfileDataException if the updates are invalid.
\end{itemize}

\noindent \texttt{validateSession(sessionToken)}:
\begin{itemize}
    \item \textbf{Output}: Returns \texttt{true} if the session is valid and not expired
    \item \textbf{Exceptions}: InvalidSessionException if the session token is invalid or expired.
\end{itemize}

\noindent \texttt{getRolePermissions(userID)}:
\begin{itemize}
    \item \textbf{Output}: Returns the permissions associated with the user's role.
    \item \textbf{Exceptions}: RoleNotFoundException if the user's role is not found.
\end{itemize}

\subsubsection{Local Functions}
None.

\section{MIS of Expense Submission \& Tracking Module}
\subsection{Module}
Expense Submission \& Tracking

\subsection{Uses}
\begin{itemize}
    \item Database Module
    \item Budget and Funding Management Module
    \item Notifications and Communication Module
    \item Policy \& Compliance Management Module
\end{itemize}

\subsection{Syntax}
\subsubsection{Exported Constants}
\begin{itemize}
    \item MAX\_RECEIPT\_SIZE: The maximum file size (in MB) for uploaded receipts.
    \item ALLOWED\_FILE\_TYPES: List of accepted file formats for receipts ["pdf", "jpg", "png"].
    \item EXPENSE\_CATEGORIES: Predefined expense categories ["conference", "travel", "supplies", "materials"].
\end{itemize}

\subsubsection{Exported Access Programs}
\begin{center}
    \scriptsize
    \begin{tabular}{|p{3cm}|p{4cm}|p{4cm}|p{4cm}|}
        \hline
        \textbf{Name} & \textbf{Input} & \textbf{Output} & \textbf{Exceptions} \\
        \hline
        submitExpense & expenseDetails (\texttt{Object}), attachments (\texttt{Array}) & requestID (\texttt{String}) & InvalidExpenseException \\
        \hline
        uploadAttachment & requestID (\texttt{String}), file (\texttt{File}) & fileID (\texttt{String}) & FileUploadException \\
        \hline
        getExpenseStatus & requestID (\texttt{String}) & status (\texttt{Object}) & RequestNotFoundException \\
        \hline
        updateExpenseDetails & requestID (\texttt{String}), updates (\texttt{Object}) & confirmation (\texttt{Boolean}) & InvalidUpdateException \\
        \hline
        searchExpenses & filters (\texttt{Object}) & expenseList (\texttt{Array}) & InvalidSearchException \\
        \hline
        categorizeExpense & expenseData (\texttt{Object}) & category (\texttt{String}) & CategorizationException \\
        \hline
    \end{tabular}
\end{center}

\subsection{Semantics}
\subsubsection{State Variables}
\begin{itemize}
    \item expenses: A mapping of request IDs to their complete expense details and status.
    \item attachments: A mapping of file IDs to their metadata and storage locations.
    \item categories: A record of expense categorization rules and patterns.
\end{itemize}

\subsubsection{Environment Variables}
\begin{itemize}
    \item File Storage: Stores uploaded receipts and attachments.
    \item OCR Service: Extracts text data from uploaded receipts for processing.
    \item ML Model: Assists in automatic expense categorization
    \item Database: Stores expense records and tracking information
\end{itemize}

\subsubsection{Assumptions}
\begin{itemize}
    \item Receipts are uploaded in a standard format (PDF, JPG, PNG).
    \item Expense categorization is based on predefined rules and ML models.
    \item All monetary values are in Canadian dollars (CAD).
    \item File storage system has sufficient capacity for attachments.
    \item Users have necessary permissions to submit expenses.
    \item Each expense request has at least one receipt attachment.
\end{itemize}

\subsubsection{Access Routine Semantics}
\noindent \texttt{submitExpense(expenseDetails, attachments)}:
\begin{itemize}
    \item \textbf{Transition}: Creates a new expense request with the provided details and attachments.
    \item \textbf{Output}: Returns the unique request ID if successful.
    \item \textbf{Exceptions}: InvalidExpenseException if the expense details are incomplete or invalid.
    \item \textbf{Precondition}: expenseDetails contains required fields (amount, date, purpose, funding source).
    \item \textbf{Postcondition}: Expense request is created with "Submitted" status.
\end{itemize}

\noindent \texttt{uploadAttachment(requestID, file)}:
\begin{itemize}
    \item \textbf{Transition}: Validates and stores the provided file attachment.
    \item \textbf{Output}: Returns a unique fileID if successful.
    \item \textbf{Exceptions}: FileUploadException if the file size or type is invalid.
    \item \textbf{Precondition}: File size and type meet system requirements.
    \item \textbf{Postcondition}: File is stored and linked to the expense request.
\end{itemize}

\noindent \texttt{getExpenseStatus(requestID)}:
\begin{itemize}
    \item \textbf{Output}: Returns the current status and tracking information for the specified request.
    \item \textbf{Exceptions}: RequestNotFoundException if the request ID is invalid.
    \item \textbf{Postcondition}: Status information is retrieved for the request.
\end{itemize}

\noindent \texttt{updateExpenseDetails(requestID, updates)}:
\begin{itemize}
    \item \textbf{Transition}: Applies valid updates to the specified expense request.
    \item \textbf{Output}: Returns \texttt{true} if successful.
    \item \textbf{Exceptions}: InvalidUpdateException if the updates are invalid.
    \item \textbf{Precondition}: Request is in an editable state.
    \item \textbf{Postcondition}: Expense details are updated and audit trail is maintained.
\end{itemize}

\noindent \texttt{searchExpenses(filters)}:
\begin{itemize}
    \item \textbf{Output}: Returns a list of expense requests matching the specified filters.
    \item \textbf{Exceptions}: InvalidSearchException if filter criteria are invalid.
    \item \textbf{Postcondition}: Filtered expense list is returned based on search criteria.
\end{itemize}

\noindent \texttt{categorizeExpense(expenseData)}:
\begin{itemize}
    \item \textbf{Transition}: Analyzes expense details to determine the appropriate category.
    \item \textbf{Output}: Returns the suggested category for the expense or raises CategorizationException.
    \item \textbf{Postcondition}: Expense category is assigned based on ML model and predefined rules.
    \item \textbf{Uses}: ML model and predefined rules for categorization.
    \item \textbf{Updates}: Category confidence score in expense metadata.
\end{itemize}

\subsubsection{Local Functions}
\begin{itemize}
    \item \texttt{validateExpenseData(data: Object) -> Boolean}: Ensures all required fields are present and valid.
    \item \texttt{sanitizeAttachment(file: File) -> File}: Processes uploaded files for security.
    \item \texttt{calculateTotalAmount(items: Array) -> Number}: Computes total expense amount including all items.
\end{itemize}

\section{MIS of Approval Workflow and Review Module}

\subsection{Module}
Approval Workflow and Review

\subsection{Uses}
\begin{itemize}
    \item Budget and Funding Management Module
    \item Notifications and Communication Module
    \item Database Module
\end{itemize}

\subsection{Syntax}
\subsubsection{Exported Constants}
None.

\subsubsection{Exported Access Programs}
\begin{center}
    \scriptsize
    \begin{tabular}{|p{2cm}|p{4cm}|p{4cm}|p{4cm}|} % Add | for vertical lines
        \hline
        \textbf{Name} & \textbf{Input} & \textbf{Output} & \textbf{Exceptions} \\
        \hline
        routeRequest & requestID (\texttt{String}), userRole (\texttt{String}) & status (\texttt{String}) & InvalidRoleException \\
        \hline
        addNote & requestID (\texttt{String}), note (\texttt{String}) & confirmation (\texttt{Boolean}) & RequestNotFoundException \\
        \hline
        updateStatus & requestID (\texttt{String}), newStatus (\texttt{String}) & confirmation (\texttt{Boolean}) & InvalidStatusException \\
        \hline
    \end{tabular}
\end{center}

\subsection{Semantics}
\subsubsection{State Variables}
\begin{itemize}
    \item requests: A mapping of request IDs to their current status and approver roles.
\end{itemize}

\subsubsection{Environment Variables}
\begin{itemize}
    \item Database: Stores information about requests, user roles, and approval workflows.
    \item Notification System: Sends alerts for pending approvals or status updates.
\end{itemize}

\subsubsection{Assumptions}
\begin{itemize}
    \item User roles are pre-validated by the authentication system.
    \item All approvers have access to the system during their review process.
\end{itemize}

\subsubsection{Access Routine Semantics}
\noindent \texttt{routeRequest(requestID, userRole)}:
\begin{itemize}
    \item Output: Returns the current status of the request or raises an InvalidRoleException if the userRole is unauthorized.
\end{itemize}

\noindent \texttt{addNote(requestID, note)}:
\begin{itemize}
    \item Transition: Adds a note to the request specified by \texttt{requestID}.
    \item Output: Returns \texttt{true} if successful or raises RequestNotFoundException if the request ID does not exist.
\end{itemize}

\noindent \texttt{updateStatus(requestID, newStatus)}:
\begin{itemize}
    \item Transition: Updates the status of the request specified by \texttt{requestID}.
    \item Output: Returns \texttt{true} if successful or raises InvalidStatusException if the new status is invalid.
\end{itemize}

\subsubsection{Local Functions}
None.

\section{MIS of Budget and Funding Management Module}

\subsection{Module}
Budget and Funding Management

\subsection{Uses}
\begin{itemize}
    \item Database Module
    \item Notifications and Communication Module
\end{itemize}

\subsection{Syntax}
\subsubsection{Exported Constants}
\begin{itemize}
    \item MAX\_BUDGET: The maximum allowable budget per department.
\end{itemize}

\subsubsection{Exported Access Programs}
\begin{center}
    \scriptsize
    \begin{tabular}{|p{2cm}|p{4cm}|p{4cm}|p{4cm}|} % Add | for vertical lines
        \hline
        \textbf{Name} & \textbf{Input} & \textbf{Output} & \textbf{Exceptions} \\
        \hline
        getBudget & departmentID (\texttt{String}) & budgetDetails (\texttt{Object}) & DepartmentNotFoundException \\
        \hline
        validateFunds & requestAmount (\texttt{Float}), departmentID (\texttt{String}) & status (\texttt{Boolean}) & InsufficientFundsException \\
        \hline
        updateBudget & departmentID (\texttt{String}), amount (\texttt{Float}) & confirmation (\texttt{Boolean}) & BudgetOverflowException \\
        \hline
    \end{tabular}
\end{center}

\subsection{Semantics}
\subsubsection{State Variables}
\begin{itemize}
    \item budgets: A mapping of department IDs to their current budget values.
\end{itemize}

\subsubsection{Environment Variables}
\begin{itemize}
    \item Database: Stores information about department budgets and transactions.
\end{itemize}

\subsubsection{Assumptions}
\begin{itemize}
    \item All transactions are logged for auditing purposes.
    \item Real-time budget data is synchronized with the university financial system.
\end{itemize}

\subsubsection{Access Routine Semantics}
\noindent \texttt{getBudget(departmentID)}:
\begin{itemize}
    \item Output: Returns an object containing the budget details of the specified department or raises DepartmentNotFoundException.
\end{itemize}

\noindent \texttt{validateFunds(requestAmount, departmentID)}:
\begin{itemize}
    \item Output: Returns \texttt{true} if sufficient funds are available, otherwise raises InsufficientFundsException.
\end{itemize}

\noindent \texttt{updateBudget(departmentID, amount)}:
\begin{itemize}
    \item Transition: Updates the budget for the specified department by the specified amount.
    \item Output: Returns \texttt{true} if successful or raises BudgetOverflowException if the update exceeds MAX\_BUDGET.
\end{itemize}

\subsubsection{Local Functions}
None.


\section{MIS of Disbursement \& Payment Processing Module}

\subsection{Module}
Disbursement \& Payment Processing

\subsection{Uses}
\begin{itemize}
    \item Database Module
    \item Budget and Funding Management Module
    \item Notifications and Communication Module
    \item Integration with Other University Systems Module
\end{itemize}

\subsection{Syntax}
\subsubsection{Exported Constants}
\begin{itemize}
    \item PAYMENT\_METHODS: Available payment types ["direct\_deposit", "check", "e\_transfer"]
    \item MAX\_PAYMENT\_AMOUNT: The maximum amount for a single payment.
\end{itemize}

\subsubsection{Exported Access Programs}
\begin{center}
    \scriptsize
    \begin{tabular}{|p{3cm}|p{4cm}|p{4cm}|p{4cm}|} % Add | for vertical lines
        \hline
        \textbf{Name} & \textbf{Input} & \textbf{Output} & \textbf{Exceptions} \\
        \hline
        processPayment & paymentDetails (\texttt{Object}) & paymentID (\texttt{String}) & PaymentProcessingException \\
        \hline
        generatePaymentRecord & paymentID (\texttt{String}) & documentData (\texttt{Object}) & DocumentGenerationException \\
        \hline
        cancelPayment & paymentID (\texttt{String}), reason (\texttt{String}) & confirmation (\texttt{Boolean}) & PaymentCancellationException \\
        \hline
        getPaymentStatus & paymentID (\texttt{String}) & status (\texttt{Object}) & PaymentNotFoundException \\
        \hline
        initiateRefund & paymentID (\texttt{String}), amount (\texttt{Number}) & refundID (\texttt{String}) & RefundProcessingException \\
        \hline
        createBatchPayment & payments (\texttt{Array}) & batchID (\texttt{String}) & BatchProcessingException \\
        \hline
        verifyBankDetails & accountDetails (\texttt{Object}) & isValid (\texttt{Boolean}) & ValidationException \\
        \hline
    \end{tabular}
\end{center}

\subsection{Semantics}
\subsubsection{State Variables}
\begin{itemize}
    \item payments: Record of all payment transactions and their states.
    \item paymentDocuments: Generated payment records and documentation.
    \item bankDetails: Cached bank account verification results.
    \item auditTrail: Log of all payment-related actions.
\end{itemize}

\subsubsection{Environment Variables}
\begin{itemize}
    \item Document Templates: Templates for payment records.
\end{itemize}

\subsubsection{Assumptions}
\begin{itemize}
    \item Payment processing is secure and compliant with financial regulations.
    \item Payment records are generated and stored for auditing purposes.
    \item Bank account details are validated before processing payments.
\end{itemize}

\subsubsection{Access Routine Semantics}
\noindent \texttt{processPayment(paymentDetails)}:
\begin{itemize}
    \item Transition: Initiates payment processing through the specified method.
    \item Output: Returns a unique paymentID on successful processing.
    \item Exceptions: PaymentProcessingException if the payment fails or is invalid.
    \item Precondition: Payment amount is approved and funds are available.
    \item Postcondition: Payment is queued for processing and audit trail is updated.
\end{itemize}

\noindent \texttt{generatePaymentRecord(paymentID)}:
\begin{itemize}
    \item Transition: Creates official payment documentation.
    \item Output: Returns payment record object including all relevant details.
    \item Exceptions: DocumentGenerationException if the record cannot be generated.
    \item Precondition: Payment exists and is valid.
    \item Postcondition: Payment record is generated and stored.
\end{itemize}

\noindent \texttt{cancelPayment(paymentID, reason)}:
\begin{itemize}
    \item Transition: Cancels pending payment and reverses any related transactions.
    \item Output: Returns confirmation of cancellation.
    \item Exceptions: PaymentCancellationException if the payment cannot be cancelled.
    \item Precondition: Payment is in a cancellable state.
    \item Postcondition: Payment is cancelled and notification is sent.
\end{itemize}

\noindent \texttt{getPaymentStatus(paymentID)}:
Includes: Processing stage, timestamps, confirmation numbers
Raises: PaymentNotFoundException if invalid paymentID
\begin{itemize}
    \item Output: Returns the current status of the payment and processing details.
    \item Exceptions: PaymentNotFoundException if the payment ID is invalid.
    \item Postcondition: Payment status is retrieved and displayed.
\end{itemize}

\noindent \texttt{initiateRefund(paymentID, amount)}:
\begin{itemize}
    \item Transition: Creates a refund transaction for the specified payment.
    \item Output: Returns a unique refundID on successful refund initiation.
    \item Exceptions: RefundProcessingException if the refund fails or is invalid.
    \item Precondition: Original payment exists and was successful.
    \item Postcondition: Refund is initiated and audit trail is updated.
\end{itemize}

\noindent \texttt{createBatchPayment(payments)}:
\begin{itemize}
    \item Transition: Processes multiple payments as a single batch.
    \item Output: Returns a unique batchID on successful batch creation.
    \item Exceptions: BatchProcessingException if the batch fails or is invalid.
    \item Precondition: All payments are valid and within processing limits.
    \item Postcondition: Batch payment is queued for processing and audit trail is updated.
\end{itemize}

\noindent \texttt{verifyBankDetails(accountDetails)}:
\begin{itemize}
    \item Output: Returns validation status of provided bank details.
    \item Precondition: Account details follow Canadian banking format.
    \item Postcondition: Validation result is cached for future reference.
\end{itemize}

\subsubsection{Local Functions}
\begin{itemize}
    \item validatePaymentAmount(amount: Number) -> Boolean: Checks payment amount against limits
    \item formatBankingDetails(details: Object) -> Object: Standardizes banking information format
    \item generateAuditEntry(action: String, details: Object) -> void: Creates audit log entry
    \item notifyPaymentStatus(paymentID: String, status: String) -> void: Triggers status notifications
\end{itemize}

\section{MIS of Notifications \& Communication Module}

\subsection{Module}
Notifications \& Communication

\subsection{Uses}
\begin{itemize}
    \item Database Module
\end{itemize}

\subsection{Syntax}
\subsubsection{Exported Constants}

\begin{itemize}
    \item NOTIFICATION\_TYPES: Available notification types ["email", "sms", "dashboard"]
    \item PRIORITY\_LEVELS: Notification priority levels ["low", "medium", "high"]
\end{itemize}


\subsubsection{Exported Access Programs}
\begin{center}
    \scriptsize
    \begin{tabular}{|p{3cm}|p{4cm}|p{4cm}|p{4cm}|} % Add | for vertical lines
        \hline
        \textbf{Name} & \textbf{Input} & \textbf{Output} & \textbf{Exceptions} \\
        \hline
        sendNotification & recipientID (\texttt{String}), message (\texttt{Object}) & notificationID (\texttt{String}) & NotificationFailedException \\
        \hline
        sendBulkNotification & recipients (\texttt{Array}), message (\texttt{Object}) & batchID (\texttt{String}) & BulkSendException \\
        \hline
        getDashboardAlerts & userID (\texttt{String}) & alerts (\texttt{Array}) & UserNotFoundException \\
        \hline
        markAsRead & notificationID (\texttt{String}) & success (\texttt{Boolean}) & InvalidNotificationException \\
        \hline
        getUserPreferences & userID (\texttt{String}) & preferences (\texttt{Object}) & UserNotFoundException \\
        \hline
    \end{tabular}
\end{center}

\subsection{Semantics}
\subsubsection{State Variables}
\begin{itemize}
    \item notifications: Record of all sent notifications.
    \item userPreferences: User notification preferences.
    \item dashboardAlerts: Active alerts for each user.
\end{itemize}

\subsubsection{Environment Variables}
\begin{itemize}
    \item Email Service: For sending email notifications
    \item SMS Gateway: For sending text messages
    \item User Dashboard: For displaying alerts
\end{itemize}

\subsubsection{Assumptions}
\begin{itemize}
    \item Notification delivery is reliable and secure.
    \item User preferences are stored securely and updated in real-time.
    \item Dashboard alerts are displayed in a timely manner.
\end{itemize}

\subsubsection{Access Routine Semantics}
\noindent \texttt{sendNotification(recipientID, message)}:
\begin{itemize}
    \item Transition: Sends a notification to the specified recipient through the preferred channel.
    \item Output: Returns a unique notificationID on successful delivery.
    \item Exceptions: NotificationFailedException if the message cannot be sent.
    \item Precondition: Recipient has valid contact information and preferences.
    \item Postcondition: Notification is sent and logged in the system.
\end{itemize}

\noindent \texttt{sendBulkNotification(recipients, message)}:
\begin{itemize}
    \item Transition: Sends the same notification to multiple recipients.
    \item Output: Returns a unique batchID on successful delivery.
    \item Exceptions: BulkSendException if the message cannot be sent to all recipients.
    \item Precondition: Recipients have valid contact information and preferences.
    \item Postcondition: Notifications are queued for all recipients and logged.
\end{itemize}

\noindent \texttt{getDashboardAlerts(userID)}:
\begin{itemize}
    \item Output: Returns a list of active alerts for the specified user.
    \item Exceptions: UserNotFoundException if the user ID is invalid.
    \item Postcondition: Alerts are retrieved and displayed on the user dashboard.
\end{itemize}

\noindent \texttt{markAsRead(notificationID)}:
\begin{itemize}
    \item Transition: Marks the specified notification as read.
    \item Output: Returns \texttt{true} if successful.
    \item Exceptions: InvalidNotificationException if the notification ID is invalid.
\end{itemize}

\noindent \texttt{getUserPreferences(userID)}:
\begin{itemize}
    \item Output: Returns the user's notification preferences.
    \item Exceptions: UserNotFoundException if the user ID is invalid.
    \item Postcondition: User preferences are retrieved and displayed.
\end{itemize}

\subsubsection{Local Functions}
\begin{itemize}
    \item \texttt{formatMessage(message: Object, template: String) -> String}: Formats notification content
    \item \texttt{validateRecipient(recipientID: String) -> Boolean}: Verifies recipient exists
    \item \texttt{logNotification(details: Object) -> void}: Records notification in audit log
\end{itemize}

\section{MIS of Reporting and Analytics Module}

\subsection{Module}
Reporting and Analytics

\subsection{Uses}
\begin{itemize}
    \item Database Module
\end{itemize}

\subsection{Syntax}
\subsubsection{Exported Constants}
None.

\subsubsection{Exported Access Programs}
\begin{center}
    \scriptsize
    \begin{tabular}{|p{2cm}|p{4cm}|p{4cm}|p{2cm}|} % Add | for vertical lines
        \hline
        \textbf{Name} & \textbf{Input} & \textbf{Output} & \textbf{Exceptions} \\
        \hline
        generateReport & filters (\texttt{Object}) & report (\texttt{PDF/CSV}) & InvalidFilter \newline Exception \\
        \hline
        getUsageStats & timeframe (\texttt{String}) & stats (\texttt{Object}) & None \\
        \hline
    \end{tabular}
\end{center}

\subsection{Semantics}
\subsubsection{State Variables}
\begin{itemize}
    \item reports: Stores past generated reports for caching and quick access.
\end{itemize}

\subsubsection{Environment Variables}
\begin{itemize}
    \item Database: Stores analytics data related to system usage and transactions.
\end{itemize}

\subsubsection{Assumptions}
\begin{itemize}
    \item Data required for analytics is periodically updated and complete.
\end{itemize}

\subsubsection{Access Routine Semantics}
\noindent \texttt{generateReport(filters)}:
\begin{itemize}
    \item Output: Returns a report in PDF or CSV format based on the filters provided, or raises InvalidFilterException if the filters are malformed.
\end{itemize}

\noindent \texttt{getUsageStats(timeframe)}:
\begin{itemize}
    \item Output: Returns an object containing system usage statistics for the specified timeframe.
\end{itemize}

\subsubsection{Local Functions}
None.

\section{MIS of Graphical User Interface (GUI) Module}

\subsection{Module}
Graphical User Interface (GUI)

\subsection{Uses}
\begin{itemize}
    \item Notifications and Communication Module
    \item Database Module
    \item Approval Workflow and Review Module
\end{itemize}

\subsection{Syntax}

\subsubsection{Exported Constants}
None.

\subsubsection{Exported Access Programs}
\begin{center}
    \scriptsize
    \begin{tabular}{|p{3cm}|p{4cm}|p{4cm}|p{4cm}|}
        \hline
        \textbf{Name} & \textbf{Input} & \textbf{Output} & \textbf{Exceptions} \\
        \hline
        \texttt{renderDashboard} & \texttt{userID (String)} & \texttt{dashboardView (HTML/JSON)} & \texttt{UserNotFoundException} \\
        \hline
        \texttt{updateView} & \texttt{viewName (String)} & \texttt{success (Boolean)} & \texttt{ViewNotFoundException} \\
        \hline
        \texttt{handleInput} & \texttt{event (Object)} & \texttt{actionResponse (Boolean)} & \texttt{InputException} \\
        \hline
    \end{tabular}
\end{center}

\subsection{Semantics}

\subsubsection{State Variables}
\begin{itemize}
    \item \textbf{activeView}: The current view being displayed to the user, identified by its name (e.g., "Dashboard", "Expense Submission").
    \item \textbf{userSession}: Details about the currently logged-in user, including preferences and session data.
\end{itemize}

\subsubsection{Environment Variables}
\begin{itemize}
    \item \textbf{Browser Window}: Displays the GUI and captures user inputs.
    \item \textbf{Server API}: Fetches data to dynamically update the GUI based on user interactions.
\end{itemize}

\subsubsection{Assumptions}
\begin{itemize}
    \item The browser supports modern web standards (HTML5, CSS3, JavaScript).
    \item Users have active and authenticated sessions before interacting with the GUI.
\end{itemize}

\subsubsection{Access Routine Semantics}

\noindent \texttt{renderDashboard(userID)}:
\begin{itemize}
    \item \textbf{Output}: Renders the dashboard for the given \texttt{userID} or raises a \texttt{UserNotFoundException} if the user is not valid.
\end{itemize}

\noindent \texttt{updateView(viewName)}:
\begin{itemize}
    \item \textbf{Output}: Updates the active view to the one specified by \texttt{viewName} and returns \texttt{true} if successful, or raises a \texttt{ViewNotFoundException}.
\end{itemize}

\noindent \texttt{handleInput(event)}:
\begin{itemize}
    \item \textbf{Transition}: Processes user interactions, such as button clicks or form submissions, and triggers corresponding actions.
    \item \textbf{Output}: Returns \texttt{true} if the input is handled successfully or raises an \texttt{InputException}.
\end{itemize}

\subsubsection{Local Functions}
None.

\section{MIS of Policy \& Compliance Management Module}

\subsection{Module}
Policy \& Compliance Management

\subsection{Uses}
\begin{itemize}
    \item Database Module
    \item Notifications and Communication Module
\end{itemize}

\subsection{Syntax}

\subsubsection{Exported Constants}
\begin{itemize}
    \item \texttt{MAX\_REIMBURSEMENT}: The maximum allowable reimbursement amount for a single request.
    \item \texttt{TRAVEL\_APPROVAL\_LIMIT}: The threshold above which travel expenses require prior approval.
\end{itemize}

\subsubsection{Exported Access Programs}
\begin{center}
    \scriptsize
    \begin{tabular}{|p{3cm}|p{4cm}|p{4cm}|p{4cm}|}
        \hline
        \textbf{Name} & \textbf{Input} & \textbf{Output} & \textbf{Exceptions} \\
        \hline
        \texttt{validateRequest} & \texttt{requestDetails (Object)} & \texttt{validationStatus (Boolean)} & \texttt{PolicyViolationException} \\
        \hline
        \texttt{getPolicyRules} & \texttt{policyType (String)} & \texttt{rules (Object)} & \texttt{PolicyNotFoundException} \\
        \hline
        \texttt{logComplianceCheck} & \texttt{requestID (String), result (Boolean)} & \texttt{confirmation (Boolean)} & \texttt{LogFailureException} \\
        \hline
    \end{tabular}
\end{center}

\subsection{Semantics}

\subsubsection{State Variables}
\begin{itemize}
    \item \textbf{policies}: A mapping of policy types to their respective rules and thresholds.
    \item \textbf{complianceLogs}: A record of compliance checks performed on submitted requests.
\end{itemize}

\subsubsection{Environment Variables}
\begin{itemize}
    \item \textbf{Policy Database}: Stores policy definitions and thresholds.
    \item \textbf{Audit System}: Logs compliance checks and violations for review.
\end{itemize}

\subsubsection{Assumptions}
\begin{itemize}
    \item Policy rules are periodically updated to align with organizational regulations.
    \item Compliance checks are triggered automatically during the request submission process.
\end{itemize}

\subsubsection{Access Routine Semantics}

\noindent \texttt{validateRequest(requestDetails)}:
\begin{itemize}
    \item \textbf{Output}: Returns \texttt{true} if the request complies with all applicable policies or raises a \texttt{PolicyViolationException} if a rule is violated.
\end{itemize}

\noindent \texttt{getPolicyRules(policyType)}:
\begin{itemize}
    \item \textbf{Output}: Returns the rules for the specified \texttt{policyType} or raises a \texttt{PolicyNotFoundException} if the policy type is invalid.
\end{itemize}

\noindent \texttt{logComplianceCheck(requestID, result)}:
\begin{itemize}
    \item \textbf{Transition}: Records the result of a compliance check for the given \texttt{requestID}.
    \item \textbf{Output}: Returns \texttt{true} if the log is updated successfully or raises a \texttt{LogFailureException}.
\end{itemize}

\subsubsection{Local Functions}
None.

\section{MIS of Integration with Other University Systems Module}

\subsection{Module}
Integration with Other University Systems

\subsection{Uses}
\begin{itemize}
    \item Database Module
    \item External APIs (University Systems)
\end{itemize}

\subsection{Syntax}

\subsubsection{Exported Constants}
\begin{itemize}
    \item \texttt{STUDENT\_INFO\_API\_URL}: The endpoint for accessing the university’s Student Information System (SIS).
    \item \texttt{FINANCE\_SYSTEM\_API\_URL}: The endpoint for accessing the university’s financial system.
\end{itemize}

\subsubsection{Exported Access Programs}
\begin{center}
    \scriptsize
    \begin{tabular}{|p{3cm}|p{4cm}|p{4cm}|p{4cm}|}
        \hline
        \textbf{Name} & \textbf{Input} & \textbf{Output} & \textbf{Exceptions} \\
        \hline
        \texttt{fetchStudentInfo} & \texttt{studentID (String)} & \texttt{studentDetails (Object)} & \texttt{StudentNotFoundException} \\
        \hline
        \texttt{syncFinancialData} & \texttt{departmentID (String)} & \texttt{syncStatus (Boolean)} & \texttt{FinanceSyncFailureException} \\
        \hline
        \texttt{verifyEnrollment} & \texttt{studentID (String)} & \texttt{enrollmentStatus (Boolean)} & \texttt{EnrollmentVerification \newline Exception} \\
        \hline
    \end{tabular}
\end{center}

\subsection{Semantics}

\subsubsection{State Variables}
\begin{itemize}
    \item \textbf{universityData}: A cache of information fetched from external university systems for performance optimization.
    \item \textbf{syncLogs}: A record of synchronization activities between the system and external APIs.
\end{itemize}

\subsubsection{Environment Variables}
\begin{itemize}
    \item \textbf{University SIS}: Provides student details such as enrollment status, department, and contact information.
    \item \textbf{University Finance System}: Manages departmental budgets, account balances, and payment records.
\end{itemize}

\subsubsection{Assumptions}
\begin{itemize}
    \item API endpoints for university systems are reliable and adhere to predefined contracts.
    \item Authentication credentials for accessing university systems are securely stored and updated as needed.
\end{itemize}

\subsubsection{Access Routine Semantics}

\noindent \texttt{fetchStudentInfo(studentID)}:
\begin{itemize}
    \item \textbf{Output}: Returns the details of the student identified by \texttt{studentID}, or raises \texttt{StudentNotFoundException} if no matching record is found.
\end{itemize}

\noindent \texttt{syncFinancialData(departmentID)}:
\begin{itemize}
    \item \textbf{Transition}: Synchronizes financial data for the given \texttt{departmentID} with the university’s finance system.
    \item \textbf{Output}: Returns \texttt{true} if the synchronization is successful or raises \texttt{FinanceSyncFailureException} if an error occurs.
\end{itemize}

\noindent \texttt{verifyEnrollment(studentID)}:
\begin{itemize}
    \item \textbf{Output}: Returns \texttt{true} if the student is enrolled, or raises \texttt{EnrollmentVerificationException} if verification fails.
\end{itemize}

\subsubsection{Local Functions}
None.

\section{MIS of Administrator and Configuration Panel Module}

\subsection{Module}
Administrator and Configuration Panel

\subsection{Uses}
\begin{itemize}
    \item Database Module
    \item Notifications and Communication Module
    \item Integration with Other University Systems Module
\end{itemize}

\subsection{Syntax}

\subsubsection{Exported Constants}
\begin{itemize}
    \item \texttt{DEFAULT\_ROLE\_PERMISSIONS}: A mapping of roles to their default access permissions.
    \item \texttt{MAX\_NOTIFICATION\_TEMPLATES}: The maximum number of customizable notification templates allowed.
\end{itemize}

\subsubsection{Exported Access Programs}
\begin{center}
    \scriptsize
    \begin{tabular}{|p{3cm}|p{4cm}|p{4cm}|p{4cm}|}
        \hline
        \textbf{Name} & \textbf{Input} & \textbf{Output} & \textbf{Exceptions} \\
        \hline
        \texttt{addRole} & \texttt{roleName (String), permissions (List)} & \texttt{confirmation (Boolean)} & \texttt{RoleAlreadyExistsException} \\
        \hline
        \texttt{updateApprovalChain} & \texttt{chainConfig (Object)} & \texttt{confirmation (Boolean)} & \texttt{InvalidChainConfigException} \\
        \hline
        \texttt{editNotification \newline Template} & \texttt{templateID (String), newTemplate (String)} & \texttt{confirmation (Boolean)} & \texttt{TemplateNotFoundException} \\
        \hline
        \texttt{viewLogs} & \texttt{timeframe (String)} & \texttt{logData (Object)} & None \\
        \hline
    \end{tabular}
\end{center}

\subsection{Semantics}

\subsubsection{State Variables}
\begin{itemize}
    \item \textbf{roles}: A mapping of roles to their permissions and associated users.
    \item \textbf{notificationTemplates}: A collection of templates for system notifications.
    \item \textbf{auditLogs}: A collection of system logs for administrative actions.
\end{itemize}

\subsubsection{Environment Variables}
\begin{itemize}
    \item \textbf{Database}: Stores configuration settings, logs, and role assignments.
    \item \textbf{Notification System}: Delivers updates to users based on configured templates.
\end{itemize}

\subsubsection{Assumptions}
\begin{itemize}
    \item Role and permission updates propagate immediately to all system components.
    \item Audit logs are retained and accessible for the configured retention period.
\end{itemize}

\subsubsection{Access Routine Semantics}

\noindent \texttt{addRole(roleName, permissions)}:
\begin{itemize}
    \item \textbf{Transition}: Adds a new role with the specified permissions to the system.
    \item \textbf{Output}: Returns \texttt{true} if the role is added successfully or raises \texttt{RoleAlreadyExistsException}.
\end{itemize}

\noindent \texttt{updateApprovalChain(chainConfig)}:
\begin{itemize}
    \item \textbf{Transition}: Updates the approval chain configuration based on the provided settings.
    \item \textbf{Output}: Returns \texttt{true} if the update is successful or raises \texttt{InvalidChainConfigException}.
\end{itemize}

\noindent \texttt{editNotificationTemplate(templateID, newTemplate)}:
\begin{itemize}
    \item \textbf{Transition}: Updates the specified notification template with the new content.
    \item \textbf{Output}: Returns \texttt{true} if the template is updated successfully or raises \texttt{TemplateNotFoundException}.
\end{itemize}

\noindent \texttt{viewLogs(timeframe)}:
\begin{itemize}
    \item \textbf{Output}: Returns log data for the specified timeframe.
\end{itemize}

\subsubsection{Local Functions}
None.


\newpage

\bibliographystyle {plainnat}
\bibliography {../../../refs/References}

\newpage

\section{Appendix} \label{Appendix}

\wss{Extra information if required. Currently, none.}

\newpage{}

\section{Appendix} \label{Appendix}

\wss{Extra information if required. Currently, none.}

\newpage{}

\section*{Appendix --- Reflection}

The information in this section will be used to evaluate the team members on the
graduate attribute of Problem Analysis and Design.

The purpose of reflection questions is to give you a chance to assess your own
learning and that of your group as a whole, and to find ways to improve in the
future. Reflection is an important part of the learning process.  Reflection is
also an essential component of a successful software development process.  

Reflections are most interesting and useful when they're honest, even if the
stories they tell are imperfect. You will be marked based on your depth of
thought and analysis, and not based on the content of the reflections
themselves. Thus, for full marks we encourage you to answer openly and honestly
and to avoid simply writing ``what you think the evaluator wants to hear.''

Please answer the following questions.  Some questions can be answered on the
team level, but where appropriate, each team member should write their own
response:


\begin{enumerate}
  \item \textbf{What went well while writing this deliverable?}  
  \newline
  Rachid: Collaborating as a team was smooth, and we were able to divide the modules effectively, which streamlined the writing process. \\
  Sufyan: Our TA meeting going over the modules and discussing it helped us understand the requirements better. \\
  Housam: Identifying each module’s secrets and responsibilities early on helped maintain clarity and reduced redundancy in our design process. \\
  Taaha: Understanding a double checking our modules over with the TA helped tremendously, ensuring we had the correct approach \\
  Omar: The TA meeting went over a lot of our confusion that we had, in addition the extra alloted time allowed us to flesh out this deliverable a bit more. \\

  \item \textbf{What pain points did you experience during this deliverable, and how did you resolve them?}  
  \newline
  Rachid: One challenge was ensuring consistency across modules. Regular team reviews and communication helped resolve any inconsistencies. \\
  Sufyan: Coordinating between the MG and MIS was a bit challenging, since we broke up who does each PDF. \\
  Housam: Aligning anticipated changes with module-level details was challenging but resolved through a thorough review of the SRS and MG. \\
  Taaha: Some of the questions were somewhat hard to understand what to do such as the ones requiring to create diagrams. \\
  Omar: A pain point I had was ensuring consistency between documents and ensuring all ideas align with eachother. \\

  \item \textbf{Which of your design decisions stemmed from speaking to your client(s) or a proxy (e.g., your peers, stakeholders, potential users)? For those that were not, why, and where did they come from?}  
  \newline
  GROUP: Many decisions, like integrating notifications, came directly from stakeholder feedback. Others, like modular decomposition, were based on best practices and team experience.

  \item \textbf{While creating the design doc, what parts of your other documents (e.g., requirements, hazard analysis, etc.), if any, needed to be changed, and why?}  
  \newline
  GROUP: The requirements document was updated to better align with the final design, specifically in the Reporting module to include export formats.

  \item \textbf{What are the limitations of your solution? Put another way, given unlimited resources, what could you do to make the project better? (LO\_ProbSolutions)}  
  \newline
  Rachid: With unlimited resources, we could enhance system scalability and user interface design, making it more robust and user-friendly. \\
  Sufyan: We could also improve the integration with external systems such as banking APIs but we are limited by Open Banking not being available. \\
  Housam: With unlimited resources, we could incorporate advanced machine learning algorithms for OCR-based receipt processing to enhance accuracy and reduce manual intervention. \\
  Taaha: Provided unlimited resources, we could integrate better support with online banking organizations to allow for easier transactions. \\
  Omar: We could increase the use cases of the application, we could allow users more options based off what the MES requires or we could refine our solution to be more user friendly. We could have additionaly made a mobile app as well as a web app. \\

  \item \textbf{Give a brief overview of other design solutions you considered. What are the benefits and tradeoffs of those other designs compared with the chosen design? From all the potential options, why did you select the documented design? (LO\_Explores)}  
  \newline
  Rachid: We considered alternative approaches for managing the workflow logic, such as using external libraries, but chose an in-house solution for simplicity and better control over implementation. \\
  Sufyan: We also considered using a third-party notification service, but opted for an in-house system to maintain data privacy and security. \\
  Housam: We debated between centralized compliance validation versus distributed validation across modules. While centralized validation offered simplicity, we chose the distributed approach to align better with modular decomposition and scalability goals.\\
  Taaha: We considered using external libraries to manage numerous functionality such as workflow process and notifications. \\
  Omar: A design solution I considered was making a mobile app instead of a web app, this in my opinion would have made the notifcaion process easier and more friendly to users. The tradeoff being the MES did not want this initially so we would not be making an appealing app to our main stakeholder. We decided to select a web app since that would appeal to all stakeholders and we are still able to notify users via email. 
\end{enumerate}


\end{document}