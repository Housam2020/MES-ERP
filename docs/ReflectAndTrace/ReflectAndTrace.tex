\documentclass{article}

\usepackage{tabularx}
\usepackage{booktabs}
\title{Reflection and Traceability Report on \progname}

\author{\authname}

\date{}

%% Comments

\usepackage{color}

\newif\ifcomments\commentstrue %displays comments
%\newif\ifcomments\commentsfalse %so that comments do not display

\ifcomments
\newcommand{\authornote}[3]{\textcolor{#1}{[#3 ---#2]}}
\newcommand{\todo}[1]{\textcolor{red}{[TODO: #1]}}
\else
\newcommand{\authornote}[3]{}
\newcommand{\todo}[1]{}
\fi

\newcommand{\wss}[1]{\authornote{blue}{SS}{#1}} 
\newcommand{\plt}[1]{\authornote{magenta}{TPLT}{#1}} %For explanation of the template
\newcommand{\an}[1]{\authornote{cyan}{Author}{#1}}

%% Common Parts

\newcommand{\progname}{ProgName} % PUT YOUR PROGRAM NAME HERE
\newcommand{\authname}{Team \#, Team Name
\\ Student 1 name
\\ Student 2 name
\\ Student 3 name
\\ Student 4 name} % AUTHOR NAMES                  

\usepackage{hyperref}
    \hypersetup{colorlinks=true, linkcolor=blue, citecolor=blue, filecolor=blue,
                urlcolor=blue, unicode=false}
    \urlstyle{same}
                                


\begin{document}

\maketitle

\plt{Reflection is an important component of getting the full benefits from a
learning experience.  Besides the intrinsic benefits of reflection, this
document will be used to help the TAs grade how well your team responded to
feedback.  Therefore, traceability between Revision 0 and Revision 1 is and
important part of the reflection exercise.  In addition, several CEAB (Canadian
Engineering Accreditation Board) Learning Outcomes (LOs) will be assessed based
on your reflections.}

\section{Changes in Response to Feedback}

\subsection{Changes in Response to Usability Testing}

\subsection{SRS and Hazard Analysis}

\subsubsection{Changes Made to the SRS}


\subsubsection{Changes Made to the Hazard Analysis}

The document has been thoroughly revised to address feedback from a rubric. Here are the key changes:

\subsubsection{Document Structure Improvements}
\begin{itemize}
  \item Added Table of Contents, List of Tables, and List of Figures
  \item Improved document organization with appropriate page breaks
  \item Updated revision history table with current changes
  \item Added proper labels to Safety/Security Requirements (SSR-1, SSR-2, SSR-3)
\end{itemize}

\subsubsection{Content Enhancements}
\begin{enumerate}
  \item \textbf{Introduction \& Scope}:
  \begin{itemize}
    \item Refined hazard definitions specifically for MES-ERP financial operations
    \item Expanded scope to clearly include all system components (frontend, backend, database, auth)
    \item Added clearer document roadmap
  \end{itemize}

  \item \textbf{System Boundaries \& Components}:
  \begin{itemize}
    \item Added technology specifics (Supabase/PostgreSQL, Next.js/React)
    \item Enhanced hazard descriptions with concrete examples
    \item Added new hazards (race conditions, XSS vulnerabilities, session hijacking)
  \end{itemize}

  \item \textbf{Critical Assumptions}:
  \begin{itemize}
    \item Clarified existing assumptions
    \item Added a fifth assumption about security of underlying infrastructure
  \end{itemize}

  \item \textbf{FMEA Implementation}:
  \begin{itemize}
    \item Created a complete FMEA table with severity, occurrence, and detection ratings
    \item Calculated Risk Priority Numbers (RPN)
    \item Added specific mitigation strategies for each failure mode
    \item Cross-referenced Safety/Security Requirements
  \end{itemize}

  \item \textbf{Safety Requirements}:
  \begin{itemize}
    \item Added formal labels and improved descriptions
    \item Enhanced rationales for implementation
  \end{itemize}

  \item \textbf{Roadmap}:
  \begin{itemize}
    \item Improved immediate implementation items with technical specifics
    \item Added cross-references to SSRs
    \item Added new future implementation item (Security Penetration Testing)
  \end{itemize}
\end{enumerate}

\vspace{1em}
The revisions make the document more precise, technically detailed, and better aligned with software engineering best practices for hazard analysis.


All changes were made to address the rubric feedback, with special attention to improving the technical specificity of hazard descriptions, creating a proper FMEA analysis, and ensuring consistent cross-referencing between identified hazards and safety requirements. The revisions significantly enhanced the document's precision and alignment with software engineering best practices for hazard analysis of financial systems.

\subsection{Design and Design Documentation}

\subsection{VnV Plan and Report}

\section{Challenge Level and Extras}

\subsection{Challenge Level}

\plt{State the challenge level (advanced, general, basic) for your project.  Your challenge level should exactly match what is included in your problem statement.  This should be the challenge level agreed on between you and the course instructor.}

\subsection{Extras}
\begin{enumerate}
    \item \textbf{Usability Testing} \\
    Conduct formal usability testing with a variety of stakeholders (students, financial staff) to refine the interface and ensure a smooth user experience.
    \item \textbf{User Documentation} \\
    Create comprehensive user documentation in the form of written documentation and video tutorials that guides end-users through every step of submitting expenses, reviewing budgets, and navigating the platform.
\end{enumerate}

\plt{Summarize the extras (if any) that were tackled by this project.  Extras
can include usability testing, code walkthroughs, user documentation, formal
proof, GenderMag personas, Design Thinking, etc.  Extras should have already
been approved by the course instructor as included in your problem statement.}

\section{Design Iteration (LO11 (PrototypeIterate))}

\plt{Explain how you arrived at your final design and implementation.  How did
the design evolve from the first version to the final version?} 

\plt{Don't just say what you changed, say why you changed it.  The needs of the
client should be part of the explanation.  For example, if you made changes in
response to usability testing, explain what the testing found and what changes
it led to.}

\section{Design Decisions (LO12)}

\plt{Reflect and justify your design decisions.  How did limitations,
 assumptions, and constraints influence your decisions?  Discuss each of these
 separately.}

\section{Economic Considerations (LO23)}

\plt{Is there a market for your product? What would be involved in marketing your 
product? What is your estimate of the cost to produce a version that you could 
sell?  What would you charge for your product?  How many units would you have to 
sell to make money? If your product isn't something that would be sold, like an 
open source project, how would you go about attracting users?  How many potential 
users currently exist?}

\section{Reflection on Project Management (LO24)}

\plt{This question focuses on processes and tools used for project management.}

\subsection{How Does Your Project Management Compare to Your Development Plan}

\plt{Did you follow your Development plan, with respect to the team meeting plan, 
team communication plan, team member roles and workflow plan.  Did you use the 
technology you planned on using?}

\subsection{What Went Well?}

\plt{What went well for your project management in terms of processes and 
technology?}

\subsection{What Went Wrong?}

\plt{What went wrong in terms of processes and technology?}

\subsection{What Would you Do Differently Next Time?}

\plt{What will you do differently for your next project?}

\section{Reflection on Capstone}

\plt{This question focuses on what you learned during the course of the capstone project.}

\subsection{Which Courses Were Relevant}

\plt{Which of the courses you have taken were relevant for the capstone project?}

\subsection{Knowledge/Skills Outside of Courses}

\plt{What skills/knowledge did you need to acquire for your capstone project
that was outside of the courses you took?}

\end{document}