\documentclass{article}

\usepackage{booktabs}
\usepackage{tabularx}

\title{Development Plan\\\progname}

\author{\authname}

\date{}

%% Comments

\usepackage{color}

\newif\ifcomments\commentstrue %displays comments
%\newif\ifcomments\commentsfalse %so that comments do not display

\ifcomments
\newcommand{\authornote}[3]{\textcolor{#1}{[#3 ---#2]}}
\newcommand{\todo}[1]{\textcolor{red}{[TODO: #1]}}
\else
\newcommand{\authornote}[3]{}
\newcommand{\todo}[1]{}
\fi

\newcommand{\wss}[1]{\authornote{blue}{SS}{#1}} 
\newcommand{\plt}[1]{\authornote{magenta}{TPLT}{#1}} %For explanation of the template
\newcommand{\an}[1]{\authornote{cyan}{Author}{#1}}

%% Common Parts

\newcommand{\progname}{ProgName} % PUT YOUR PROGRAM NAME HERE
\newcommand{\authname}{Team \#, Team Name
\\ Student 1 name
\\ Student 2 name
\\ Student 3 name
\\ Student 4 name} % AUTHOR NAMES                  

\usepackage{hyperref}
    \hypersetup{colorlinks=true, linkcolor=blue, citecolor=blue, filecolor=blue,
                urlcolor=blue, unicode=false}
    \urlstyle{same}
                                


\begin{document}

\maketitle

\begin{table}[hp]
\caption{Revision History} \label{TblRevisionHistory}
\begin{tabularx}{\textwidth}{llX}
\toprule
\textbf{Date} & \textbf{Developer(s)} & \textbf{Change}\\
\midrule
Sept 24th, 2024 & Housam \& Taaha & Completed the confidential information, IP info, Copyright, Team plan and Workflow plan \\
Sept 24th, 2024 & Omar, Rachid & Changes to part 8, 9, Appendix\\
\bottomrule
\end{tabularx}
\end{table}

\newpage{}

\section*{Introduction}

This Development Plan outlines the key aspects of our project, the MES (McMaster Engineering Society) Custom Financial Expense Reporting Platform. The document is structured to provide an overview of the project's goals, workflow, and expected technologies. It includes sections on confidential information handling, intellectual property considerations, copyright licensing, team roles, and communication plans. Additionally, the plan covers our coding standards, Git workflow, issue management, and CI/CD implementation. 

The document also provides a project decomposition and scheduling plan, along with a proof of concept demonstration plan to address potential risks. Finally, the appendices include reflections on the importance of a development plan, the advantages and disadvantages of CI/CD, and our team charter, which outlines our external goals, attendance expectations, accountability, and decision-making processes.

\section{Confidential Information}
This project deals with a large amount of confidental information related to the financials of all the MES clubs and organizations. Such details must maintain confidentiality and can only be shared within MES and student groups. All group member will sign an NDA provided by the MES before gaining access to said information. Confidential information includes, but is not limited to, the following:

\begin{itemize}
  \item \textbf{Expense Receipts}: Detailed records of all expenses incurred by the MES clubs and organizations, including receipts for purchases, travel expenses, and other financial transactions.
  \item \textbf{Budget Reports}: Comprehensive reports outlining the budget allocations, expenditures, and financial planning for various MES activities and events.
  \item \textbf{Financial Statements}: Detailed financial statements that provide insights into the financial health of the MES, including balance sheets, income statements, and cash flow statements.
  \item \textbf{Membership Fees and Dues}: Information on membership fees collected from MES members, including payment records, dues schedules, and any outstanding balances.
  \item \textbf{Event Financials}: Detailed financial records for events organized by MES, including income from ticket sales, sponsorships, and expenses related to event logistics and execution.
  \item \textbf{Vendor Contracts}: Agreements with vendors providing goods and services to MES, including payment terms, service agreements, and financial obligations.
\end{itemize}


\section{IP to Protect}
No IP to protect. All current software projects by the MES are publically available at \href{https://github.com/McMaster-Engineering-Society}{https://github.com/McMaster-Engineering-Society}


\section{Copyright License}
This project will follow a proprietary licensing agreement, which can be found in the LICENSE file.

\section{Team Meeting Plan}

We will hold team meetings twice a week, either online or in person, based on availability and need. Advisor meetings will take place monthly through Teams unless urgent issues arise. In such cases, meetings may be held in person, depending on the severity of the matter and in consultation with the advisor.

\subsection{Meeting Structure}

\subsubsection{Team Meetings}
Each team member will share updates on completed tasks and current work. During this time, others can ask questions or provide feedback. After updates, we will discuss potential next steps and address any concerns as a group. Finally, each member will have an opportunity to raise questions or seek support on any challenges they are facing.

\subsubsection{Advisor Meetings}
We will begin by providing an overview of the project’s progress and potential next steps. The advisor will then have the opportunity to raise any concerns or provide guidance. We will conclude by discussing any questions or clarifications needed regarding the project.

\subsection{Meeting Roles}
The role of meeting chair will rotate among team members. Prior to advisor meetings, the team will collaboratively decide on the agenda and compile a list of questions and concerns to be addressed.

\section{Team Communication Plan}
In terms of team communication for this project, we will use discord as the main messaging platform. Discord will be used for messaging between the group members, to plan meetings, talk about issues, share resources and concepts. In addition, all group members will stay up to date on the GitHub issues to track any issue with their commits and be required to address (fix or simply respond) all their issues. Additionally, any merge conflicts or issues pushing or pulling code will be notified to team members via discord or if urgent through phone call.

\section{Team Member Roles}

We have assigned the following roles to team members, detailing their responsibilities.
Every month we will meet back to adjust the roles of each person as needed according to what people want to do and what their strengths are. In addition, group members can request a vote to switch to another role at any time.

\begin{itemize}
    \item \textbf{Meeting Chair: Taaha Atif}
    \begin{itemize}
        \item Organizes and leads team meetings.
        \item Prepares the meeting agenda and ensures all topics are covered.
        \item Facilitates discussions and ensures that meetings run smoothly and on time.
    \end{itemize}

    \item \textbf{Notetaker: Rachid Khneisser}
    \begin{itemize}
        \item Takes detailed notes during meetings.
        \item Distributes meeting minutes to all team members after each meeting.
        \item Ensures that action items and decisions are clearly documented.
    \end{itemize}

    \item \textbf{Reviewer: Sufyan Motala}
    \begin{itemize}
        \item Reviews all project documents and deliverables for accuracy and completeness.
        \item Provides feedback and suggestions for improvements.
        \item Ensures that all work meets the project’s quality standards.
    \end{itemize}

    \item \textbf{Notetaker/Helper: Omar Muhammad}
    \begin{itemize}
        \item Reviews all project documents and deliverables for accuracy and completeness (with Rachid).
        \item Talks to anyone else in the group to see if they need assistance for any work.
    \end{itemize}

    \item \textbf{Leader: Housam Alamour}
    \begin{itemize}
        \item Oversees the overall progress of the project.
        \item Coordinates tasks and ensures that deadlines are met.
        \item Acts as the main point of contact between the team and the advisor.
    \end{itemize}
\end{itemize}


\section{Workflow Plan}

\section{Workflow Plan}

\subsection{Git Usage}
Git will be used to pull/push code (share code) between the group. The following practices will be followed:
\begin{itemize}
  \item \textbf{Branches}: Each feature or bug fix will be developed in its own branch. Branch names will follow a consistent naming convention, such as \texttt{feature/feature-name} or \texttt{bugfix/issue-number}.
  \item \textbf{Pull Requests (PRs)}: Pull requests will be used to merge changes from feature branches into the main branch. PRs will require at least one peer review before being merged to ensure code quality and adherence to project standards.
  \item \textbf{Commit Messages}: Commit messages will be descriptive and follow a consistent format, detailing what changes have been made, what files are affected, and why these changes are necessary.
  \item \textbf{Tags}: Tags will be used to mark significant milestones and releases in the project. Tags will follow a semantic versioning format, such as \texttt{v1.0.0}.
  \item \textbf{Issue Tracking}: Issues will be tracked using a project management tool like GitHub Issues. Issues will be categorized based on their type (e.g., bug, feature request) and priority (e.g., high, medium, low).
  \item \textbf{Project Management Tools}: We will use GitHub Projects to organize and manage tasks, track progress, and ensure that all team members are aligned with the project goals.
\end{itemize}

\subsection{Issue Management}
Issues will be managed to ensure smooth progress and timely resolution of problems:
\begin{itemize}
    \item \textbf{Issue Templates}: Templates will be used for creating new issues to ensure all necessary information is provided. Templates will include sections for description, steps to reproduce, expected behavior, and actual behavior.
    \item \textbf{Issue Classification}: Issues will be classified based on their type (e.g., bug, feature request, enhancement) and priority (e.g., high, medium, low).
    \item \textbf{Assignment}: Issues will be assigned to team members based on their expertise and availability. Team members can volunteer for issues or be assigned by the project manager.
\end{itemize}

\subsection{CI/CD Implementation}
Continuous Integration and Continuous Deployment (CI/CD) will be implemented to maintain code quality and streamline the development process:
\begin{itemize}
    \item \textbf{Auto-formatting}: On commit, code will be auto-formatted to adhere to the project's coding standards.
    \item \textbf{Linting}: Linting will be performed via CI/CD to catch syntax and style issues early.
    \item \textbf{Static Analysis}: Static analysis will be conducted on committed code to identify potential issues and improve code quality.
    \item \textbf{Automated Testing}: Automated tests will be run on each commit to ensure that new changes do not break existing functionality.
\end{itemize}

\section{POC Demo Plan}

\subsection{Proof of Concept (POC) Demonstration}
The Proof of Concept (POC) demonstration will be a critical part of our project, showcasing the feasibility and functionality of our proposed solution. The following elements will be included in the POC demo plan:

\begin{itemize}
    \item \textbf{Objective}: Clearly define the objectives of the POC, outlining what we aim to demonstrate and achieve.
    \item \textbf{Scope}: Specify the scope of the POC, detailing the features and functionalities that will be included in the demonstration.
    \item \textbf{Risks and Mitigation}: Identify potential risks associated with the POC and provide detailed mitigation strategies. This includes considering the consequences of any risks that may be difficult to mitigate.
    \item \textbf{Implementation Plan}: Outline the steps and timeline for implementing the POC, ensuring that all necessary components are developed and tested.
    \item \textbf{Evaluation Criteria}: Establish clear criteria for evaluating the success of the POC, including performance metrics and user feedback.
\end{itemize}

The POC demo plan will provide a good overview of our steps taken to complete this project, making sure that all aspects are thoroughly considered and addressed. This will help us identify and mitigate risks early, leading to the success of the project.

Our project will be scheduled as such: \\

\begin{tabbing}
  \hspace{8cm} \= \hspace{3cm} \= \kill
  Requirements Document Revision 0 \> October 9 \\
  Hazard Analysis 0 \> October 23 \\
  V\&V Plan Revision 0 \> November 1 \\
  Proof of Concept Demonstration \> November 11--22 \\
  Design Document Revision 0 \> January 15 \\
  Revision 0 Demonstration \> February 3--February 14 \\
  V\&V Report Revision 0 \> March 7 \\
  Final Demonstration (Revision 1) \> March 24--March 30 \\
  EXPO Demonstration \> April TBD \\
\end{tabbing}

\section{Proof of Concept Demonstration Plan}

\vspace{0.5cm}

The Proof of Concept (POC) demonstration will be a critical part of our project, showcasing the feasibility and functionality of our proposed solution.

The main risk for our project is handling unexpected inputs from the receipts we are given. Since this is the primary purpose of the project, failure to handle these inputs 100\% of the time would be a significant issue. To demonstrate success at the POC demo, we will bring a variety of receipts and showcase how our project operates. We will show that we can handle normal, boundary, and unusual inputs with appropriate responses. Successfully demonstrating this capability will indicate that we have effectively mitigated the risk of being unable to handle some inputs.

\subsection{Link to GitHub Project:} \href{https://github.com/Housam2020/MES-ERP}{https://github.com/Housam2020/MES-ERP} \\
\section{Expected Technology}

Please note that any specific technologies referenced in this section are merely suggestions and exist mainly as an illustrative example for what the potential product used for each section could be and what it's capabilities should be. These options are not fixed and are subject to be changed as the project continues to be developed.

\begin{itemize}
  \item \textbf{Specific programming language:}
  \begin{itemize}
      \item \textbf{TypeScript}: For this project, we will be using Typescript as it has been set out as a constraint by the supervisor for consistency reasons as many of the previous MES tools use this language. However, its benefits are that it provides static typing, which helps catch errors early and improves code quality. Additionally, it integrates well with Next.js, the other requirement for this project.
  \end{itemize}
  
  \item \textbf{Specific libraries:}
  
  \begin{itemize}
    \item First, we have a constraint that we must use Next.js imposed by the supervisor for similar reasons above as to why Typescript will be used. Our team wants to choose libraries that facilitate the development of a web application, as that is the main goal of our project. We are looking for libraries that help with server-side rendering, static site generation, and routing. We also need libraries that will help us sort through the old financial information as part of this project involves reading and parsing the old Excel documents that were used to keep track of financials for clubs in the past.
  \end{itemize}
  
  \item \textbf{Pre-trained models and Machine Learning Technologies:}
  \begin{itemize}
      \item TensorFlow or PyTorch: Frameworks for utilizing pre-trained machine learning models for the machine learning based expense categorization stretch goal.
  \end{itemize}
  
  \item \textbf{Specific linter tool:}
  \begin{itemize}
    \item The team will pick a linter tool to ensure code quality and consistency that will help to identify and fix problems in the codebase. It helps identify and fix problems in TypeScript code.
  \end{itemize}
  
  \item \textbf{Specific unit testing framework:}
  \begin{itemize}
    \item  A unit testing framework such as Jest will be chosen to test the individual components, ensuring that each part of the application functions correctly.
  \end{itemize}
  
  \item \textbf{Investigation of code coverage measuring tools:}
  \begin{itemize}
    \item Code coverage tools will be used to measure the percent of the codebase tested, showing the team how to improve test coverage and code quality. This may be potentially done with Jest and GitHub actions. This will help ensure  that all parts of the codebase are tested.
  \end{itemize}
  
  \item \textbf{Specific plans for Continuous Integration (CI):}
  \begin{itemize}
    \item Continuous Integration (CI) tools will be implemented to automate the testing and deployment process. This will help ensure that code changes are continuously integrated and tested. GitHub Actions has most of the features needed for our development as it automates the testing and deployment process.
  \end{itemize}
  
  \item \textbf{Specific performance measuring tools:}
  \begin{itemize}
    \item While not likely that we will need specific performance measuring tools for a web application, as this is not quite a real time system, we still want the tool to be responsive, fast and easy to use for all users. For this reason we will mostly be basing our performance tests off how long pages take to load on different hardware, and if needed we will utilize other tools like Lighthouse to assess the efficiency and speed of the application, identifying areas for optimization.
  \end{itemize}
  
  \item \textbf{Tools you will likely be using:}
  \begin{itemize}
    \item We will choose development tools and editors to increase productivity and streamline the development process. We want good "return on investment" for our time to implement and learn how to use any tools in this project. Thus, we will mainly be using tools already used by our team members. This includes Visual Studio Code: This code editor will be used for development. It offers a wide range of extensions and integrations that enhance productivity. We will also extensively be using GitHub and GitHub projects as part of this effort for version control.
  \end{itemize}
\end{itemize}

\section{Coding Standard}

We will adopt the following coding standards to ensure code quality and maintainability:

\begin{itemize}
    \item \textbf{Language-Specific Guidelines:}
    \begin{itemize}
        \item \textbf{TypeScript}: Follow official TypeScript guidelines to ensure type safety and consistency.
    \end{itemize}

    \item \textbf{Code Formatting:}
    \begin{itemize}
        \item Use Prettier for consistent code formatting through the codebase.
    \end{itemize}

    \item \textbf{Commenting and Documentation:}
    \begin{itemize}
        \item Add inline comments for logic that is complex and need explanation. This will help to improve code readability.
    \end{itemize}

    \item \textbf{Version Control:}
    \begin{itemize}
        \item Follow the Git branching model for feature development and bug fixes.
        \item Use descriptive commit messages, this can be done by following the \href{https://www.conventionalcommits.org/en/v1.0.0/}{Conventional Commits} specification.
    \end{itemize}

    \item \textbf{Code Reviews:}
    \begin{itemize}
        \item Require peer reviews for all code changes to ensure quality and adherence to standards.
    \end{itemize}

    \item \textbf{Testing:}
    \begin{itemize}
        \item Write unit tests for each major feature using a set testing framework.
        \item Monitor code coverage to ensure an adequate amount of code coverage has been reached (currently set to 90\%).
    \end{itemize}
    \item \textbf{Variable Naming:}
  \begin{itemize}
      \item Use camelCase for variable names to ensure consistency and readability. For example, use \texttt{totalAmount} instead of \texttt{total\_amount}.
  \end{itemize}
\end{itemize}

\newpage{}

\section*{Appendix --- Reflection}

This appendix was discussed and completed as a group.

\begin{enumerate}
    \item Why is it important to create a development plan prior to starting the
    project (answered as a group)?
        It is important to create a development plan prior to the project 
    because it can be used to guide us during our project. A development plan can ensure
    we stay on track and meet specific criteria/checkpoints throughout the timeline 
    of a project. It also reduces confusion since you will know generally what your agenda 
    will look like for upcoming weeks. 
    \item In your opinion, what are the advantages and disadvantages of using
    CI/CD (answered as a group)?
    Using CI/CD has many advantages such as automating the integration, testing, and deployment 
    of code. This saves time by alerting us on issues early on and allows us to more confidently 
    write code knowing issues will be caught. It essentially allows us to always have our project 
    in a working state. A disadvantage to CI/CD comes from the time needed to setup and maintain it. 
    This can take development time away from working on the project itself. Some form of computational 
    costs will also be included to verify that the code works. 
    \item What disagreements did your group have in this deliverable, if any,
    and how did you resolve them (answered as a group)?
    We had disagreements with team member roles, some students have a larger
    course load this semester, so to keep it reasonable we gave people with less courses this 
    semester slightly more work to do (but we will try to keep everything even if possible). We resolved 
    this by evaluating all of our course work and trying to remain fair to all teammates, we planned and talked 
    the course through and how we would plan it. So we decided to compromise on certain parts and were able to split
    the work in a way the whole group was happy with and in a way that is fair to the project itself (All group member will do at least "some" work, where some is defined as an amount of work where all group members feel like the workload was distributed evenly).
\end{enumerate}

\newpage{}

\section*{Appendix --- Team Charter}

\subsection*{External Goals}
Our team’s external goals for this project are to further develop our skills as software engineers and gain hands-on experience in real-world project management and development. We aim to build a strong foundation of practical skills that we can confidently showcase in our resumes and discuss in interviews. We want to use this project to show our abilities when demonstrating our expertise and contributions to future employers.

\subsection*{Attendance}

\subsubsection*{Expectations}
All team members are expected to attend every meeting on time. If someone is delayed, they must notify the group in advance. If a member needs to leave early or miss a meeting for a valid reason, they are required to update the team on their progress via text and stay informed about the work and progress of others.

\subsubsection*{Acceptable Excuse}
Valid reasons for missing a meeting or deadline include family emergencies or severe illness that prevents the individual from working. Excuses like having a midterm or an upcoming assignment will not be considered acceptable.

\subsubsection*{In Case of Emergency}
In the event of an emergency, the work that was initially assigned to the affected team member will be equally distributed amongst the others to be completed. The team member that was unable to complete their work last time will have a slight increase in work for the next deliverable to make up.


\subsection*{Accountability and Teamwork}

\subsubsection*{Quality} 
Team members are to be expected to have questions prepared for team meetings as well as a detailed understanding of what they are currently working on and what they have completed. This information is crucial as it will be required to present each team meeting. The quality of the deliverables that members bring to the team should be detailed and well done. The team is expected to look at the rubric while completing their portions for the deliverables.


\subsubsection*{Attitude}
All team members are expected to maintain a welcoming and open-minded attitude towards all ideas and suggestions. While discussions on the effectiveness of an idea are encouraged, no idea should be dismissed outright. Everyone is expected to collaborate and contribute to the task at hand. If conflicts arise, members involved are expected to communicate their concerns openly and work together to find a resolution as quickly as possible.

\subsubsection*{Stay on Track}
To keep the team on track, we will hold weekly meetings to review current progress and outline upcoming tasks. Each member will provide a brief summary of their contributions for the week. Members who fall behind compared to their peers will receive warnings and be encouraged to meet the group’s standards. Those who do not contribute at all will be reported to the professor or TA.

We will set a target number of commits for each team member. At the end of the month, the member with the highest number of commits will be rewarded with a free meal of their choice.
\subsubsection*{Team Building}
We will have bi-weekly meetups where we go for food or do an activity together.

\subsubsection*{Decision Making} 
All decisions will be made through a voting process. If a member strongly disagrees with the outcome, they must present their concerns and explain why an alternative option might be preferable. A revote can then be conducted based on this new information.

\end{document}